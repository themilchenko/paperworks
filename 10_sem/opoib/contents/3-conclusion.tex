\conclusion

В заключение следует отметить, что интенсивное развитие цифровых технологий невозможно
без эффективного правового регулирования в сфере интеллектуальной собственности.
Нарушения авторских прав в сети Интернет представляют собой серьёзную проблему,
затрагивающую как правовые, так и социально-экономические аспекты информационного
общества.

В ходе работы были рассмотрены ключевые понятия, нормативно-правовая база, а также
современные проблемы защиты авторских прав в цифровой среде. На основе проведённого
анализа можно выделить следующие рекомендации для правообладателей:

\begin{enumerate}
    \item Владение знаниями о своих правах и умение применять их на практике — важнейший
          элемент правовой защиты.
    \item При необходимости автору следует привлекать квалифицированных специалистов в
          области авторского и информационного права.
    \item Учитывая высокий уровень цифрового пиратства, правообладателям необходимо
          осознанно подходить к размещению своих произведений в открытом доступе,
          оценивая риски их несанкционированного использования.
\end{enumerate}

Таким образом, можно сделать вывод, что меры защиты авторских прав в сети Интернет в
целом являются достаточными по своей правовой конструкции. Однако реализация этих норм
на практике сталкивается с рядом проблем: недостаточной судебной практикой, сложностью
сбора доказательств, а также низкой правовой культурой пользователей.

Законодательство России в области защиты авторских прав в цифровой среде требует
совершенствования. Для эффективного противодействия нарушениям необходима не только
адаптация нормативно-правовой базы к новым вызовам,
