\section{Правовое регулирование авторских прав в сети Интернет}

Правоотношения в сфере защиты авторских прав регламентируются нормами части четвертой
Гражданского кодекса РФ, Гражданского процессуального кодекса РФ \cite{gpkRF},
Федерального закона от 27.07.2006 № 149-ФЗ «Об информации, информационных технологиях и о
защите информации» \cite{fz149} и некоторых других нормативных-правовых актов.

Глобальный характер Интернета обуславливает необходимость международного сотрудничества
для эффективной защиты авторских прав. Различия в правовых системах разных стран создают
проблемы при преследовании нарушителей, действующих через границы. Гармонизация
национальных законов с международными конвенциями (например, положениями Бернской
конвенции) и активное взаимодействие правоохранительных органов способствуют
преодолению этих трудностей.

Одной из ключевых проблем является сложность доказательства факта нарушения авторских
прав в условиях цифровой среды. При массовом и анонимном распространении контента
традиционные методы сбора доказательств оказываются недостаточно эффективными. Здесь
применяются современные технологии, такие как системы цифровых водяных знаков, метаданные
и автоматизированный мониторинг, что требует значительных ресурсов и создания единых
стандартов их применения.
