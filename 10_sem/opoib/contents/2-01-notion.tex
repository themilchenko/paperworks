\structure{ОСНОВНАЯ ЧАСТЬ}

\section{Понятие авторского права и его особенности в цифровой среде}

Авторское право — это форма защиты, предоставляемая законодательством страны авторам «оригинальных авторских произведений», при которой авторы получают определенные исключительные права на свои произведения \cite{boichenko}.

В соответствии с Гражданским кодексом Российской Федерации (часть 4) \cite{gkRF4},
авторские права возникают автоматически в момент создания произведения, независимо от его
публикации. Aвторское произведение может быть передано неограниченному кругу лиц во всем
мире посредством сети интернет, что существенно усложняет контроль за его
распространением.

Одной из ключевых особенностей авторского права является его применение к произведениям,
выраженным в любой объективной форме –- будь то текст, изображение, музыка или
программный код. Это означает, что защита распространяется не только на опубликованные
произведения, но и на неопубликованные, что подтверждается нормами гражданского
законодательства.

Главная особенность цифровой среды -- это стирание географических границ и техническая
легкость копирования. Произведение, размещенное в сети, мгновенно становится доступным
неограниченному кругу лиц по всему миру, при этом контроль за его дальнейшим
распространением практически невозможен. Как отмечают исследователи, современные
технологии распространения контента развиваются настолько быстро, что законодательство
просто не успевает адекватно реагировать на все возникающие вызовы.
