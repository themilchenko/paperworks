\structure{ВВЕДЕНИЕ}

В современном цифровом обществе Интернет стал неотъемлемой частью повседневной жизни,
предоставляя беспрецедентные возможности для создания, распространения и потребления
контента. Однако, вместе с этими возможностями возросло и количество случаев нарушения
авторских прав, что делает изучение правовых аспектов защиты интеллектуальной
собственности в сети Интернет особенно актуальным.

Нарушения авторских прав в Интернете проявляются в различных формах: от незаконного
копирования и распространения текстов, изображений, музыки и видео до использования
чужих произведений без соответствующего разрешения. Такие действия не только ущемляют
права создателей контента, но и наносят значительный экономический ущерб
правообладателям.

Цель данного реферата – рассмотреть правовые аспекты нарушения авторских прав в сети Интернет, выявить особенности регулирования в цифровой среде, проанализировать основные проблемы и определить возможные способы защиты.
