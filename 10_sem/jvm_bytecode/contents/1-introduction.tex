\structure{ВВЕДЕНИЕ}

Java является одним из наиболее популярных и востребованных языков программирования в
мире. Его уникальность заключается в использовании промежуточного представления программ,
известного как байткод (bytecode). Это позволяет исполнять Java-программы на различных
платформах без перекомпиляции исходного кода. Центральным компонентом, отвечающим за
выполнение такого байткода, является виртуальная машина Java (Java Virtual Machine, JVM).

Актуальность изучения байткода JVM обусловлена его важной ролью в обеспечении
кроссплатформенности и безопасности приложений, а также возможностью оптимизации и
анализа программного обеспечения на низком уровне. Понимание того, как устроен и
функционирует байткод JVM, дает разработчику существенные преимущества в написании
эффективного, безопасного и производительного кода.

Целью данной работы является изучение особенностей и структуры байткода Java, а также
принципов его исполнения виртуальной машиной JVM. В задачи работы входит рассмотрение
архитектуры JVM, описание процесса компиляции и исполнения байткода, а также анализ
практического применения и инструментов, связанных с байткодом Java.
