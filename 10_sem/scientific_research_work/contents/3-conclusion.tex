\conclusion

В ходе проектирования и реализации распределённого key-value хранилища на основе алгоритма Raft были достигнуты основные цели:
отказоустойчивость, согласованность данных и возможность горизонтального масштабирования системы. Хранилище эффективно справляется
с задачами репликации, синхронизации данных и восстановления после сбоев, обеспечивая высокую доступность и целостность данных при
отказах узлов.

Использование алгоритма консенсуса Raft позволяет гарантировать, что данные всегда остаются согласованными в распределённой среде,
даже в случае сбоя одного или нескольких узлов. Разработанная система легко масштабируется и может добавлять новые узлы без потери
данных, обеспечивая возможность добавления, удаления и чтения данных с разных узлов кластера.

Пользовательский API предоставляет удобные инструменты для взаимодействия с хранилищем, позволяя пользователю эффективно управлять
данными через простые HTTP запросы. Утилита для управления кластером упрощает процесс настройки и мониторинга системы, позволяя
запускать и контролировать состояние всех узлов кластера с помощью нескольких команд.

В рамках тестирования отказоустойчивости было проверено, что система корректно восстанавливает работоспособность при падении лидера,
а данные не теряются. Репликация и синхронизация данных между узлами подтверждают стабильность работы кластера в условиях реальных
сбоев и изменений конфигурации.

В целом, работа демонстрирует успешную реализацию ключевых принципов распределённых систем с использованием алгоритма Raft,
обеспечивая высокую доступность и согласованность данных в распределённом хранилище.
