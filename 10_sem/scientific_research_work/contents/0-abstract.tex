\abstract

Ключевые слова: алгоритм Raft, распределённые системы, key-value хранилище, отказоустойчивость, репликация, Go.

Целью проведённой научно-исследовательской работы является проектирование и разработка отказоустойчивого распределённого key-value
хранилища, использующего алгоритм Raft для достижения консенсуса между узлами кластера.

В процессе работы выполнено исследование существующих реализаций Raft (\textit{etcd/raft}, \textit{HashiCorp Raft}, \textit{Braft},
\textit{Apache Ratis}, \textit{raft-rs}) и проанализированы их преимущества и ограничения. На основании критериев надёжности,
активности сообщества, гибкости интеграции и производительности выбрана библиотека \textit{HashiCorp Raft} на языке Go для
последующей реализации механизма консенсуса. Разработана архитектура хранилища в модели "лидер—последователь", определён HTTP/RPC
API, формат конфигурационных файлов и структура каталогов данных.

Реализованы:
\begin{itemize}
    \item узел хранилища с поддержкой операций \textit{GET / POST / DELETE}, журнал Raft, снапшоты и механизм восстановления;
    \item кластерный менеджер \textit{clusterctl}, обеспечивающий запуск, остановку, мониторинг и динамическое масштабирование
          кластера;
    \item процедуры репликации логов, смены лидера, безопасного добавления и удаления узлов.
\end{itemize}

В ходе тестирования подтверждена устойчивость системы к отказам: при остановке лидирующего узла кластер автоматически выбирает
нового лидера, а согласованность данных сохраняется. Производственные сценарии репликации и восстановления продемонстрированы на
кластере из 5 узлов.

В результате работы создано работоспособное распределённое key-value хранилище, пригодное для применения в сервисах,
где критичны высокая доступность и целостность данных. Предложены направления дальнейшего развития: расширение API, интеграция
с системами мониторинга, внедрение шифрования и поддержка сложных типов данных.
