\introduction

В современном мире обработки данных отказоустойчивость и надежность распределенных
систем являются ключевыми требованиями для множества приложений, начиная от облачных
сервисов и заканчивая базами данных и системами хранения информации. Одним из
фундаментальных механизмов, обеспечивающих согласованность данных в распределенных
системах, является алгоритм консенсуса. Среди различных алгоритмов консенсуса, таких
как Paxos, Viewstamped Replication и другие, алгоритм Raft выделяется своей простотой
понимания, формальной доказанностью корректности и эффективностью работы.

Raft представляет собой алгоритм консенсуса, предназначенный для управления
распределенными системами и поддержания согласованности реплицированных логов.
Он был предложен Диего Онгаро и Джоном Оустером в 2014 году как более понятная
альтернатива Paxos, что делает его привлекательным для реализации в практических
системах.

В рамках данной научно-исследовательской работы рассматривается задача создания
отказоустойчивого распределенного хранилища данных на основе алгоритма Raft. В первой
части работы проводится анализ существующих реализаций Raft, обосновывается выбор одной
из библиотек для репликации данных, а также разрабатывается архитектура хранилища с
учетом интеграции данной библиотеки. Кроме того, рассматриваются механизмы взаимодействия
узлов для достижения консенсуса и обеспечения надежности системы.

Во второй части работы будет осуществлена реализация основных операций хранилища,
включая механизм репликации данных, а также проведено тестирование системы на предмет
отказоустойчивости.

Таким образом, цель данной работы заключается в проектировании и разработке
распределенного хранилища данных, использующего алгоритм Raft для обеспечения
согласованности и высокой доступности данных.

