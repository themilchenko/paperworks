\structure{ОСНОВНАЯ ЧАСТЬ}

Перед разработкой распределенного хранилища данных на основе алгоритма Raft необходимо
провести анализ существующих реализаций данного алгоритма. В настоящее время существует
множество библиотек и фреймворков, реализующих Raft, каждая из которых обладает своими
преимуществами и особенностями. Выбор подходящей библиотеки играет ключевую роль в
построении отказоустойчивой системы, поскольку от него зависит производительность,
масштабируемость и надежность репликации данных.

Анализ существующих решений позволит выявить их сильные и слабые стороны, определить
критерии выбора наиболее подходящей библиотеки, а также учесть возможные ограничения и
особенности интеграции в разрабатываемую систему. В данной части работы будут
рассмотрены наиболее популярные и широко используемые реализации Raft, их архитектура,
функциональные возможности и сферы применения. На основе проведенного анализа будет
обоснован выбор конкретной библиотеки для дальнейшего использования в проекте.

\section{Анализ существующих реализаций алгоритма}

Для анализа были использованы данные с официального ресурса Raft GitHub, который
содержит список наиболее популярных и активно поддерживаемых реализаций алгоритма на
разных языках программирования. Среди них можно выделить:

\subsection{etcd/raft}

Библиотека Raft реализует протокол консенсуса Raft для поддержания согласованного
состояния распределённого конечного автомата через реплицируемый журнал. Она
используется в отказоустойчивых системах, таких как etcd, Kubernetes и CockroachDB.

В отличие от монолитных реализаций, библиотека обеспечивает только алгоритм консенсуса,
оставляя сетевое взаимодействие и хранение данных на усмотрение пользователя. Это
повышает её гибкость, детерминированность и производительность.

Функционал включает выборы лидера, репликацию журнала, изменения состава кластера и
оптимизированные запросы на чтение. Raft представлен в виде детерминированного
конечного автомата, что упрощает моделирование и тестирование.

\subsection{HashiCorp Raft}

HashiCorp Raft — это мощная библиотека на Go, реализующая алгоритм консенсуса Raft. Она
предназначена для управления реплицируемыми журналами и согласованными конечными
автоматами, что делает её ключевым инструментом для построения распределённых систем с
высокой отказоустойчивостью и согласованностью (CP по CAP-теореме).

Raft использует три роли узлов — лидер, кандидат и последователь, — обеспечивая надёжный
механизм выбора лидера и согласованной репликации данных. Журнал логов может бесконечно
расти, но библиотека поддерживает механизм снапшотов, позволяющий автоматически удалять
старые записи без потери согласованности. Кроме того, Raft позволяет динамически
обновлять состав кластера, добавляя или удаляя узлы при наличии кворума.

Производительность Raft сопоставима с Paxos: при стабильном лидерстве новый лог
подтверждается за один раунд сетевого взаимодействия с большинством узлов. При этом
кластер из 3 узлов выдерживает отказ одного, а из 5 — до двух. Для хранения данных
HashiCorp предлагает два бэкенда: "raft-mdb" (основной) и "raft-boltdb" (чистая
Go-реализация).

\subsection{Braft}

Braft — это промышленная реализация алгоритма консенсуса Raft на C++ от Baidu,
ориентированная на высокую нагрузку и низкие задержки. Она построена на базе "brpc",
что делает её эффективным решением для отказоустойчивых распределённых систем.  

Эта библиотека широко применяется внутри Baidu для построения различных высокодоступных
сервисов: хранилищ (KV, блоковых, файловых, объектных), распределённых SQL-систем,
модулей управления метаданными и сервисов блокировок.  

Основной упор в разработке Braft сделан на производительность и понятность кода, что
позволяет инженерам Baidu самостоятельно создавать распределённые системы без глубокой
экспертизы в алгоритмах консенсуса. В отличие от других реализаций Raft, Braft
оптимизирована для минимизации задержек и высокой пропускной способности, что делает
её особенно подходящей для сценариев с высокой нагрузкой.  

Сборка библиотеки требует установки brpc, а установка возможна через "vcpkg", что
облегчает интеграцию. Braft также предоставляет обширную документацию, включая
информацию о модели репликации, поддержке других протоколов консенсуса (Paxos, ZAB, QJM)
и бенчмарки производительности. Это делает её удобным инструментом для разработки
отказоустойчивых распределённых решений.

\subsection{Ratis}

Apache Ratis — это Java-библиотека, реализующая алгоритм консенсуса Raft,
предназначенная для управления реплицируемым журналом. В отличие от Paxos, Raft
предлагает более понятную структуру, что делает его удобной основой для построения
практических систем.  

Основная цель Apache Ratis — предоставить гибкую и удобную встраиваемую реализацию Raft,
которая может использоваться любыми системами, нуждающимися в реплицируемом логе.
Она поддерживает модульность, позволяя разработчикам легко интегрировать собственные
реализации конечных автоматов, журналов Raft, RPC-коммуникаций и механизмов метрик.  

Одной из ключевых особенностей Ratis является ориентация на высокую пропускную
способность при записи данных, что делает её подходящим выбором не только для
классических систем консенсуса, но и для более общих задач репликации данных.
Это особенно важно для сценариев, требующих быстрого и надёжного распространения
обновлений между узлами распределённой системы.  

Проект активно развивается в рамках Apache, обеспечивая поддержку современных требований
к отказоустойчивым распределённым сервисам.

\subsection{raft-rs}

raft-rs — это реализация алгоритма консенсуса Raft на Rust, созданная для управления
реплицируемым логом в распределённых системах. Этот crate предлагает мощный и гибкий
базовый модуль консенсуса, который можно адаптировать под разные сценарии, но требует
реализации собственных компонентов для хранения логов, управления конечными автоматами
и сетевого взаимодействия.

Одним из ключевых преимуществ raft-rs является его производительность и модульность.
Он поддерживает работу с rust-protobuf и Prost для кодирования rpc-сообщений, что
делает интеграцию с различными системами более удобной. В экосистеме Rust этот crate
широко используется в таких проектах, как TiKV — распределённая транзакционная база
данных.

Проект активно развивается, предлагая инструменты для тестирования (cargo test),
форматирования (cargo fmt) и анализа (cargo clippy), а также бенчмаркинг через
Criterion. Благодаря этому raft-rs не только удобен для использования, но и
предоставляет разработчикам возможности для оценки и оптимизации производительности их
решений.

\subsection{Обоснование выбора языка программирования}

При выборе языка программирования для реализации отказоустойчивого распределенного
хранилища важно учитывать такие факторы, как производительность, удобство работы с
конкурентностью, а также наличие проверенных решений для репликации данных. В этом
контексте Go является оптимальным выбором, поскольку он изначально разрабатывался для
высоконагруженных распределенных систем и предлагает мощные механизмы работы с
многопоточностью через goroutines и каналы. Кроме того, язык имеет встроенную поддержку
профилирования и мониторинга, что критически важно для отказоустойчивых систем.

\subsection{Обоснование выбора реализации алгоритма}

После анализа существующих реализаций алгоритма Raft на разных языках программирования и
выбора Go в качестве основного языка разработки, следующим этапом является выбор
конкретной библиотеки для механизма консенсуса и репликации данных.

Основными критериями выбора являются:

\begin{itemize}
    \item Надежность и проверенность - библиотека должна успешно применяться в
    промышленных системах.
    \item Активная поддержка и развитие – регулярные обновления, исправления ошибок и
    качественная документация.
    \item Гибкость интеграции - удобные API для работы с реплицируемыми логами и
    кластерным управлением.
    \item Производительность – эффективная обработка логов и минимальные накладные
    расходы на консенсус.
\end{itemize}

На основе этих критериев в данной работе будет использоваться \textbf{HashiCorp Raft} -
легковесная и гибкая реализация Raft, разработанная компанией HashiCorp. Она отличается
минимальными зависимостями и удобством интеграции, что делает её оптимальным выбором для
отказоустойчивых распределённых систем.  

Библиотека поддерживает:

\begin{enumerate}
    \item Динамическое управление кластером – безопасное добавление и удаление узлов
    без нарушенияконсистентности.
    \item Гибкость хранения логов – возможность использовать различные механизмы
    хранения, включая BoltDB и MDB.
    \item Высокую производительность – механизм снапшотов и эффективная репликация
    обеспечивают низкие задержки.
\end{enumerate}

HashiCorp Raft применяется в таких продуктах, как Consul и Nomad, что подтверждает её
стабильность и эффективность. Таким образом, данная библиотека удовлетворяет всем
требованиям, предъявляемым к отказоустойчивой системе репликации данных, и будет
использоваться в данной работе в качестве основы для механизма консенсуса.
