\section{ВВЕДЕНИЕ}

Создание систем защищённого общения в открытых средах, таких как Интернет, представляет собой сложную задачу, существенно
зависящую от используемых криптографических протоколов. Вместе с тем, на подобные системы можно осуществлять серьёзные атаки,
эксплуатируя лишь внутренние уязвимости самих криптографических протоколов. Уязвимости криптографических протоколов легко
пропустить на стадии проектирования, так как злоумышленник может контролировать сеть передачи данных и комбинировать сообщения,
относящиеся к разным сеансам работы протокола. Кроме того, участники взаимодействия могут вести себя недобросовестно. Необходимость
строгого формального анализа и наличия надёжных инструментов для анализа криптографических протоколов осознавалась
уже давно. Так называемая модель Долева–Яо (Dolev–Yao), берущая начало из работы Долева и Яо \cite{DolevYao1983}, стала
доминирующей формальной моделью анализа протоколов (см. обзор в \cite{Meadows2000}). Было предложено много процедур для автоматического
анализа криптографических протоколов в рамках модели Долева–Яо \cite{Amadio2002}, и на основе этих процедур разработаны многочисленные инструменты \cite{MillenShmatikov2001}, которые успешно применялись для обнаружения уязвимостей в опубликованных протоколах \cite{ClarkJacob1997}.

Большинство методов и инструментов делают упрощающее предположение о том, что криптографические алгоритмы являются совершенными
(предположение о совершенной криптографии). Иначе говоря, предполагается, что для извлечения открытого текста из зашифрованного
сообщения необходим именно ключ шифрования; без него никакая информация об открытом тексте не может быть извлечена. Также
предполагается, что зашифрованное сообщение может быть создано только при наличии соответствующего ключа и открытого текста. Однако
такая простая модель становится недостаточной при анализе многочисленных протоколов, в которых используются операторы с
алгебраическими свойствами, такие как исключающее ИЛИ (XOR) и модульная экспоненциация. Причина этого двояка: во-первых, без учёта
алгебраических свойств оператора протоколы могут не достигать целей безопасности даже без присутствия злоумышленника (в \cite{BoydMathuria2003} приводится много таких примеров). Например, базовый протокол Диффи–Хеллмана явно использует коммутативность операции
возведения в степень: $(g^a)^b = (g^b)^{a}$. Во-вторых, многие атаки основаны на эксплуатации именно алгебраических свойств
операторов и поэтому не могут быть выявлены в рамках модели совершенной криптографии. Например, рекурсивный протокол аутентификации
Булла и Отвея \cite{BullOtway1997} был доказан безопасным при предположении совершенных криптографических функций \cite{Paulson1997},
однако был впоследствии признан небезопасным при реализации на основе оператора XOR из-за его нильпотентности \cite{RyanSchneider1998}. В разделе 6 приводится ещё один пример — протокол A-GDH.2 (см. \cite{BoydMathuria2003}). Таким образом, при анализе
современных криптографических протоколов важно учитывать алгебраические свойства операторов, таких как XOR, экспоненциация
Диффи–Хеллмана и RSA-шифрование.

\textit{Вклад данной работы.} В этой работе мы показываем, что задача незащищённости протоколов, использующих экспоненциацию
Диффи–Хеллмана с произвольными произведениями в показателях, является NP-полной при анализе в ограниченном числе сеансов. Мы
демонстрируем, что наши модель протокола и модель злоумышленника достаточно выразительны, чтобы воспроизвести атаки, впервые
описанные Pereira \& Quisquater на примере протокола A‑GDH.2 \cite{PereiraQuisquater2001}. Аналогичный результат NP-полноты мы
получаем и для протоколов с коммутирующим шифрованием с открытым ключом (например, RSA с общим модулем). В качестве следствия наших
доказательств получаем, что задача вывода сообщения (derivation problem), то есть проверка возможности злоумышленника вывести
заданное сообщение из конечного набора известных ему сообщений, решается за детерминированное полиномиальное время как для
экспоненциации Диффи–Хеллмана, так и для коммутирующего шифрования.

Наши доказательства NP-полноты строятся в два этапа. Сначала мы расширяем стандартную модель злоумышленника Долева–Яо с помощью
обобщённых «правил-оракулов» и показываем, что общая задача незащищённости при таком расширении является NP-полной. Затем мы
конкретизируем эти правила для экспоненциации Диффи–Хеллмана и коммутирующего шифрования и показываем, что задачи вывода и анализа
безопасности протокола решаются за детерминированное и недетерминированное полиномиальное время соответственно.

\medskip
\textit{Связанные работы.} Первые исследования, рассматривающие алгебраические свойства операторов, включая их в модель Долева–Яо,
были выполнены Chevalier et al. \cite{ChevalierXOR2003} и Comon-Lundh \& Shmatikov \cite{ComonLundhShmatikov2003}, где модель
была расширена оператором XOR и его свойствами. Для экспоненциации Диффи–Хеллмана и коммутирующего шифрования, как показано здесь,
ситуация сложнее: в отличие от XOR, мы не имеем нильпотентного свойства и должны учитывать число вхождений элементов в произведении,
что требует решения уравнений над целыми числами.

Meadows \& Narendran \cite{MeadowsNarendran2002} разработали алгоритмы унификации для свойств криптографических систем на основе
Диффи–Хеллмана. Эти результаты полезны, но не решают более общую проблему незащищённости (см. также \cite{KapurNarendranWang2003}).

Pereira \& Quisquater \cite{PereiraQuisquater2004} предложили систематический метод анализа семейств протоколов, расширяющих схему
обмена ключами Диффи–Хеллмана до группового контекста. Хотя они обнаружили интересные атаки, основанные на алгебраических свойствах
экспоненциации, они не исследуют вопросы разрешимости и сложности.

Goubault-Larrecq et al. \cite{GoubaultLarrecq2005} создали систему верификации протоколов с модульной экспоненциацией на
фиксированном генераторе \(g\). Их метод основан на аппроксимациях и включает правило вывода, позволяющее злоумышленнику получить
\(g^{a\cdot b}\) из \(g^a\) и \(b\), но не учитывает обратимые операции и потому упускает реалистичные атаки.

Boreale \& Buscemi \cite{BorealeBuscemi2003} рассмотрели схожую задачу, но ввели априорную границу на число множителей в
произведениях, тогда как в нашей работе число множителей неограниченно; кроме того, они не дают результатов о сложности.

Millen \& Shmatikov \cite{MillenShmatikov2003} исследовали абелевы группы и применяли их к экспоненциации Диффи–Хеллмана, но не
предложили алгоритма решения и предполагали фиксированную основу в показателях. Shmatikov \cite{Shmatikov2004}, опираясь на результаты
Chevalier et al. \cite{ChevalierDH2003}, предложил процедуру для поиска атак в варианте нашей модели, но также не дал оценок
сложности. В отличие от нашего подхода, в его модели множители могли находиться вне показателей, а сами множители не начинаться с
операции экспоненциации, что является дополнительным ограничением.

\medskip
\textit{Структура работы.} В разделе 2 мы вводим нашу модель протокола и злоумышленника, включая упомянутые правила-оракулы. В
разделе 3 доказывается NP-полнота общей задачи незащищённости. В разделах 4 и 5 эти правила конкретизируются для экспоненциации
Диффи–Хеллмана и коммутирующего шифрования, и демонстрируется, что задачи вывода и анализа безопасности решаются за детерминированное
и недетерминированное полиномиальное время соответственно. В разделе 6 формализуется протокол A‑GDH.2 и приводится атака, впервые
описанная Pereira \& Quisquater \cite{PereiraQuisquater2001}. В разделе 7 наш метод применяется к протоколам с коммутирующим
шифрованием. Заключение содержится в разделе 8. Некоторые подробности доказательств вынесены в приложение.
