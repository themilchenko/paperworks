\conclusion

Мы показали, что задача проверки небезопасности для протоколов,
использующих возведение в степень Диффи–Хеллмана с произвольными
произведениями в показателях, является NP-полной и что в этой
модели задача вывода может быть решена за детерминированное
полиномиальное время.  
Кроме того, продемонстрировано, каким образом эти результаты
переносятся на протоколы, основанные на коммутативном
шифровании с открытым ключом.
