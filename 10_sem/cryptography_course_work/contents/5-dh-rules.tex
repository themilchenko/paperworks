\section{Правила DH допускают атаки с полиномиальным произведением показателей}

В этом разделе мы показываем, что задача \textsc{Insecure} NP-полна для нарушителя DH (см.~Теорему 5.23). В силу Теоремы 3.1, Предложения 4.7 и Предложения 4.8 остаётся показать, что правила DH действительно допускают атаки
с полиномиально ограниченным произведением показателей.
Для этого мы свяжем с минимальной атакой $(\pi,\sigma)$
подстановку $\sigma^{2}$ и линейную систему уравнений,
обладающую двумя свойствами:

\begin{enumerate}\itemsep0pt
\item[(i)] $\sigma^{Z}$ совпадает с $\sigma$, за исключением того,
      что все показатели произведения в $\sigma$
      заменены новыми целочисленными переменными;
\item[(ii)] $(\pi,\sigma')$ остаётся атакой для каждой подстановки $\sigma'$,  
      получаемой из $\sigma^{Z}$ подстановкой значений переменных,
      удовлетворяющих линейной системе.
\end{enumerate}
Размер этой системы можно ограничить полиномом от размера протокола,
а значит и размер её решений тоже ограничивается полиномиально
(см.\ \cite{Bockmayr2001}).
Тем самым мы получаем атаку с полиномиально
ограниченными произведениями показателей
(см.\ Предложение~5.22).

В следующем подразделе мы формально определим сообщения,
в которых показатели могут быть линейными выражениями.
Перед тем как перейти к подробному изложению,
в §\,5.2 мы дадим интуитивное объяснение доказательства
Предложения~5.22.
Полное доказательство приведено в 5.3–5.7.

\subsection{Открытые сообщения и системы уравнений}

В этом разделе мы вводим открытые сообщения и произведения, отображения
оценки, системы уравнений, а также различные меры их размера.

\begin{definition}
Пусть $Z$ — множество переменных.
Обозначим через $\mathcal M=\mathcal M(Z)$
множество \emph{открытых сообщений} над $Z$,
через $\mathcal P=\mathcal P(Z)$ —
множество \emph{открытых произведений} над $Z$,
а через $\mathcal L_{\text{exp}} = \mathcal L_{\text{exp}}(Z)$ —
множество \emph{линейных выражений} над $Z$.
Эти множества задаются следующей грамматикой:

\[
\begin{aligned}
  \mathcal M\ ::=&\ A \;\bigl|\;
                   \langle \mathcal M,\,\mathcal M\rangle \;\bigl|\;
                   \{\,\mathcal M\}^{s}_{\mathcal M}\;\bigl|\;
                   \{\,\mathcal M\}^{p}_{K}\;\bigl|\;
                   Exp(\mathcal M,\mathcal P),\\[2pt]
  \mathcal P\ ::=&\ \mathcal M^{\mathcal L_{\text{exp}}}\;\bigl|\;
                   \mathcal M^{\mathcal L_{\text{exp}}}\!\cdot\mathcal P,\\[2pt]
  \mathcal L_{\text{exp}}\ ::=&\ \mathbb Z \;\bigl|\;
                       Z \;\bigl|\;
                       \mathcal L_{\text{exp}}+\mathcal L_{\text{exp}} \;\bigl|\;
                       \mathbb Z\!\cdot\!\mathcal L_{\text{exp}}.
\end{aligned}
\]
\end{definition}

\noindent
Размер~$|e|$ линейного выражения~$e$ — это число символов,
необходимых для записи~$e$ (целые кодируются двоично).
Говорим, что $e$ и $e'$ \emph{равны},
если они эквивалентны по ассоциативности и коммутативности сложения
(сокращённо \text{AC\textsubscript{+}}).
Для множества линейных выражений $S$ и выражения $e$ мы говорим, что $e$ принадлежит $S$,
если класс эквивалентности по модулю $\text{AC\textsubscript{+}}(e)$ для $e$ является одним из
классов эквивалентности классов, индуцированных по $S$. Тогда $e\in S$ означает,
что $\text{AC\textsubscript{+}}(e)$ совпадает с одним из классов, индуцированных~$S$.
Отношение включения между наборами линейных выражений определяется аналогично.

\smallskip
Для открытого сообщения или произведения~$t$
обозначим через $\mathcal L_{\text{exp}}(t)$
множество линейных выражений, встречающихся в~$t$.
Пусть $\mathcal S(t)$ — множество подтермов~$t$,
а $|t|=\operatorname{Card}(\mathcal S(t))$ — их количество.
Следуя принятой ранее нотации, положим
\[
  S_{\text{ext}}(t)=\mathcal S(t)\cup\{\,M\mid \operatorname{Exp}(u,M)\in \mathcal S(t)\},
  \qquad
  |t|_{\text{ext}}=\operatorname{Card}(\mathcal S_{\text{ext}}(t)).
\]
Также определим
$|t|_{\text{exp}}=0$, если $t$ не является произведением,
и $|t|_{\text{exp}}=|e_{1}|+\dots+|e_{n}|$, если
$t = e_{1}^{\,1}\!\dots e_{n}^{\,n}$.
Для конечного множества открытых сообщений или произведений $E$
обозначим $|E|_{\text{exp}}=\sum_{s\in E}|s|_{\text{exp}}$.
Наконец,
\[
  \Vert t\Vert_{\text{exp}} = |S(t)|_{\text{exp}},\qquad
  \Vert t\Vert = |t| + \Vert t\Vert_{\text{exp}},\qquad
  \Vert t\Vert_{\text{ext}} = |t|_{\text{ext}} + \Vert t\Vert_{\text{exp}}.
\]

\medskip
\textsc{Лемма 5.2.}
Для любого открытого сообщения или произведения~$t$ выполняется
\[
  |t|_{\text{ext}}\ \le\ 2\cdot |t|,
  \qquad\text{и, следовательно,}\qquad
  \Vert t\Vert_{\text{ext}}\ \le\ 2\cdot\Vert t\Vert.
\]

Заметим, что определения $|\cdot|$, $\Vert\cdot\Vert_{\text{exp}}$ и $\Vert\cdot\Vert$
для открытых сообщений и произведений совпадают с соответствующими
определениями для (закрытых) сообщений.

Приведённые выше определения и меры для открытых сообщений и произведений
естественным образом переносятся на множества открытых сообщений,
открытых произведений и т.\,д.

Назовём отображение $\beta : Z \to \mathbb Z$ \emph{отображением оценки}.
Значение $\beta(e)\in\mathbb Z$ линейного выражения $e$
определяется обычным образом.
Очевидно, $\beta$ естественным образом распространяется
на открытые сообщения, открытые произведения,
а также на их множества.

В дальнейшем в этом подразделе
$\beta$ всегда фиксируется как отображение оценки из $Z$ в $\mathbb Z$.

\medskip
\emph{Линейной системой уравнений} $\mathcal E$ (над $Z$) называется
конечное множество равенств вида $e = e'$, где
$e$ и $e'$ — линейные выражения над $Z$.
Её \emph{размер}
\(
  \lvert\mathcal E\rvert
  = \sum_{\,e=e'\in\mathcal E}\bigl(\lvert e\rvert+\lvert e'\rvert\bigr).
\)
Отображение оценки $\beta$ является \emph{решением} системы
(обозначаем $\beta\models\mathcal E$),
если $\beta(e)=\beta(e')$ для каждого уравнения $e=e'\in\mathcal E$.
\(
  L_{\text{exp}}(\mathcal E)=
  \bigl\{\,e \mid e=e'\ \text{или}\ e'=e\in\mathcal E\bigr\}.
\)
Положим
\(
  R_{\mathcal E} =
  \bigl\{\, (e,e') \mid e=e' \in \mathcal E \bigr\}
  \subseteq \mathcal L_{\text{exp}}(\mathcal E)\times \mathcal L_{\text{exp}}(\mathcal E),
  \qquad
  R_{\mathcal E}^{*}\text{ — рефлексивно-транзитивное замыкание }R_{\mathcal E}.
\)
Пишем $\mathcal E \cong \mathcal E'$, если
$R_{\mathcal E}^{*}=R_{\mathcal E'}^{*}$,
и $\mathcal E\subseteq\mathcal E'$, если
$R_{\mathcal E}^{*}\subseteq R_{\mathcal E'}^{*}$.
Таким образом, линейные уравнения рассматриваются
с точностью до рефлексивности и транзитивности равенства.
Напомним также, что линейные выражения эквивалентны по правилу AC\textsubscript{+}.

\subsection{Обзор доказательства Предложения 5.22}

Ниже приведён неформальный «вид сверху» на доказательство
Предложения 5.22, позволяющего полиномиально ограничить величины
показателей произведения в атаках.

\medskip
Ключевым элементом является \textsc{Лемма 5.21}, в которой говорится
следующее.  
Пусть $t,t_{1},\dots,t_{n}$ — открытые сообщения и
$\beta$ — отображение оценки, такое что
\(
  \ulcorner\beta(t)\urcorner\;\in\;
  \operatorname{forge}
  \bigl(\ulcorner\beta(t_{1})\urcorner,\dots,
        \ulcorner\beta(t_{n})\urcorner\bigr).
\)
Тогда существует расширение $\beta$
(обозначаем его тем же символом) и система линейных уравнений
$\mathcal E$, удовлетворяющие

\begin{enumerate}\itemsep0pt
\item[(1)] $\beta\models\mathcal E$;
\item[(2)] для всякого $\beta'\models\mathcal E$
\[
  \ulcorner\beta'(t)\urcorner\in
  \operatorname{forge}
  \bigl(\ulcorner\beta'(t_{1})\urcorner,\dots,
        \ulcorner\beta'(t_{n})\urcorner\bigr);
\]
\item[(3)] размер $\mathcal E$ ограничен полиномом
      от $\Vert t_{1},\dots,t_{n},t\Vert_{\text{ext}}$.
\end{enumerate}

Доказательство этой леммы довольно громоздко. Основная идея состоит в том, чтобы заменить сообщения в выводе
\(D\) от \(\ulcorner\beta(t_{1})\urcorner,\dots,\ulcorner\beta(t_{n})\urcorner\) к \(\ulcorner\beta(t)\urcorner\)
на \emph{открытые} сообщения, совпадающие с исходными, за исключением показателей степеней произведения.
Более точно, каждое \(t_{i}\) заменяется на открытое сообщение \(t'_{i}\),
которое мы называем \(\beta\)-нормальной формой (или \(\beta\)-термом) и для которого
\(\beta(t'_{i})=\ulcorner \beta(t_{i})\urcorner\).
Иными словами, \(t'_{i}\) служит символическим представлением нормальной формы \(\beta(t_{i})\).
После этого можно имитировать вывод \(D\) в (символическом) выводе \(D'\),
начинающемся с \(\beta\)-нормальных форм \(t'_{1},\dots,t'_{n}\).
Промежуточные термы, возникающие в \(D'\), также являются \(\beta\)-нормальными формами
соответствующих термов из \(D\).
Система линейных уравнений \(\mathcal E\) постепенно эволюционирует
по мере замены \(t_{i}\) на их \(\beta\)-нормальные формы и симуляции вывода \(D\).

Более формально, преобразуя $t_i$ в $\beta$-нормальную форму,
мы одновременно «прикрепляем» к этой нормальной форме систему уравнений.
То есть вводим то, что будем называть \emph{$\beta$-кортежем}
\((t_i',\mathcal E_i)\),
где $t_i'$ — $\beta$-нормальная форма терма $t_i$,
а $\mathcal E_i$ — такая система линейных уравнений, что
\(\beta\models\mathcal E_i\)
и для любой оценки $\beta'$,
удовлетворяющей $\mathcal E_i$, выполняется
\(
  \ulcorner\beta'(t_i)\urcorner = \ulcorner\beta'(t_i')\urcorner.
\)

Иными словами,
$t_i'$ служит символическим представлением нормальной формы не только
для $\beta(t_i)$, но и для $\beta'(t_i)$ при всех $\beta'\models\mathcal E_i$
(в последнем случае $t_i'$ возможно требуется дополнительно нормализовать,
чтобы совпасть с $\ulcorner\beta'(t_i)\urcorner$).
Итоговая система уравнений $\mathcal E$
получается как объединение систем
из $\beta$-кортежей для $t_1,\dots,t_n,t$
и уравнений, возникающих в процессе симуляции вывода $D$.

Очевидно, чтобы доказать лемму, необходимо ограничить размер
$\beta$-кортежей, то есть $\beta$-нормальных форм
($\beta$-термов) вместе с присоединившимися к ним системами
уравнений, а также размер уравнений, возникающих при симуляции
вывода~$D$ (подробности см.\ в последующих подразделах).

Опираясь на сформулированную выше лемму, нетрудно доказать
Предложение 5.22, утверждающее, что для каждой минимальной атаки
$(\pi,\sigma)$ существует атака $(\pi,\sigma')$ той же структуры
(то есть $\sigma$ и $\sigma'$ совпадают с точностью до показателей
степеней произведений), причём полный размер $\sigma'$
и её показатели можно ограничить полиномом от размера протокола.

Суть доказательства Предложения 5.22 такова.
К подстановке $\sigma$ приписывается символическая версия
$\sigma^{Z}$, в которой все показатели заменены новыми целыми
переменными. Далее вышеупомянутая лемма применяется к случаю
$t=R_{i}\sigma^{Z}$ и $t_{j}=S_{j}\sigma^{Z}$ для всех
$j\in\{0,\dots,i-1\}$.
Для каждого $i$ лемма даёт систему уравнений $\mathcal E_{i}$, и
любое решение $\beta'$ объединённой системы
$\bigcup_{i}\mathcal E_{i}$ порождает новую атаку
\((\pi,\beta'(\sigma^{Z}))\) на протокол.
Поскольку объединённая система уравнений «небольшая», а линейные
системы имеют «небольшие» решения $\beta'$
(см.\ \cite{Bockmayr2001}), получаем атаку
$(\pi,\sigma')$, где $\sigma'=\beta'(\sigma^{Z})$, с
полиномиально ограниченными показателями.

В следующем подразделе мы вводим $\beta$-кортежи. Затем
показываем, что такие кортежи существуют (раздел 5.4).
В 5.5–5.6 оцениваем размеры $\beta$-термов и соответствующих им
систем уравнений, а следовательно, и размер $\beta$-кортежей в целом.
Наконец, в § 5.7 доказываем упомянутые Лемму 5.21 и
Предложение 5.22 и, опираясь на них, выводим, что задача
\textsc{Insecure} NP-полна для нарушителя DH.

\subsection{$\beta$-эквивалентность, $\beta$-кортежи и $\approx_{\beta}$-системы уравнений}

\textbf{Определение 5.3.}
Пусть заданы отображение оценки $\beta$ и два открытых сообщения
(или открытых произведений) $t$ и $t'$.
Говорим, что $t$ и $t'$ \emph{$\beta$-равны}
(обозначаем $t =_{\beta} t'$), если $\beta(t)=\beta(t')$.\footnote{%
  Равенство понимается по модулю ассоциативности и коммутативности
  умножения в произведении.}
Называем $t$ и $t'$ \emph{$\beta$-эквивалентными}
(пишем $t \approx_{\beta} t'$), если
\(
  \ulcorner\beta(t)\urcorner = \ulcorner\beta(t')\urcorner .
\)

\medskip
\textbf{Определение 5.4.}
Пусть $t =_{\beta} t'$.
Система линейных уравнений $\mathcal E$
называется \emph{$=_{\beta}$-системой уравнений} для $t$ и~$t'$,
если выполняются оба условия

\begin{enumerate}\itemsep0pt
\item[(1)] $\beta \models \mathcal E$;
\item[(2)] $t \approx_{\beta'} t'$ для всякого $\beta' \models \mathcal E$.
\end{enumerate}

Мы теперь показываем, что «небольшие» $\;=_{\beta}$-системы
уравнений действительно существуют.
Напомним, что, например, запись
$\Vert t,t'\Vert_{\text{ext}}$
означает $\Vert\!\{t,t'\}\!\Vert_{\text{ext}}$.

\textsc{Лемма 5.5.}
Пусть заданы отображение оценки $\beta$ и открытые сообщения
(или открытые произведения) $t$ и $t'$, причём $t =_{\beta} t'$.
Тогда существует $=_{\beta}$-система уравнений
$\mathcal E^{=_{\beta}}_{t,t'}$
размера не более $2\lVert t,t'\rVert_{\text{ext}}^{3}$,
соответствующая $t$ и $t'$.

\textit{Доказательство.}
Определим
\[
  R \;\subseteq\;
  S_{\text{ext}}(t,t')\times S_{\text{ext}}(t,t')
\]
так, чтобы для каждой пары $(s,s')\in R$ выполнялось $s =_{\beta} s'$.
Точнее, $R$ — наименьшее бинарное отношение на
$S_{\text{ext}}(t,t')$, удовлетворяющее условиям

— $(t,t')\in R$;  

— Если $(s,s')\in R$ и, по построению, $s =_{\beta} s'$, причём
  $s=\langle t_{1},t_{2}\rangle$, то найдутся открытые сообщения
  $t_{1}',t_{2}'$ такие, что $s'=\langle t_{1}',t_{2}'\rangle$.
  Тогда $\langle t_{1},t_{1}'\rangle\in R$ и
  $\langle t_{2},t_{2}'\rangle\in R$. Для случаев шифрования вводятся аналогичные условия на~$R$.

— Если $(s,s')\in R$ и $s$ — произведение
  $t_{1}^{e_{1}}\!\cdots t_{n}^{e_{n}},\; n\ge1,$
  то $s'$ также является произведением
  $t_{1}'^{\,e'_{1}}\!\cdots t_{n}'^{\,e'_{n}},\; n\ge1.$
  Если для некоторых $i,j$ выполнено $t_{i}=_{\beta}t_{j}'$,
  то пара $\langle t_{i},t_{j}'\rangle$ принадлежит $R$.
  Заметим, что из равенства $s =_{\beta} s'$ следует:
  для каждого $t_{i}$ существует хотя бы один
  $=_{\beta}$-равный ему терм $t_{j}'$.

— Если $t=Exp(u,M)$ для некоторого открытого сообщения $u$
  и открытого произведения $M$, то $t'=Exp(u',M')$
  для некоторого открытого сообщения $u'$ и открытого произведения $M'$.
  Тогда $(u,u')\in R$ и $(M,M')\in R$.

Для каждой пары $(s,s')\in R$ определим систему уравнений
$\mathcal E_{(s,s')}$ следующим образом.
Если $s$ (а значит и $s'$) не является произведением,
то $\mathcal E_{(s,s')}$ — пустое множество.
Иначе $s$ имеет вид
$t_{1}^{e_{1}}\!\cdots t_{n}^{e_{n}},\;n\ge 1,$
а $s'$ — вид
$t_{1}'^{\,e_{1}}\!\cdots t_{n}'^{\,e_{n}}$.
Положим
\(
  \mathcal E_{(s,s')}=\bigl\{\,e_{i}=e_{j}' \;\bigm|\;
        (t_{i},t_{j}')\in R\bigr\}.
\)

Структурной индукцией нетрудно показать, что
\(
  \mathcal E_{R}= \bigcup_{(s,s')\in R}\mathcal E_{(s,s')}
\)
является $=_{\beta}$-системой уравнений для $t$ и $t'$.
Очевидно, её размер не превосходит
$2\lVert t,t'\rVert_{\text{ext}}^{\,3}$. \qed

В дальнейшем мы будем обозначать
$\mathcal E^{=_{\beta}}_{t,t'}$, построенную в доказательстве,
как \emph{$=_{\beta}$-систему уравнений, индуцированную парой $t$ и $t'$}.

\textbf{Замечание 5.6}
Система $\mathcal E^{=_{\beta}}_{t,t'}$, сконструированная в Лемме 5.5,
определена однозначно.

Нам также потребуется приписывать
$\approx_{\beta}$-эквивалентным термам
систему уравнений, которую мы называем
\emph{$\approx_{\beta}$-системой уравнений}.

\textbf{Определение 5.7}
Пусть заданы $\beta$ и открытые сообщения (или произведения) $t$ и $t'$
такие, что $t \approx_{\beta} t'$.
Говорим, что $\mathcal E$ является
\emph{$\approx_{\beta}$-системой уравнений} для $t$ и $t'$, если

\begin{enumerate}
  \item $\beta\models\mathcal E$, и
  \item $t \approx_{\beta'} t'$ для всякой $\beta'\models\mathcal E$.
\end{enumerate}

Чтобы построить такую систему уравнений для пары $t$ и $t'$,
введём понятие $\beta$-кортежа.

\textbf{Определение 5.8.}
Пусть задано отображение оценки $\beta$ и открытое сообщение
(или открытое произведение) $t$.
Пара $(t',\mathcal E)$, где $t'$ — открытое сообщение (или произведение),
а $\mathcal E$ — система линейных уравнений,
называется \emph{$\beta$-кортежем} для $t$, если выполняются условия

\begin{enumerate}\itemsep0pt
\item[(1)] $\beta(t')=\ulcorner\beta(t)\urcorner$;
\item[(2)] $\beta\models\mathcal E$;
\item[(3)] $t\approx_{\beta'} t'$ для всякой $\beta'\models\mathcal E$.
\end{enumerate}

терм $t'$ называют \emph{$\beta$-термом}
(или \emph{$\beta$-нормальной формой}) терма $t$,
а $\mathcal E$ — \emph{$\beta$-системой уравнений} для $t$.

\medskip
Следующая лемма показывает, как
$\approx_{\beta}$-систему уравнений можно получить из $\beta$-кортежей.

\textsc{Лемма 5.9.}
Пусть $t$ и $t'$ — открытые сообщения или произведения такие, что
$t\approx_{\beta} t'$.
Предположим, что для $t$ задан $\beta$-кортеж $(s,\mathcal E)$,
а для $t'$ — $\beta$-кортеж $(s',\mathcal E')$.
Пусть $\mathcal E_{s,s'}^{=_{\beta}}$ — какая-нибудь
$=_{\beta}$-система уравнений для $s$ и $s'$ 
(такая система всегда существует).
Тогда
\[
  \mathcal E^{\approx_{\beta}}_{t,t'}
      \;=\;
      \mathcal E\;\cup\;\mathcal E'\;\cup\;\mathcal E_{s,s'}^{=_{\beta}}
\]
является $\approx_{\beta}$-системой уравнений для $t$ и $t'$.

\begin{proof}
Сначала покажем, что для $s$ и $s'$ всегда существует
$=_{\beta}$-система уравнений.
Поскольку
\(
  \beta(s)=\ulcorner\beta(t)\urcorner
          =\ulcorner\beta(t')\urcorner
          =\beta(s'),
\)
имеем $s =_{\beta} s'$, а по лемме 5.5 существует
$=_{\beta}$-система, которую обозначим
$\mathcal E^{=_{\beta}}_{s,s'}$.

Теперь докажем, что
$\mathcal E^{\approx_{\beta}}_{t,t'}
      =\mathcal E\cup\mathcal E'\cup\mathcal E^{=_{\beta}}_{s,s'}$
является $\approx_{\beta}$-системой для $t$ и $t'$.
Очевидно, $\beta\models\mathcal E^{\approx_{\beta}}_{t,t'}$.
Пусть $\beta'\models\mathcal E^{\approx_{\beta}}_{t,t'}$;
нужно показать, что
$\ulcorner\beta'(t)\urcorner=\ulcorner\beta'(t')\urcorner$.

Из $\beta'\models\mathcal E,\,\mathcal E'$ получаем
$\beta'(s)=\beta'(t)$ и $\beta'(s')=\beta'(t')$,
а из $\beta'\models\mathcal E^{=_{\beta}}_{s,s'}$ следует
$\beta'(s)=\beta'(s')$.
Следовательно,
\(
  \ulcorner\beta'(t)\urcorner
  =\ulcorner\beta'(s)\urcorner
  =\ulcorner\beta'(s')\urcorner
  =\ulcorner\beta'(t')\urcorner,
\)
что и требовалось.
\end{proof}

\subsection{Существование $\beta$-кортежей}

Покажем, что для любого открытого сообщения или произведения
существует $\beta$-кортеж.

\textsc{Лемма 5.10.}\;
Пусть $t$ — открытое сообщение или открытое произведение,
а $\beta$ — отображение оценки.
Тогда для $t$ существует $\beta$-кортеж.

\textit{Доказательство.}
Построим пару \((t^{\beta},\mathcal E^{\beta}_{t})\) индукцией по~$t$.

\medskip
— Если $t \in \mathcal A$, то полагаем $t^{\beta}=t$,
\(\mathcal E^{\beta}_{t}=\varnothing\).
Очевидно, \((t^{\beta},\mathcal E^{\beta}_{t})\) —
$\beta$-кортеж для~$t$.

— Пусть $t=\langle t_{1},t_{2}\rangle$.
По индукционному предположению имеем
\((t_{1}^{\beta},\mathcal E_{1}^{\beta})\)
и \((t_{2}^{\beta},\mathcal E_{2}^{\beta})\)
— $\beta$-кортежи для $t_{1}$ и $t_{2}$.
Определяем
\(
  t^{\beta}=\langle t_{1}^{\beta},t_{2}^{\beta}\rangle,
  \qquad
  \mathcal E^{\beta}_{t}= \mathcal E_{1}^{\beta}\cup\mathcal E_{2}^{\beta}.
\)
Аналогично строятся $\beta$-кортежи для случая шифрования.
Индукцией легко проверить, что $(t^{\beta},\mathcal E^{\beta}_{t})$
действительно является $\beta$-кортежем для~$t$.

— Пусть $t = t_{1}^{e_{1}}\!\cdots t_{n}^{e_{n}}$. Для каждого $i$ возьмём $\beta$-кортеж $(t_{i}^{\beta},\mathcal E_{t_{i}}^{\beta})$. Разобьём $\{t_{1},\dots,t_{n}\}$ на классы эквивалентности $C_{1},\dots,C_{l}$ по $\approx_{\beta}$; положим $e_{C_{j}}=\sum_{t_{i}\in C_{j}}e_{i}$ и выберем представителя $s_{C_{j}}\in C_{j}$, причём без ограничения общности считаем, что для всех $s\in C_{1}$ выполняется $s\approx_{\beta}1$ (класс $C_{1}$ может быть пустым). Индукционное предположение даёт $s^{\beta}=1$ для каждого $s\in C_{1}$, так как $\beta(s^{\beta})=\ulcorner\beta(s)\urcorner=1$. Обозначим $J=\{\,j\in\{2,\dots,l\}\mid\beta(e_{C_{j}})=0\}$; если $C=\{s_{1},\dots,s_{k}\}$ — класс, элементы которого попарно $\approx_{\beta}$-эквивалентны, а $s_{j}^{\beta}$ — $\beta$-терм для $s_{j}$, то кладём $\mathcal E_{C}^{\beta}=\bigcup_{i\ne j}\mathcal E^{=_{\beta}}_{s_{i}^{\beta},s_{j}^{\beta}}$.
Если \(C=\{s_{1},\dots,s_{k}\}\) —
класс, элементы которого попарно
\(\approx_{\beta}\)-эквивалентны,
и \(s_{j}^{\beta}\) — $\beta$-терм для $s_{j}$,
положим
\[
  \mathcal E_{C}^{\beta}=
    \bigcup_{\,i\neq j}
      \mathcal E_{s_{i}^{\beta},s_{j}^{\beta}}^{=_{\beta}}.
\]

Заметим, что
$\beta(s_i^{\beta})=\ulcorner\beta(s_i)\urcorner
                    =\ulcorner\beta(s_j)\urcorner
                    =\beta(s_j^{\beta})$,
следовательно $s_i^{\beta}=_{\beta}s_j^{\beta}$,
и по лемме 5.5 существует $=_{\beta}$-система
$\mathcal E_{s_i^{\beta},s_j^{\beta}}^{=_{\beta}}$.
Положим
\[
  t^{\beta}=
  \begin{cases}
    1, & \text{если }J=\{2,\dots,l\},\\[4pt]
    \displaystyle\prod_{j\notin J\cup\{1\}}
       (s_{C_j}^{\beta})^{e_{C_j}}, & \text{иначе}.
  \end{cases}
\]
Далее положим
\[
  \mathcal E^{\beta}=
      \bigcup_{i=1}^{n}\mathcal E_{t_i}^{\beta}\;\cup
      \bigcup_{j=2}^{l}\mathcal E_{C_j}^{\beta}\;\cup
      \bigcup_{j\in J}\{\,e_{C_j}=0\}.
\]

Индукцией легко показать, что $(t^{\beta},\mathcal E^{\beta})$ —
$\beta$-кортеж для $t$: действительно,
из индукционного предположения следует $\beta\models\mathcal E^{\beta}$.
По определению функции нормализации нетрудно проверить, что,
если $J=\{2,\dots,l\}$, то $\ulcorner\beta(t)\urcorner=1$,
а значит $t^{\beta}= \ulcorner\beta(t)\urcorner$.
В противном случае
\[
  \mathcal E_{t}^{\beta}
  =\bigcup_{i=1}^{n}\mathcal E_{t_i}^{\beta}\;
   \cup\!
   \bigcup_{j=2}^{l}\mathcal E_{C_j}^{\beta}\;
   \cup\!
   \bigcup_{j\in J}\{\,e_{C_j}=0\}.
\]
Индукционное предположение даёт, что $(t^{\beta},\mathcal E_{t}^{\beta})$ является $\beta$-кортежем для $t$: действительно, индукция обеспечивает $\beta\models\mathcal E_{t}^{\beta}$; по определению функции нормализации нетрудно проверить, что при $J=\{2,\dots,l\}$ имеем $\ulcorner\beta(t)\urcorner=1$, а значит $t^{\beta}=\ulcorner\beta(t)\urcorner$, тогда как при $J\ne\{2,\dots,l\}$ выполняется $\ulcorner\beta(t)\urcorner=\prod_{j\notin J\cup\{1\}}\ulcorner\beta(s_{C_{j}})\urcorner^{\,\beta(e_{C_{j}})}$, и, поскольку по индукции $\beta(s_{C_{j}}^{\beta})=\ulcorner\beta(s_{C_{j}})\urcorner$, получаем $\beta(t^{\beta})=\ulcorner\beta(t)\urcorner$; следовательно, $\beta(t^{\beta})=\ulcorner\beta(t)\urcorner$. Пусть теперь $\beta'\models\mathcal E_{t}^{\beta}$; если $s,s'\in C_{j}$ и $s\ne s'$.По определению $\mathcal E_{t}^{\beta}$ имеем
\(
  \beta'\;\models\;
  \mathcal E_{t}^{\beta}\cup
  \mathcal E_{s}^{\beta}\cup
  \mathcal E_{s_i^{\beta},s_j^{\beta}}^{=_{\beta}}
  \;(=\mathcal E_{s_i^{\beta},s_j^{\beta}}^{\approx_{\beta}}).
\)
Следовательно, по лемме 5.9
$\ulcorner\beta'(s)\urcorner=\ulcorner\beta'(s')\urcorner$.
Поскольку $s^{\beta}=1$, получаем $\beta'(s^{\beta})=1$
для каждого $s\in C_{1}$; кроме того, для $j\in J$ имеем
$\beta'(e_{C_j})=0$.
Отсюда непосредственно следует
$\ulcorner\beta'(t)\urcorner=\ulcorner\beta'(t^{\beta})\urcorner$.

\medskip\noindent
— Если $t=\operatorname{Exp}(u,M)$ и при этом
$\beta(u)\neq\operatorname{Exp}(\,\cdot,\cdot\,)$,
то по индукционному предположению существуют $\beta$-кортежи
$(u^{\beta},\mathcal E_{u}^{\beta})$ для $u$
и $(M^{\beta},\mathcal E_{M}^{\beta})$ для $M$.
Положим
\[
  t^{\beta}=
  \begin{cases}
    u^{\beta}, & \text{если } \beta(M)=1,\\[4pt]
    \operatorname{Exp}\!\bigl(u^{\beta},M^{\beta}\bigr), & \text{иначе},
  \end{cases}
  \qquad
\]

В обоих случаях мы присваиваем:
\[
\mathcal E_{t}^{\beta}= \mathcal E_{u}^{\beta}\cup\mathcal E_{M}^{\beta}.
\]

Окончим доказательство, показав, что $(t^{\beta},\mathcal E_{t}^{\beta})$
действительно является $\beta$-кортежем для $t$.
Очевидно, $\beta\models\mathcal E_{t}^{\beta}$, поэтому осталось
убедиться, что
\[
  \ulcorner\beta(t^{\beta})\urcorner=\ulcorner\beta(t)\urcorner
  \quad\text{и}\quad
  \ulcorner\beta'(t^{\beta})\urcorner=\ulcorner\beta'(t)\urcorner
  \;\; \text{для всякого } \beta'\models\mathcal E_{t}^{\beta}.
\]
Возникают два случая.

\medskip
(1) $\ulcorner\beta(M)\urcorner=1$.
Тогда $\ulcorner\beta(t)\urcorner=\ulcorner\beta(u)\urcorner$.
По индукции
\(
  \ulcorner\beta(t)\urcorner
  =\ulcorner\beta(u)\urcorner
  =\ulcorner\beta(u^{\beta})\urcorner
  =\ulcorner\beta(t^{\beta})\urcorner.
\)
Кроме того,
$\beta(M^{\beta})=\ulcorner\beta(M)\urcorner=1$, откуда $M^{\beta}=1$
и, по индукции,
$\ulcorner\beta'(M^{\beta})\urcorner=\ulcorner\beta'(M)\urcorner=1$.
Следовательно,
\[
  \ulcorner\beta'(t)\urcorner
  \;\overset{(*)}{=}\;
  \ulcorner\beta'(u)\urcorner
  \;\overset{(**)}{=}\;
  \ulcorner\beta'(u^{\beta})\urcorner
  =\ulcorner\beta'(t^{\beta})\urcorner,
\]
где $(*)$ следует из индукционного предположения и определения
$\mathcal E_{t}^{\beta}$, а $(**)$ — из определения $t^{\beta}$.

\medskip
(2)\;Или $\ulcorner\beta(M)\urcorner\neq1$. Тогда
\[
  \ulcorner\beta(t)\urcorner
    \;=\;
  \operatorname{Exp}\!\bigl(
      \ulcorner\beta(u)\urcorner,\,
      \ulcorner\beta(M)\urcorner
    \bigr)
  \;\overset{(*)}{=}\;
  \operatorname{Exp}\!\bigl(
      \ulcorner\beta(u^{\beta})\urcorner,\,
      \beta(M^{\beta})
    \bigr)
  \;\overset{(**)}{=}\;
  \ulcorner\beta(t^{\beta})\urcorner,
\]
где $(*)$ получено по индукции и определению $\mathcal E_{t}^{\beta}$,
а $(**)$ — по определению $t^{\beta}$.

Пусть теперь $\beta' \models \mathcal E_{t}^{\beta}$.
Тогда
\[
  \ulcorner\beta'(t)\urcorner
    \;=\;
  \operatorname{Exp}\!\bigl(
      \ulcorner\beta'(u)\urcorner,\,
      \ulcorner\beta'(M)\urcorner
    \bigr)
  \;\overset{(*)}{=}\;
  \operatorname{Exp}\!\bigl(
      \ulcorner\beta'(u^{\beta})\urcorner,\,
      \ulcorner\beta'(M^{\beta})\urcorner
    \bigr)
  \;\overset{(**)}{=}\;
  \ulcorner\beta'(t^{\beta})\urcorner,
\]
где $(*)$ снова следует из индукции и определения $\mathcal E_{t}^{\beta}$,
а $(**)$ — из определения $t^{\beta}$.

— Если $t=\operatorname{Exp}(u,M)$ и
  $\ulcorner\beta(u)\urcorner=\operatorname{Exp}(u',M')$,
  то по индукции существует $\beta$-кортеж
  $\bigl(u^{\beta},\mathcal E_{u}^{\beta}\bigr)$ для $u$.
  В частности, $\beta(u^{\beta})=\ulcorner\beta(u)\urcorner$,
  так что $u^{\beta}$ имеет вид
  $\operatorname{Exp}(u'',M'')$, где
  $\beta(u'')=u'$ и $\beta(M'')=M'$.
  Кроме того, по индукции существует $\beta$-кортеж
  $\bigl((M''\!\cdot M)^{\beta},\mathcal E_{(M''\cdot M)}^{\beta}\bigr)$
  для $(M''\!\cdot M)$.
  Положим
  \[
      t^{\beta}=
      \begin{cases}
        u'', & \text{если }\ulcorner\beta(M''\!\cdot M)\urcorner=1,
               \ \text{т.\,е. } \ulcorner M'\!\cdot\beta(M)\urcorner=1,\\[4pt]
        \operatorname{Exp}\!\bigl(u'',(M''\!\cdot M)^{\beta}\bigr), & \text{иначе},
      \end{cases}
  \]

В обоих случаях устанавливаем, что:
\[
\qquad
\mathcal E_{t}^{\beta}= \mathcal E_{u}^{\beta}\cup
                        \mathcal E_{(M''\cdot M)}^{\beta}.
\]

Индукцией теперь можно показать, что $(t^{\beta},\mathcal E_{t}^{\beta})$
действительно является $\beta$-кортежем для $t$.
Очевидно, $\beta\models\mathcal E_{t}^{\beta}$.
Следовательно, остаётся доказать равенства
$\beta(t^{\beta})=\ulcorner\beta(t)\urcorner$ и
$\ulcorner\beta'(t)\urcorner=\ulcorner\beta'(t^{\beta})\urcorner$ для всякого
$\beta'\models\mathcal E_{t}^{\beta}$.
Сначала положим
\(
  (u^{\beta},\mathcal E_{u}^{\beta})
  =\operatorname{Exp}(u'',M'')
\)
, как было выше.
Мы знаем, что
\(
  \ulcorner\beta(t)\urcorner=\ulcorner\operatorname{Exp}\!\bigl(\beta(u),\beta(M)\bigr)\urcorner
          =\ulcorner\operatorname{Exp}\!\bigl(u',\ulcorner\,M'\!\cdot\!\beta(M)\urcorner\bigr)\urcorner.
\)
Если $\ulcorner M'\!\cdot\!\beta(M)\urcorner=1$, то
\(
  \ulcorner\beta(t)\urcorner=u'=\beta(u'')=\beta(t^{\beta}).
\)
Иначе

\[
\begin{aligned}
  \ulcorner\beta(t)\urcorner
      &= Exp\!\bigl(u',\ulcorner\,M'\!\cdot\!\beta(M)\urcorner\bigr)
      &= \operatorname{Exp}\!\bigl(\beta(u''),
                                   \ulcorner\beta(M'')\!\cdot\!\beta(M)\urcorner\bigr)\\
      &= \operatorname{Exp}\!\bigl(\beta(u''),\,
                                   \ulcorner\beta\bigl(M''\!\cdot\!M\bigr)\urcorner\bigr)
      &\overset{(*)}{=} \operatorname{Exp}\!\bigl(\beta(u''),\,
                                   \beta\bigl((M''\!\cdot\!M)^{\beta}\bigr)\bigr)
         \;{(**)}
      &\overset{(**)}{=} \beta(t^{\beta}),
\end{aligned}
\]

\noindentгде равенство $(*)$ получено по индукции,
а $(**)$ — по определению $t^{\beta}$.
Пусть теперь $\beta'(t) \models \mathcal E_{t}^{\beta}$.
Тогда
\(
  \ulcorner\beta'(t)\urcorner=\ulcorner Exp\bigl(\beta'(u),\beta'(M)\bigr)\urcorner.
\)
Поскольку $\beta'\models \mathcal E_{u}^{\beta}$,
из индукционного предположения следует
\(
  \ulcorner\beta'(u)\urcorner=\ulcorner\beta'(u^{\beta})\urcorner
           =\ulcorner\beta'\!\bigl(\operatorname{Exp}(u'',M'')\bigr)\urcorner.
\)
Отсюда
\(
  \ulcorner\beta'(t)\urcorner=\ulcorner\operatorname{Exp}\!\bigl(
               \beta'(u^{\beta}),\,
               \beta'(M)
            \bigr)\urcorner
           =\ulcorner\operatorname{Exp}\!\bigl(
               \beta'(u''),\,
               \beta'(M'')\!\cdot\!\beta'(M)
            \bigr)\urcorner
           =\ulcorner\operatorname{Exp}\!\bigl(
               \beta'(u''),\,
               \beta'(M''\!\cdot\!M)
            \bigr)\urcorner.
\)
Так как $\beta'\models \mathcal E_{(M''\!\cdot M)}^{\beta}$,
по индукции имеем
\(
  \ulcorner\beta'\!\bigl((M''\!\cdot M)^{\beta}\bigr)\urcorner
  =\ulcorner\beta'(M''\!\cdot M)\urcorner.
\)
Следовательно,
\(
  \ulcorner\beta'(t)\urcorner
  =\ulcorner\operatorname{Exp}\!\bigl(
       \beta'(u''),\,
       \beta'\!\bigl((M''\!\cdot M)^{\beta}\bigr)
    \bigr)\urcorner.
\)
Если $t^{\beta}=\operatorname{Exp}\!\bigl(u'',(M''\!\cdot M)^{\beta}\bigr)$,
то $\ulcorner\beta'(t)\urcorner=\ulcorner\beta'(t^{\beta})\urcorner$.
В противном случае $t^{\beta}=u''$ и
$\ulcorner\beta(m''\!\cdot M)\urcorner=1$, а значит $(M''\!\cdot M)^{\beta}=1$,
и снова $\beta'(t)=\beta'(t^{\beta})$.
Таким образом, $\ulcorner\beta'(t)\urcorner=\ulcorner\beta'(t^{\beta})\urcorner$.
\qed

\subsection{Ограничение размера $\beta$-термов}

Начиная с этого места, через $t^{\beta}$ мы будем обозначать
$\beta$-терм сообщения (или произведения) $t$, построенный в доказательстве
леммы 5.10.
Наша цель — показать, что всегда существует $\beta$-кортеж для $t$,
размер которого ограничен полиномом от $\lVert t\rVert$.
Доказательство разбивается на два шага:
в настоящем подразделе мы получаем оценку размера $t^{\beta}$,
а в следующем — оцениваем размер системы уравнений,
ассоциированной с $t^{\beta}$.

Для начала установим, что $\beta$-терм определяется однозначно.

\textsc{Лемма 5.11.}
Для любого открытого сообщения или произведения $t$,
такого что $\beta(t)=\ulcorner\beta(t)\urcorner$, выполнено $t^{\beta}=t$.

\textit{Доказательство.} См. Приложение В.\hfill$\square$

\noindentДля множества $E$ открытых сообщений или произведений положим
\[
  E^{\beta}=\{\,t^{\beta}\mid t\in E\}.
\]
В следующей лемме мы получим верхнюю границу для
$\lvert t^{\beta}\rvert_{\text{ext}}$,
а в Лемме 5.14 — для $\Vert t^{\beta}\Vert_{\text{exp}}$.
Объединив эти результаты, Лемма 5.15 даёт оценку
для $\Vert t^{\beta}\Vert_{\text{ext}}$.

\textsc{Лемма 5.12.}
Пусть $t,t_{1},\dots,t_{n}$ — открытые сообщения или произведения,
а $\beta$ — отображение оценки.
Тогда выполняются следующие утверждения:

\begin{enumerate}\itemsep0pt
\item $\mathcal S(t^{\beta}) \subseteq \mathcal S(t)^{\beta}$;
\item $\lvert\ulcorner\beta(t)\urcorner\rvert \le \lvert t\rvert$
      и
      $\lvert\ulcorner\beta(t_{1})\urcorner\rvert,\dots,
       \lvert\ulcorner\beta(t_{n})\urcorner\rvert
       \le \lvert t_{1},\dots,t_{n}\rvert$;
\item $\lvert t^{\beta}\rvert_{\text{ext}} \le 2\cdot\lvert t\rvert$.
\end{enumerate}

\begin{proof}
Докажем пункты по отдельности:

(1) Доказательство ведём по структурной индукции по~$t$.

— Если $t\in\mathcal A$, то $\mathcal S(t^{\beta})=\{t\}=\mathcal S(t)^{\beta}$. Пусть $t=\langle t_{1},t_{2}\rangle$. По индукционному предположению
  \(
      S(t^{\beta})
      =\{t^{\beta}\}\cup S(t_{1}^{\beta})\cup S(t_{2}^{\beta})
      \subseteq \{t^{\beta}\}\cup S(t_{1})^{\beta}\cup S(t_{2})^{\beta}
      = S(t)^{\beta},
  \)
  и тот же довод проводится для случая шифрования.

— Пусть $t=t_{1}^{e_{1}}\!\cdots t_{n}^{e_{n}}$.
  Рассмотрим два подслучая. Если $t^{\beta}=1$, то очевидно $S(t^{\beta})\subseteq S(t)^{\beta}$.
  Иначе по доказательству леммы 5.10 имеем
  \(
      t^{\beta}=\prod_{j\notin J\cup\{1\}}
                (s_{C_{j}}^{\beta})^{e_{C_{j}}},
  \)
  причём для каждого $C_{j}$ найдётся $t_{i}$,
  такое что $t_{i}^{\beta}=s_{C_{j}}^{\beta}$.
  Индукция даёт
  \[
    S(t^{\beta})
      \subseteq
      \{t^{\beta}\}\cup
      \bigcup_{j\notin J\cup\{1\}}S\!\bigl(s_{C_{j}}^{\beta}\bigr)
      \subseteq
      \{t^{\beta}\}\cup
      \bigcup_{i=1}^{n}S(t_{i})^{\beta}
      = S(t)^{\beta}.
  \]

— Если $t=\operatorname{Exp}(u,M)$ и $t^{\beta}=u^{\beta}$,
  то индукционное предположение немедленно даёт
  $S(t^{\beta})\subseteq S(t)^{\beta}$.

— Если $t=\operatorname{Exp}(u,M)$, $t^{\beta}=\operatorname{Exp}(u^{\beta},M^{\beta})$
и $M=t_{1}^{e_{1}}\!\cdots t_{n}^{e_{n}}$, то $M^{\beta}\ne1$ и, по индукции,
\[
  S(t^{\beta})
    \subseteq \{t^{\beta}\}\cup S(u^{\beta})\cup\bigcup_{i}S(t_{i}^{\beta})
    \subseteq \{t^{\beta}\}\cup S(u)^{\beta}\cup\bigcup_{i}S(t_{i})^{\beta}
    = S(t)^{\beta}.
\]

— Если $t=\operatorname{Exp}(u,M)$, $u^{\beta}=\operatorname{Exp}(u',M')$ и
  $t^{\beta}=u^{\beta}$, то
  \[
      S(t^{\beta})\subseteq S(u^{\beta})\subseteq S(u)^{\beta} \subseteq S(t)^{\beta}.
  \]

— Наконец, предположим, что
   \(t=\operatorname{Exp}(u,M)\),
   \(u^{\beta}=\operatorname{Exp}(u'',M'')\),
   а
   \(t^{\beta}=\operatorname{Exp}\!\bigl(u'',(M''\!\cdot M)^{\beta}\bigr)\),
   причём \((M''\!\cdot M)^{\beta}\neq1\).
   Тогда \((M''\!\cdot M)^{\beta}\) имеет вид
   \(\prod_{j\notin J\cup\{1\}}\bigl(s_{C_{j}}^{\beta}\bigr)^{e'_{j}}\)
   для некоторых показателей \(e'_{j}\); при этом
   \(M''=t''_{1}{}^{e_{1}}\!\cdots t''_{n}{}^{e_{n}}\),
   \(M=t_{1}{}^{e_{1}}\!\cdots t_{n}{}^{e_{n}}\),
   и каждый \(s_{C_{j}}^{\beta}\) совпадает с некоторым
   \(t''_{i}\) или \(t_{i}^{\beta}\)
   (заметим, что \(t''_{i}=t_{i}^{\beta}\) по лемме 5.11).
   По индукционному предположению
   \(t''_{i}\in S(u^{\beta})\subseteq S(u)^{\beta}\), откуда
\[
\begin{aligned}
S(t^{\beta})
  &= \{t^{\beta}\}\,\cup\,S(u'')\,
     \cup\,\bigcup_{j\notin J\cup\{1\}} S\!\bigl(s_{C_j}^{\beta}\bigr)\\
  &\subseteq \{t^{\beta}\}\,\cup\,S(u'')\,
     \cup\,\bigcup_{i=1}^{n} S\!\bigl(t_{i}^{\beta}\bigr)\,
     \cup\,\bigcup_{i=1}^{n''} S\!\bigl(t''_{i}\bigr)\\
  &\subseteq \{t^{\beta}\}\,\cup\,\bigcup_{i=1}^{n} S\!\bigl(t_{i}\bigr)^{\beta}\,
     \cup\,S(u^{\beta})\\
  &\subseteq \{t^{\beta}\}\,\cup\,\bigcup_{i=1}^{n} S\!\bigl(t_{i}\bigr)^{\beta}\,
     \cup\,S(u)^{\beta}
  \;=\; S(t)^{\beta}.
\end{aligned}
\]

(2)\; Сначала заметим, что $\ulcorner\beta(t)\urcorner=\beta(t^{\beta})$.
Легко видеть, что для любого открытого сообщения или произведения $s$
выполняется $|\beta(s)|\le |s|$.
Следовательно,
\[
  |\ulcorner\beta(t)\urcorner|
     =|\beta(t^{\beta})|
     \le |t^{\beta}|
     \;\overset{(*)}{\le}\;
     \operatorname{Card}\bigl(S(t)^{\beta}\bigr)
     \le \operatorname{Card}\bigl(S(t)\bigr)
     =|t|,
\]
где в~$(*)$ используется пункт~1. 
Тот же довод даёт
$|\ulcorner\beta(t_{1})\urcorner|,\dots,
 |\ulcorner\beta(t_{n})\urcorner|
 \le |t_{1},\dots,t_{n}|$.

\smallskip
(3)\; Это непосредственное следствие пункта~1 и леммы 5.2:
\[
  |t^{\beta}|_{\text{ext}}
     \le 2\cdot|t^{\beta}|
     \le 2\cdot\operatorname{Card}\bigl(S(t^{\beta})\bigr)
     \le 2\cdot|t|.
\quad\square
\]
\end{proof}

Теперь нам нужно получить верхнюю оценку для
$\lVert t^{\beta}\rVert_{\text{exp}}$.
Для этого понадобится следующая лемма.

\textsc{Лемма 5.13.}
Пусть $E$ — конечное множество открытых сообщений или произведений,
такое что $S_{\text{ext}}(E)=E$, а $t$ — максимальный
(относительно упорядочения «строгий подтерм»)
элемент множества $E$.
Тогда
\[
  \bigl\lvert\,
        \bigcup_{s\in E} S_{\text{ext}}\!\bigl(s^{\beta}\bigr)
        \Bigr{\rvert}_{\text{exp}}
  \;\le\;
  \bigl\lvert\,
        \bigcup_{s\in E\setminus\{t\}}
            S_{\text{ext}}\!\bigl(s^{\beta}\bigr)
            \Bigr{\rvert}_{\text{exp}}
  +\lVert t\rVert_{\text{ext}}^{\,2}.
\]

\textit{Доказательство.} См. Приложение В.\hfill$\square$

Используя предыдущую лемму, получаем:

\textsc{Лемма 5.14.}\;
Для всякого открытого сообщения или произведения $t$ выполняется
\(
   \bigl\lVert t^{\beta}\bigr\rVert_{\text{exp}}
   \;\le\;
   \lVert t\rVert_{\text{ext}}^{\,3}.
\)

\textit{Доказательство.}
Положим $E = S_{\text{ext}}(t)$.
По Лемме 5.13 имеем
\[
   \bigl\lVert t^{\beta}\bigr\rVert_{\text{exp}}
   \;=\;
   \lvert S_{\text{ext}}(t^{\beta})\rvert_{\text{exp}}
   \;\le\;
   \Bigl\lvert\;
       \bigcup_{s\in E} S_{\text{ext}}\!\bigl(s^{\beta}\bigr)
   \Bigr\rvert_{\text{exp}} .
\]
Последовательно удаляя из $E$ (максимальный) элемент
и применяя Лемму 5.13 на каждом шаге,
получаем
\(
   \Bigl\lvert\;
       \bigcup_{s\in E} S_{\text{ext}}\!\bigl(s^{\beta}\bigr)
   \Bigr\rvert_{\text{exp}}
   \;\le\;
   \lvert t\rvert_{\text{ext}}\;\cdot\;\lVert t\rVert_{\text{ext}}^2 ,
\)
а так как $\operatorname{Card}(E)=\lvert t\rvert_{\text{ext}}$,
получаем требуемое неравенство
\(
   \lVert t^{\beta}\rVert_{\text{exp}}
   \le
   \lVert t\rVert_{\text{ext}}^{\,3}.
\quad\square
\)

Мы уже ограничили как количество расширенных подтермов терма $t^{\beta}$,
так и величину его целых коэффициентов (см.~Леммы 5.12 и 5.14).
Объединив эти оценки, мы, наконец, получаем полиномиальную верхнюю
границу на размер $\beta$-термов.

\textbf{Лемма 5.15.}\;
Для любого открытого сообщения или произведения $t$ выполняется
\[
      \bigl\lVert t^{\beta}\bigr\rVert_{\text{ext}}
      \;\le\;
      3\cdot\lVert t\rVert_{\text{ext}}^{\,3}.
\]

\subsection{Ограничение размера $\beta$-систем уравнений}

В предыдущем подразделе мы получили полиномиальную оценку
для размера $\beta$-термов.
Теперь ограничим размер $\beta$-систем уравнений,
связанных с этими термами, а тем самым и общий размер
$\beta$-кортежей.
Для этого сначала зададим конкретную $\beta$-систему уравнений
$\mathcal E_{t}^{\beta}$ для терма $t$,
размер которой полиномиально ограничен величиной $\lVert t\rVert_{\text{ext}}$.

Пусть далее $\beta$ — отображение оценки, а $t$ — открытое сообщение
или произведение. Опишем систему уравнений
$\mathcal E_{t}^{\prime\,\beta}$, которая добавляется
при переходе от системы, построенной для подтермов $t$,
к системе уравнений самого $t$:

\medskip
— Если $t$ — атом, пара или шифрование, то $\mathcal E_{t}^{'\beta}=\varnothing$.

— Если $t=t_{1}^{e_{1}}\!\cdots t_{n}^{e_{n}}$, то, используя обозначения
из доказательства Леммы 5.10, положим
\[
      \mathcal E_{t}^{'\beta}
      =\bigcup_{j\ge2}\mathcal E_{C_{j}}^{\beta}
       \;\cup\;
       \bigcup_{j\in J}\{\,e_{C_{j}}=0\}.
\]

— Если $t=\operatorname{Exp}(u,M)$ и
  $\ulcorner\beta(u)\urcorner\neq\operatorname{Exp}(\,\cdot,\cdot\,)$,
  то $\mathcal E_{t}^{'\beta}=\varnothing$.
  В противном случае $u^{\beta}=\operatorname{Exp}(u'',M')$ и
  полагаем $\mathcal E_{t}^{'\beta}=\mathcal E_{(M''\!\cdot M)}^{'\beta}$.

\textit{Замечание 5.16.}
Система уравнений $\mathcal E_{t}^{\prime\,\beta}$ однозначно определяется
(по модулю AC\textsubscript{+}).

Система $\mathcal E_{t}^{\prime\,\beta}$ фиксирует ограничения
на показатели степеней \emph{на одном уровне} терма $t$.
Полная $\beta$-система уравнений для $t$ задаётся как объединение
систем, построенных для всех его подтермов:
\[
  \mathcal E_{t}^{\beta}
  \;=\;
  \bigcup_{s\in S_{\text{ext}}(t)} \mathcal E_{s}^{\prime\,\beta}.
\]

Далее мы докажем, что пара $(t^{\beta},\mathcal E_{t}^{\beta})$
является $\beta$-кортежем для $t$, а также то, что размер
$\mathcal E_{t}^{\beta}$ полиномиально ограничен размером $t$.
Однако прежде потребуется следующая лемма о $\beta$-кортежах
для произведений:

\textsc{Лемма 5.17.}
Пусть 
\(M = t_{1}^{e_{1}}\!\cdots t_{n}^{e_{n}}\)
и
\(M' = t_{1}'^{\,e_{1}'}\!\cdots t_{n'}'^{\,e_{n'}'}\) —
два открытых произведения такие, что 
\(\beta(t_{i}) = \ulcorner\beta(t_{i}')\urcorner\)
для всех соответствующих множителей.
Пусть для каждого \(i\) задан $\beta$-кортеж 
\((t_{i}^{\beta},\mathcal E_{i})\) терма \(t_{i}\).
Тогда пара
\[
\Bigl((M'\!\cdot M)^{\beta},\;
      \mathcal E_{(M'\!\cdot M)}^{\prime\,\beta}
      \,\cup\!
      \bigcup_{i}\mathcal E_{i}\Bigr)
\]
является $\beta$-кортежем для произведения \(M'\!\cdot M\).

\begin{proof}Смотреть приложение В.\end{proof}

\textbf{Лемма 5.18.}\;
Для любого открытого сообщения или произведения $t$ и всякого отображения оценки $\beta$ выполняются следующие утверждения:
\begin{enumerate}\itemsep0pt
\item[(1)] Пара $\bigl(t^{\beta},\mathcal E_{t}^{\beta}\bigr)$ является $\beta$-кортежем для $t$.
\item[(2)] Размер $\mathcal E_{t}^{\beta}$ ограничен полиномом от $\lVert t\rVert_{\text{ext}}$.
\item[(3)] Общий размер $\bigl(t^{\beta},\mathcal E_{t}^{\beta}\bigr)$, равный сумме размеров $t^{\beta}$ и $\mathcal E_{t}^{\beta}$, также ограничен полиномом от $\lVert t\rVert_{\text{ext}}$.
\end{enumerate}

\textit{Доказательство.}
Докажем пункты по отдельности.

\noindent(1)\; Проводим структурную индукцию по $t$ в соответствии с построением из леммы 5.10. \qedhere

— Если $t\in\mathcal A$, случай тривиален. Предположим, что  
  $t=\langle t_{1},t_{2}\rangle$. Тогда  
  \(
      \mathcal E_{t}^{\beta}
      =\mathcal E_{t}^{\prime\,\beta}
       \;\cup\!
       \bigcup_{s\in S_{\text{ext}}(t_{1})}\mathcal E_{s}^{\prime\,\beta}
       \;\cup\!
       \bigcup_{s\in S_{\text{ext}}(t_{2})}\mathcal E_{s}^{\prime\,\beta}.
  \)
  По определению имеем  
  \(
      \mathcal E_{t}^{\beta}
      =\mathcal E_{t_{1}}^{\beta}\cup\mathcal E_{t_{2}}^{\beta}
  \)
  (здесь используется Замечание 5.16).  
  Точно так же, как в доказательстве леммы 5.10, отсюда следует, что  
  $\bigl(t^{\beta},\mathcal E_{t}^{\beta}\bigr)$ является $\beta$-кортежем для $t$.  
  Рассуждение для случая шифрования проводится аналогично.

— Если \(t = t_{1}^{e_{1}}\!\cdots t_{n}^{e_{n}}\), то, пользуясь
  обозначениями, введёнными в доказательстве леммы 5.10,
  а также определением \(\mathcal E_{t}^{\prime\,\beta}\) и
  Замечанием 5.16, получаем
\(
   \mathcal E_{t}^{\beta}
     = \bigcup_{i=1}^{n}\mathcal E_{t_{i}}^{\beta}
       \;\cup\!
       \bigcup_{j=2}^{\ell}\mathcal E_{C_{j}}^{\beta}
       \;\cup\!
       \bigcup_{j\in J}\{\,e_{C_{j}} = 0\}.
\)
Точно так же, как в доказательстве леммы 5.10,
из этого следует, что пара \(\bigl(t^{\beta},\mathcal E_{t}^{\beta}\bigr)\)
является \(\beta\)-кортежем для \(t\).

— Если $t=\operatorname{Exp}(u,M)$ и
  $\beta(u)\neq\operatorname{Exp}(\,\cdot,\cdot\,)$, то по определению
  $\mathcal E_{t}^{\beta}$ и в силу Замечания 5.16 имеем
  $\mathcal E_{t}^{\beta}= \mathcal E_{u}^{\beta}\cup\mathcal E_{M}^{\beta}$;
  как и в лемме 5.10, отсюда следует, что
  $(t^{\beta},\mathcal E_{t}^{\beta})$ является $\beta$-кортежем для $t$.

— Наконец, пусть $t=\operatorname{Exp}(u,M)$, причём
$\ulcorner\beta(u)\urcorner\neq\operatorname{Exp}(\,\cdot,\cdot\,)$ и
$u^{\beta}= \operatorname{Exp}(u'',M'')$.
Предположим, что $M=t_{1}^{e_{1}}\!\cdots t_{n}^{e_{n}}$.
По лемме 5.17 пара
\(
  \Bigl((M''\!\cdot M)^{\beta},\;
        \mathcal E_{(M''\cdot M)}^{\prime\,\beta}\cup
        \bigcup_{i}\mathcal E_{t_{i}}^{\beta}\Bigr)
\)
является $\beta$-кортежем для произведения $M''\!\cdot M$.
По определению $\mathcal E_{t}^{\beta}$ и в силу Замечания 5.16 имеем
\(
  \mathcal E_{u}^{\beta}\;\cup\;
  \bigcup_{i}\mathcal E_{t_{i}}^{\beta}\;\cup\;
  \mathcal E_{(M''\cdot M)}^{\prime\,\beta}
  \;\subseteq\; \mathcal E_{t}^{\beta}.
\)
Следовательно, точно так же, как в доказательстве леммы 5.10,
получаем, что $\bigl(t^{\beta},\mathcal E_{t}^{\beta}\bigr)$
является $\beta$-кортежем для $t$.

\medskip
\noindent(2)\; Если $t$ — атом, пара либо шифрование, то дополнительных
оценок не требуется. Если же $t$ является \emph{произведением}, то из
лемм 5.5 и 5.15 немедленно следует, что размер
$\mathcal E_{t}^{\prime\,\beta}$ ограничен полиномом от
$\lVert t\rVert_{\text{ext}}$.  
Для случая $t=\operatorname{Exp}(\,\cdot,\cdot\,)$ ту же полиномиальную
оценку получают, применяя лемму 5.15 совместно с уже рассмотренным
случаем произведения.

\medskip
\noindent(3)\; Пусть $p$ — полином, ограничивающий размер
$\mathcal E_{t}^{\prime\,\beta}$. Тогда величина
\(p\!\bigl(\lVert t\rVert_{\text{ext}}\bigr)\cdot\lVert t\rVert_{\text{ext}}\)
ограничивает размер $\mathcal E_{t}^{\beta}$.
Кроме того, по лемме 5.15 размер $t^{\beta}$ также
полиномиально ограничен величиной $\lVert t\rVert_{\text{ext}}$.
\qed

Наконец, леммы 5.5, 5.9 и 5.18 обеспечивают существование
конкретных $\approx_{\beta}$-систем уравнений, чьи размеры
полиномиально ограничены.

\textsc{Предложение 5.19.}
Пусть $t$ и $t'$ — открытые сообщения или произведения, а
$\beta$ — отображение оценки, такое что $t \approx_{\beta} t'$.
Тогда существует $\approx_{\beta}$-система уравнений для
$t$ и $t'$, чей размер полиномиально ограничен величиной
$\lVert t,t'\rVert_{\text{ext}}$.

Такую систему уравнений будем обозначать
\(
   \mathcal E^{\,\approx_{\beta}}_{t,t'}.
\)

\subsection{Ограничение величины показателей степеней в атаках}

Теперь мы готовы доказать ключевую лемму этого раздела — Лемму 5.21.
Как отмечалось выше, из этой леммы следует, что правила Диффи–Хеллмана
позволяют осуществлять атаки, в которых показатели степеней произведений
ограничены полиномом (Предложение 5.22).
Немедленным следствием является то, что задача
\textsc{Insecure} NP-полна для нарушителя DH (Теорема 5.23).

Доказательство Леммы 5.21 проводится в два этапа.
Сначала рассматривается ограниченный вариант, когда
применяется лишь одно правило нарушителя (Лемма 5.20);
затем этот результат распространяется на полное построение вывода.

В дальнейших доказательствах нам понадобятся расширения отображений
оценки. Будем говорить, что отображение
\(\beta' : Z' \to Z\) является \emph{расширением}
отображения оценки \(\beta : Z \to Z\), если \(Z\subseteq Z'\) и \(\beta'(z)=\beta(z)\) для всех \(z\in Z\).
Так как на множестве \(Z\) отображения \(\beta\) и \(\beta'\) совпадают,
то, не перегружая обозначения, расширение \(\beta\) будем
часто обозначать тем же символом \(\beta\).

\textsc{Лемма 5.20.}
Пусть $t_{1},\dots,t_{n}$ — открытые сообщения,
$s$ — нормализованное сообщение,
$\beta$ — отображение оценки,
а $L\in\mathcal L$ — правило нарушителя такое, что
$\beta(t_{i})=\ulcorner\beta(t_{i})\urcorner$ для всех~$i$
и
$\ulcorner\beta(t_{1})\urcorner,\dots,
 \ulcorner\beta(t_{n})\urcorner \;\rightarrow\; s\in L$.
Тогда существует открытое сообщение $t$,
система линейных уравнений~$\mathcal E$
и расширение $\beta$ отображения $\beta$ такие, что
\begin{enumerate}\itemsep0pt
  \item[\textup{(1)}] $\beta(t)=s$;
  \item[\textup{(2)}] $\beta \models \mathcal E$;
  \item[\textup{(3)}] $\ulcorner\beta'(t_{1})\urcorner,\dots,
        \ulcorner\beta'(t_{n})\urcorner
        \;\rightarrow\;
        \ulcorner\beta'(t)\urcorner$
        для всякого $\beta'\models\mathcal E$;
  \item[\textup{(4)}] $\displaystyle
        \max\bigl\{|e|\;\bigm|\;e\in \mathcal L_{\text{exp}}(t)\bigr\}
        \;\le\;
        \max\bigl\{|e|\;\bigm|\;e\in \mathcal L_{\text{exp}}(t_{1},\dots,t_{n})\bigr\}
        + n$;
  \item[\textup{(5)}] размер $\mathcal E$ полиномиально ограничен
        величиной $\lVert t_{1},\dots,t_{n}\rVert_{\text{ext}}$.
\end{enumerate}

\textit{Доказательство.} См.\ Приложение В.\hfill$\square$

\textsc{Лемма 5.21.}
Пусть $t,t_{1},\dots,t_{n}$ — открытые сообщения такие, что
существует вывод, показывающий
\(
   \ulcorner\beta(t)\urcorner \in \mathrm{forge}
   \bigl(\ulcorner\beta(t_{1})\urcorner,\dots,
         \ulcorner\beta(t_{n})\urcorner\bigr)
\).
Тогда существует расширение отображения оценки $\beta$
и система линейных уравнений $\mathcal E$ такие, что

\begin{enumerate}\itemsep0pt
\item[\textup{(1)}] \(\beta\models\mathcal E\);
\item[\textup{(2)}] для всякого \(\beta'\models\mathcal E\)
      выполняется
      \(
        \ulcorner\beta'(t)\urcorner \in
        \mathrm{forge}\bigl(\ulcorner\beta'(t_{1})\urcorner,\dots,
                            \ulcorner\beta'(t_{n})\urcorner\bigr);
      \)
\item[\textup{(3)}] размер $\mathcal E$ полиномиально ограничен
      величиной \(\lVert t_{1},\dots,t_{n},t\rVert_{\text{ext}}\).
\end{enumerate}

\begin{proof}
Пусть \(E=\{\ulcorner\beta(t_{1})\urcorner,\dots,\ulcorner\beta(t_{n})\urcorner\}\), а \(D\) — корректный вывод,
свидетельствующий тому, что
\(\ulcorner\beta'(t)\urcorner\in\mathrm{forge}(E)\).
Известно, что длина \(l\) вывода \(D\) полиномиально ограничена величиной
\(\lvert\ulcorner\beta(t_{1})\urcorner,\dots,\ulcorner\beta(t_{n})\urcorner,\ulcorner\beta(t)\urcorner\rvert\).
По пункту 2 леммы 5.12
\(\lvert\ulcorner\beta(t_{1})\urcorner,\dots,\ulcorner\beta(t_{n})\urcorner,\ulcorner\beta(t)\urcorner\rvert\)
ограничено полиномом от
\(\lvert t_{1},\dots,t_{n},t\rvert\),
а значит и от
\(\lVert t_{1},\dots,t_{n},t\rVert_{\text{ext}}\).
Пусть \(i\)-й шаг вывода \(D\) имеет вид
\(
  E,s_{1},\dots,s_{i-1}
  \;\rightarrow_{\,L_{i}\,}\;
  E,s_{1},\dots,s_{i},
  \qquad 1\le i\le l,
\)
где каждое сообщение \(s_{i}\) нормализовано и
\(s_{l}=\ulcorner\beta(t)\urcorner\).
Поскольку вывод \(D\) корректен, для всех \(i\) выполняется
\(s_{i}\in S\!\bigl(\ulcorner\beta(t_{1})\urcorner,\dots,\ulcorner\beta(t_{n})\urcorner,\ulcorner\beta(t)\urcorner\bigr)\).

Пусть \((t^{\beta},\mathcal E_{t}^{\beta})\) — $\beta$-кортеж терма \(t\),
а для каждого \(i\) задан $\beta$-кортеж \(t_{i}^{\beta}\) терма \(t_{i}\).
Тогда \(\beta(t_{i}^{\beta})=\ulcorner\beta(t_{i})\urcorner\), и, следовательно,  
\(E=\{\beta(t_{1}^{\beta}),\dots,\beta(t_{n}^{\beta})\}\).
К первому шагу вывода \(D\) можно применить лемму 5.20, получив
открытое сообщение \(s_{1}'\), систему уравнений \(\mathcal E_{1}\)
и расширение отображения оценки \(\beta\) такие, что  
\(\beta(s_{1}') = s_{1}\), \(\beta\models\mathcal E_{1}\) и
\(
  \ulcorner\beta'(t_{1}^{\beta})\urcorner,\dots,
  \ulcorner\beta'(t_{n}^{\beta})\urcorner
  \;\xrightarrow{\,L_{1}\,}\;
  \ulcorner\beta'(t_{1}^{\beta})\urcorner,\dots,
  \ulcorner\beta'(t_{n}^{\beta})\urcorner,\,
  \ulcorner\beta'(s_{1}')\urcorner
  \qquad\text{для всякого } \beta'\models\mathcal E_{1}.
\)

Заметим, что для каждого $i$ выполняется
$\beta(t_{i}^{\beta})=\ulcorner\beta(t_{i}^{\beta})\urcorner$, 
а также $\beta(s_{1}')=\ulcorner\beta(s_{1}')\urcorner=s_{1}$.
Следовательно, лемму 5.20 можно применять индуктивно.
Для каждого шага $j$ ($1\le j\le l$) получаем открытое сообщение $s_{j}'$,
систему уравнений $\mathcal E_{j}$
и расширение отображения оценки $\beta$ такие, что

\(
  \beta(s_{j}') = s_{j}, 
  \beta \models \mathcal E_{j},
  \ulcorner\beta'(t_{1}^{\beta})\urcorner,\dots,
  \ulcorner\beta'(t_{n}^{\beta})\urcorner,
  \ulcorner\beta'(s_{1}')\urcorner,\dots,
  \ulcorner\beta'(s_{j-1}')\urcorner
  \;\rightarrow_{\,L_{j}\,}\;
  \ulcorner\beta'(t_{1}^{\beta})\urcorner,\dots,
  \ulcorner\beta'(t_{n}^{\beta})\urcorner,
  \ulcorner\beta'(s_{1}')\urcorner,\dots,
  \ulcorner\beta'(s_{j}')\urcorner
\)
для всякого $\beta' \models \mathcal E_{j}$.
Следовательно,
\(
   \beta \models \bigcup_{j=1}^{l}\mathcal E_{j}
   \text{и}
   \ulcorner\beta'(s_{l}')\urcorner
      \in
   \mathrm{forge}\bigl(
      \ulcorner\beta'(t_{1}^{\beta})\urcorner,\dots,
      \ulcorner\beta'(t_{n}^{\beta})\urcorner
   \bigr)
   \text{для всякого } \beta' \models \bigcup_{j=1}^{l}\mathcal E_{j}.
\)

Если \(\beta' \models \bigcup_{i=1}^{n}\mathcal E_{t_i}^{\beta}\),
то \(\ulcorner\beta'(t_i)\urcorner=\ulcorner\beta'(t_i^{\beta})\urcorner\)
для всех \(i\).
Поскольку \(\beta(t^{\beta})=\ulcorner\beta(t)\urcorner=s_{l}=\beta(s_{l}')\),
получаем \(t^{\beta}=_{\beta} s_{l}'\).
Следовательно, по лемме 5.5 существует
\(=_{\beta}\)-система уравнений
\(\mathcal E_{t^{\beta},\,s_{l}'}^{=_{\beta}}\) для \(t^{\beta}\) и \(s_{l}'\).

\smallskip
Теперь, если
\(\beta' \models \mathcal E_{t}^{\beta}\cup\mathcal E_{t^{\beta},\,s_{l}'}^{=_{\beta}}\),
имеем
\(
  \ulcorner\beta'(t)\urcorner
  =\ulcorner\beta'(t^{\beta})\urcorner
  =\ulcorner\beta'(s_{l}')\urcorner.
\). Положим

\[
  \mathcal E \;=\;
  \bigcup_{i=1}^{n} \mathcal E_{t_i}^{\beta}
  \;\cup\;
  \mathcal E_{t}^{\beta}
  \;\cup\;
  \bigcup_{j=1}^{l} \mathcal E_{j}
  \;\cup\;
  \mathcal E_{t^{\beta},\,s_{l}'}^{=_{\beta}} .
\]
Отсюда имеем $\beta\models\mathcal E$ и 
\(
  \ulcorner\beta'(t)\urcorner \in
  \mathrm{forge}\bigl(\ulcorner\beta'(t_{1})\urcorner,\dots,
                     \ulcorner\beta'(t_{n})\urcorner\bigr)
\)
для любого $\beta'\models\mathcal E$.

Остаётся показать, что размер $\mathcal E$ полиномиально ограничен
величиной $\lVert t_{1},\dots,t_{n},t\rVert_{\text{ext}}$.
По лемме 5.20 каждый
\(\mathcal E_{j}\)
полиномиально ограничен
в \(\lVert t_{1},\dots,t_{n},s_{1}',\dots,s_{j-1}'\rVert_{\text{ext}}\).
Заметим, что
\(
  \beta(s_{j}') = s_{j}\in
  S\!\bigl(\ulcorner\beta(t_{1})\urcorner,\dots,\ulcorner\beta(t_{n})\urcorner,\ulcorner\beta(t)\urcorner\bigr)
\).
Следовательно, по лемме 5.12
\(|s_{j}'|\) полиномиально ограничено
в \(|t_{1},\dots,t_{n},t|\),
а по лемме 5.2 то же верно и для
\(|s_{j}'|_{\text{ext}}\).

Из леммы 5.20 получаем
\[
   \max\{|e| \mid e\in \mathcal L_{\text{exp}}(s_{j}')\}
   \;\le\;
   \max\{|e| \mid e\in \mathcal L_{\text{exp}}(t_{1}^{\beta},\dots,t_{n}^{\beta})\}
   + n\,(j-1).
\]
Поскольку \(j\le l\), а длина $l$ вывода полиномиально ограничена
значением \(\lVert t_{1},\dots,t_{n},t\rVert_{\text{ext}}\),
и, кроме того,
\(
   \max\{|e| \mid e\in \mathcal L_{\text{exp}}(t_{1}^{\beta},\dots,t_{n}^{\beta})\}
   \le \lVert t_{i}\rVert_{\text{ext}}^{\,3}
\)
для некоторого~$i$ (лемма 5.14), существует полином $p$
такой, что \(\lVert s_{j}'\rVert_{\text{ext}}\le
p\bigl(\lVert t_{1},\dots,t_{n},t\rVert_{\text{ext}}\bigr)\).
Отсюда
\[
  \lVert t_{1},\dots,t_{n},t,s_{1}',\dots,s_{j}'\rVert_{\text{ext}}
  \;\le\;
  \bigl(p'(\lVert t_{1},\dots,t_{n},t\rVert_{\text{ext}})+1\bigr)\,
  p\bigl(\lVert t_{1},\dots,t_{n},t\rVert_{\text{ext}}\bigr),
\]
где $p'$ — полином, ограничивающий $l$.
По лемме 5.20 это означает, что каждый
\(\mathcal E_{j}\)
полиномиально ограничен
в \(\lVert t_{1},\dots,t_{n},t\rVert_{\text{ext}}\).
Леммы 5.18 и 5.5 теперь дают, что и
\(\mathcal E\)
в целом полиномиально ограничена той же величиной.

\end{proof}

Теперь мы можем показать, что правила Диффи–Хеллмана допускают атаки,
в которых показатели степеней произведений остаются полиномиально
ограниченными.

\textit{Предложение 5.22}
Правила DH допускают атаки с полиномиально ограниченными
показателями степеней произведений.

\begin{proof}
Пусть \((\pi,\sigma)\) — минимальная атака на протокол \(P\).
Обозначим через \(\sigma^{Z}\) подстановку, полученную из~\(\sigma\)
заменой \emph{всех} показателей степеней на новые переменные.
Положим, что отображение оценки \(\beta\) сопоставляет каждой из этих
переменных соответствующий показатель произведения, то есть
\(\sigma(x)=\beta\bigl(\sigma^{Z}(x)\bigr)\) для любого
\(x\in\mathcal V(P)\).
По следствию 3.16 длина \(|\sigma^{Z}|\) полиномиально
ограничена величиной \(|P|\).
Так как в \(\sigma^{Z}\) показатели степеней являются
переменными, получаем \(\lVert\sigma^{Z}\rVert_{\text{exp}}\le|\sigma^{Z}|^{2}\).
Следовательно,
\(
  \lVert\sigma^{Z}\rVert_{\text{ext}}
    \text{ полиномиально ограничена величиной }
    \lVert P\rVert_{\text{ext}}.\qedhere
\)

Пусть $k,R_{1},\dots,R_{k},S_{0},\dots,S_{k}$ определены, как и раньше.  
Без ограничения общности будем считать, что $S_{0}$ состоит из одного
сообщения, а не из множества сообщений
(иначе запишем $S_{0}=\{a_{1},\dots,a_{n}\}$ в виде
\(\langle a_{1},\,(a_{2}\,\dots\,(a_{n-1},a_{n})\dots )\rangle\)).
Положим \(R_{k+1}=\mathit{secret}\).
Известно, что
\[
   \beta\bigl(\ulcorner R_{i}\sigma^{Z}\urcorner\bigr)
   \;\in\;
   \mathrm{forge}\!\bigl(
       \ulcorner\beta(S_{0}\sigma^{Z})\urcorner,\dots,
       \ulcorner\beta(S_{i-1}\sigma^{Z})\urcorner
   \bigr),
   \qquad 1\le i\le k+1.
\]

По лемме 5.21 для каждого $i$ существует расширение отображения оценки
$\beta$ (причём все такие расширения независимы друг от друга)
и система линейных уравнений~\(\mathcal E_{i}\) такие, что
\begin{itemize}\itemsep0pt
\item \(\beta\models\mathcal E_{i}\);
\item для любого \(\beta'\models\mathcal E_{i}\) выполняется
      \[
         \ulcorner\beta'(R_{i}\sigma^{Z})\urcorner \;\in\;
         \mathrm{forge}\!\bigl(
            \ulcorner\beta'(S_{0}\sigma^{Z})\urcorner,\dots,
            \ulcorner\beta'(S_{i-1}\sigma^{Z})\urcorner
         \bigr);
      \]
\item размер \(\mathcal E_{i}\) полиномиально ограничен значением
      \(\lVert S_{0}\sigma^{Z},\dots,S_{k}\sigma^{Z},
              R_{1}\sigma^{Z},\dots,R_{k+1}\sigma^{Z}\rVert_{\text{ext}}\),
      которое, в свою очередь, полиномиально ограничено
      величиной \(\lVert P\rVert_{\text{ext}}\).
\end{itemize}

Следовательно, \(\beta \models \bigcup_{i=1}^{k}\mathcal E_{i}=:\mathcal E\);
отсюда система \(\mathcal E\) разрешима, и для всякого
\(\beta'\models\mathcal E\) пара
\(\bigl(\pi,\beta'(\sigma^{Z})\bigr)\) представляет собой атаку на~\(P\).
По результату \cite{Bockmayr2001} существует решение
\(\beta'\) системы \(\mathcal E\), для которого двоичная запись всех
целых коэффициентов полиномиально ограничена размером \(\mathcal E\);
следовательно, благодаря лемме 5.21 эта запись
полиномиально ограничена величиной \(\lVert P\rVert_{\text{ext}}\).
Определим \(\sigma' := \beta'(\sigma^{Z})\);
тогда \((\pi,\sigma')\) является атакой на~\(P\). Так же \(\sigma \approx \sigma'\), то есть подстановки \(\sigma\) и
\(\sigma'\) различаются лишь показателями степеней,
а величина \(\lVert\sigma'\rVert_{\text{exp}}\) полиномиально ограничена
\(\lVert P\rVert_{\text{ext}}\) и, по лемме 5.2, \(\lVert P \rVert\).\qed
\end{proof}

\textsc{Теорема 5.23.}\; Задача \textsc{Insecure} NP-полна для нарушителя DH. 
