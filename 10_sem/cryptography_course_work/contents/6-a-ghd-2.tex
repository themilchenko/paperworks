\section{Протокол A--GDH.2}

Протокол A--GDH.2, предложенный в работе \cite{SteinerTsudikWaidner1998},
позволяет группе участников, обладающих попарными долгосрочными ключами,
установить общий сеансовый ключ с помощью возведения в степень
по Диффи–Хеллману.
Подробное описание протокола можно найти
в~\cite{SteinerTsudikWaidner1998}.

Пусть \(P=\{1,\dots,n,I\}\)~— множество субъектов,
которые могут участвовать в одном запуске протокола A--GDH.2,
где \(I\) обозначает нарушителя
(нарушитель может быть как честным, так и нечестным субъектом).
Любая пара участников \(i,j\in P\) разделяет долгосрочный общий ключ
\(K_{ij}(=K_{ji})\).
В одном протокольном запуске
некоторая подгруппа \(G\subseteq P\) субъектов
(состав группы может меняться от запуска к запуску)
должна выработать сеансовый ключ, известный только членам группы~\(G\),
при условии, что все субъекты из \(G\) честны
(неявная аутентификация сеансового ключа).
В ходе выполнения протокола один из субъектов играет роль так называемого
\emph{мастера}.
Предположим, к примеру, что \(A,B,C,D\in P\) хотят разделить сеансовый ключ
и что мастером является \(D\).
Тогда \(A\) посылает сообщение \(B\), \(B\) — сообщение \(C\),
а \(C\) — сообщение мастеру \(D\).
После этого \(D\) вычисляет сеансовый ключ для себя и
распространяет вспомогательный материал,
позволяющий \(A,B,C\), используя свои долгосрочные ключи
с другими субъектами, получить тот же сеансовый ключ.
Обозначим \(A\) первым, \(B\) — вторым, а \(C\) — третьим участником группы.

Теперь дадим формальное описание протокола в нашей модельной схеме.
Термы \((t_{1}\langle t_{2}\dots\langle t_{n-1},t_{n}\rangle\dots\rangle)\)
будем сокращённо записывать как \(t_{1},\dots,t_{n}\).
Для \(l\in\{1,2\}\) положим, что
\(\Pi^{\,l,j}_{p,p'}\) — это \(l\)-й шаг субъекта \(p\in P\)
в~\(j\)-м экземпляре протокола, \(j\ge 0\),
когда \(p\) действует как \(l\)-й участник группы,
а \(p'\in P\) является мастером.
Отношение
\(
   \Pi^{\,1,j}_{p,p'} < \Pi^{\,2,j}_{p,p'}
\)
— единственное непустое отношение частичного порядка между правилами.

Через \(\mathit{r}^{\,p,j}\) обозначим случайное число  
(атомарное сообщение), сгенерированное субъектом \(p\)
в~\(j\)-м экземпляре, а  
\(\mathit{secret}^{\,p,j}\) — секрет  
(некоторое атомарное сообщение) того же субъекта \(p\).

Определим \(\Pi^{\,1,j}_{p,p'}\) — первый шаг субъекта \(p\)  
в~экземпляре \(j\), когда \(p\) выступает первым участником группы
(то есть инициатором протокола), а \(p'\) — мастером:

\[
1 \;\Rightarrow\; \alpha,\; Exp\!\bigl(\alpha,r^{\,p,j}\bigr)
\]

где $\alpha$ — порождающий элемент группы (атомарное сообщение).  
Для $i>1$ первое правило
\(\Pi_{i,1,p'}^{\,p,j}\) определяется так:
\[
  x_{1}^{p,j},\dots,x_{i}^{p,j}
  \; \Longrightarrow \;
  \operatorname{Exp}\!\bigl(x_{1}^{p,j},r^{\,p,j}\bigr),
  \;\dots\;,
  \operatorname{Exp}\!\bigl(x_{i-1}^{p,j},r^{\,p,j}\bigr),
  \;x_{i}^{p,j},
  \;\operatorname{Exp}\!\bigl(x_{i}^{p,j},r^{\,p,j}\bigr),
\]
где все \(x_{k}^{p,j}\) являются переменными.
Второй шаг \(\Pi_{i,2,p'}^{\,p,j}\) участника \(p\) в экземпляре \(j\)
(при \(i>0\)) задаётся правилом
\[
  y^{\,p,j}
  \;\Longrightarrow\;
  \bigl\{\mathit{secret}^{\,p,j}\bigr\}_{
  \operatorname{Exp}\!\bigl(y^{\,p,j},\,r^{\,p,j}\!\cdot\!K_{p',p}^{-1}\bigr)}^s,
\]

Заметим, что
\(\operatorname{Exp}(y^{\,p,j},\,r^{\,p,j}\!\cdot\!K^{-1}_{p,p})\)
есть сеансовый ключ, вычисляемый субъектом~\(p\);
условие неявной аутентификации ключа требует, чтобы
ни один участник, не входящий в группу, не смог получить
\(\mathit{secret}^{\,p,j}\).

Определим протокольное правило
\(M_{p_{1}\dots p_{h}}^{\,p,j}\),
описывающее поведение субъекта \(p\in P\)
в~\(j\)-м экземпляре как мастера группы
\(p_{1},\dots,p_{h},p\) (в указанном порядке),
где \(p\) — последний член группы.
Положим
\[
  \begin{aligned}
     M_{p_{1}\dots p_{h}}^{\,p,j}\;:=\;
     z_{1}^{\,p,j},\dots,z_{h+1}^{\,p,j}
     \;\Longrightarrow\;
     \operatorname{Exp}\!\bigl(z_{1}^{\,p,j},\,r^{\,p,j}\!\cdot\!K_{p_{1},p}\bigr),
     \dots,
     \operatorname{Exp}\!\bigl(z_{h}^{\,p,j},\,r^{\,p,j}\!\cdot\!K_{p_{h},p}\bigr),
     \,\\
     \bigl\{\mathit{secret}^{\,p,j}\bigr\}_{
            \operatorname{Exp}(z_{h+1}^{\,p,j},\,r^{\,p,j})}^s.
  \end{aligned}
\]

где $z_{k}^{\,p,j}$ — переменные,
$\operatorname{Exp}(z_{k}^{\,p,j},\,r^{\,p,j}\!\cdot\!K_{p_{k},\,p})$
служат ключевым материалом для $p_{k}$,
а сообщение $\operatorname{Exp}(z_{h+1}^{\,p,j},r^{\,p,j})$
является сеансовым ключом, вычисленным мастером~$p$.

Ниже задаётся протокол $P$, который описывает два запуска схемы A--GDH.2: первый —
для группы $p,p',I,p''\in P$, и второй — для группы $p,p',p''$, причём в
обоих случаях мастером является $p''$. В первом запуске действия
нарушителя $I$ можно не специфицировать. Формально множество правил
протокола $P$ состоит из следующих выражений: правила участника $p$ в
первом сеансе $\Pi^{p,1}_{1,1,p''}$ и $\Pi^{p,1}_{1,2,p''}$ (причём
$\Pi^{p,1}_{1,1,p''}<\Pi^{p,1}_{1,2,p''}$), правила участника $p'$ в том же
сеансе $\Pi^{p',1}_{2,1,p''}$ и $\Pi^{p',1}_{2,2,p''}$, а также правило
мастера $M^{\,p'',1}_{pp'I}$; правила второго сеанса —
$\Pi^{p,2}_{1,1,p''}$, $\Pi^{p,2}_{1,2,p''}$, $\Pi^{p',2}_{2,1,p''}$,
$\Pi^{p',2}_{2,2,p''}$ и $M^{\,p'',2}_{pp'}$.  Начальные знания нарушителя
равны $\{\alpha,r^{I,1}\}\cup\{K_{pI}\mid p\in P\}$.  Пусть
$\mathit{secret}$ — один из секретов, выдаваемых $p$ или $p'$ во втором
сеансе.  Поскольку нарушитель не входит во вторую группу, он не должен
получить $\mathit{secret}$. Однако, как показано в~\cite{PereiraQuisquater2001},
для $P$ существует атака, и нетрудно убедиться, что наша процедура
вывода её обнаружит.
