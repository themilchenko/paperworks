\appendixsection{Характеризация сомножителей минимальных атак}

\textsc{Лемма 3.4}
Пусть $u$ — нормализованный терм, а $M$ и $M'$ — два произведения такие,
что для любого $t\in\mathcal F(M)$ ($t\in\mathcal F(M')$, терм $t$ нормализован.
Пусть $s$ — стандартный нормализованный терм и
$\delta$ — замена $[s \leftarrow 1]$. Тогда
\begin{enumerate}
\item $\ulcorner(M\!\cdot\!M')\delta\urcorner= \,\ulcorner\ulcorner M\cdot M'\urcorner\delta\urcorner$; в частности
      $\ulcorner M\delta\urcorner = \ulcorner\ulcorner M\urcorner\delta\urcorner$;
\item $\ulcorner Exp(u,M)\delta\urcorner=
      \ulcorner\ulcorner Exp(u,M)\urcorner\delta\urcorner$
      если $s\neq\ulcorner Exp(u,M)\urcorner$, и, кроме того,
      если $s$ имеет вид $ Exp(\,\cdot\,,\cdot\,)$,
      то также $s\neq u$.
\end{enumerate}

\begin{proof}
Пункт~1 очевиден.  
Докажем пункт~2 при указанных ограничениях на~$s$.

Рассмотрим сначала случай, когда $u$ \emph{не} имеет формы
$ Exp(\,\cdot\,,\cdot\,)$.  
Тогда i) $ \ulcorner Exp(u,M) \urcorner=u$, и, следовательно, $\ulcorner M\urcorner =1$, или ii) $\ulcorner Exp(u,M)\urcorner =  Exp(u,\ulcorner M \urcorner)$
      и при этом $\ulcorner M\urcorner\neq 1$. Рассмотрим оба подслучая.

В случае~i) получаем $\ulcorner M'\urcorner^{\delta}=1$.
По пункту~1 уже известно, что
$\ulcorner M\urcorner^{\delta}= \ulcorner M'\urcorner^{\delta}(=1)$.
Следовательно
\begin{align*}
  \ulcorner Exp(u,M)\delta\urcorner
    &= \ulcorner Exp(u\delta,M\delta)\urcorner\tag{*} \\
    &= \ulcorner Exp\bigl(u\delta,\ulcorner M\delta\urcorner\bigr)\urcorner \\[2pt]
    &= \ulcorner u\delta\urcorner \\[2pt]
    &= \ulcorner\ulcorner Exp(u,M)\urcorner\delta\urcorner .
\end{align*}
Здесь в~($\ast$) использовано, что $Exp(u,M)\neq s$
(иначе $\ulcorner Exp(u,M)\urcorner = s$, поскольку $s$ нормализован).


\medskip
В Случае~ii) положим
\begin{align*}
\ulcorner\ulcorner Exp(u,M)\urcorner\delta\urcorner
   &= \ulcorner Exp\!\bigl(u,\ulcorner M\urcorner\bigr)\delta\urcorner\\
   &= \ulcorner Exp\!\bigl(u \delta,\ulcorner M\urcorner\delta\bigr)\urcorner
      \tag{*}\\
   &= \ulcorner Exp\!\bigl(u \delta,\ulcorner M\urcorner\delta\bigr)\urcorner\\
   &= \ulcorner Exp\!\bigl(u\delta,\ulcorner M\delta\urcorner\bigr)\urcorner
      \tag{**}\\
   &= \ulcorner Exp(u\delta,M\delta)\urcorner\\
   &= \ulcorner Exp(u,M)\delta\urcorner
      \tag{***}
\end{align*}
где в~$(\ast)$ мы используем соотношение $Exp(u,\ulcorner M \urcorner)\neq s$
(иначе получилось бы $\ulcorner Exp(u,M)\urcorner=\ulcorner Exp(u,\ulcorner M \urcorner)\urcorner=s$);
в~$(\ast\ast)$ ссылаемся на пункт~1 леммы;
а в~$(\ast\ast\ast)$ вновь задействуем условие $Exp(u,M)\neq s$.
Заметим, что и в случаях~(i), и в~(ii) дополнительное требование
$u\neq s$ (если $s$ имеет форму $Exp(\,\cdot\,,\cdot\,)$) не требуется.

Теперь предположим, что $u=Exp(v,M')$ для некоторых $v$ и $M'$. Тогда
\[
\begin{aligned}
  \ulcorner\ulcorner Exp(u,M)\urcorner\delta\urcorner
      &= \ulcorner\ulcorner Exp(v,\,M'\!\cdot\!M)\urcorner\delta\urcorner\\
      &= \ulcorner Exp\!\bigl(v,\,M'\!\cdot\!M\bigr)\delta\urcorner
         &&(\ast)\\
      &= \ulcorner Exp\!\bigl(v\delta,(M'\delta\cdot M\delta\bigr)\urcorner
         &&(\ast\ast)\\
      &= \ulcorner Exp\!\bigl(Exp(v\delta,M'\delta),\,M\delta\bigr)\urcorner\\
      &= \ulcorner Exp(u\delta,M\delta)\urcorner
         &&(\ast\ast\ast)\\
      &= \ulcorner Exp(u,M)\delta\urcorner.
         &&(\ast\ast\ast\ast)
\end{aligned}
\]

где $(\ast)$ выводится точно так же, как и в первом случае:
используется то, что $v$ не имеет формы $Exp(\,\cdot\,,\cdot\,)$ и
$Exp(v,M'\!\cdot\!M)\neq s$
(иначе $\ulcorner Exp(v,M'\!\cdot\!M)\urcorner=\ulcorner Exp(u,M)\urcorner=s$).
Напомним, что для первого случая предположение $v\neq s$ не требовалось.
В $(\ast\ast)$ вновь задействуется условие $Exp(v,M'\!\cdot\!M)\neq s$.
В $(\ast\ast\ast)$ используется факт $u\neq s$
(иначе $u=s$, причём $s$ имеет вид $Exp(\,\cdot\,,\cdot\,)$).
Наконец, $(\ast\ast\ast\ast)$ опирается на неравенство $Exp(u,M)\neq s$.

\end{proof}

\textsc{Лемма 3.5.} 
Пусть $\sigma$ — нормализованная замкнутая подстановка,
$E$ — множество нормализованных терминов,
$s$ — нормализованный стандартный неатомарный термин,
а $\delta$ — замена $[s \leftarrow 1]$.
Обозначим $\sigma'=\ulcorner\sigma\delta\urcorner$.
Если не существует стандартного подтермина $t\in E$ такого, что  
$t\sqsubseteq_{\sigma}s$, то $\ulcorner E\sigma'\urcorner = \ulcorner\ulcorner E\sigma\urcorner\delta\urcorner$.

\begin{proof}
Предположим, что не существует стандартного подтерма $t$ множества $E$
такого, что $t\sqsubseteq_{\sigma}s$.  
Рассмотрим множество
\(
  \Omega_{s}=\bigl\{\,t\in S(E)\mid \ulcorner t\sigma'\urcorner\neq \ulcorner\ulcorner t\sigma\urcorner\delta\bigr\}.
\)

\noindent
Допустим противное, то есть $\Omega_{s}\neq\varnothing$.
Выберем $u\in\Omega_{s}$ минимальным по отношению
\emph{подтерм}~$\sqsubseteq$.
По определению $\sigma'$ терм $u$ не может быть переменной,
а так как $s$ не атом, то $u$ не является и константой.

\medskip\noindent
Если $u=\{u_{1}\}_{u_{2}}^{s}$, $\{u_{1}\}_{u_{2}}^{a}$
или $\langle u_{1},u_{2}\rangle$, то $\ulcorner u\sigma\urcorner=s$, то $u\sqsubseteq_{\sigma}s$, противоречие.
Из минимальности $u$ получаем
$\ulcorner u_{1}\sigma'\urcorner=\ulcorner\ulcorner u_{1}\sigma\urcorner\delta\urcorner$ и $\ulcorner u_{2}\sigma'\urcorner=\ulcorner\ulcorner u_{2}\sigma\urcorner\delta\urcorner$,
откуда $\ulcorner u\sigma'\urcorner=\ulcorner\ulcorner u\sigma\urcorner\delta\urcorner$ — противоречие с $u\in\Omega_{s}$.

\medskip\noindent
Следовательно, остаётся единственная форма  
\(
  u=Exp(u_{1},M).
\)
Минимальность $u$ даёт
$\ulcorner v\sigma'\urcorner=\ulcorner\ulcorner v\sigma\urcorner\delta\urcorner$ для всех $v\in\mathcal F(u)$, и, значит,
\[
  Exp(\ulcorner u_{1}\sigma'\urcorner,\ulcorner\,M\sigma'\urcorner)
    =Exp(\ulcorner\ulcorner u\sigma\urcorner\delta\urcorner,\ulcorner\ulcorner\,M\sigma\urcorner\delta\urcorner).
\]

Поскольку $u\sigma\neq s$ и $u_{1}\sigma\neq s$,  
из пункта 2 леммы 3.4 получаем требуемое равенство.
\end{proof}

\textsc{Лемма 3.6.} 
Пусть $t',\,t_1,\dots,t_n,\,t,\,u$ — нормализованные стандартные термины,
$z_1,\dots,z_n\in\mathbb Z$, а $\delta$ — замена $[u\leftarrow1]$ такая, что
$u\neq t$ и $t=\ulcorner Exp\bigl(t',\,t_1^{\,z_1}\!\cdots t_n^{\,z_n}\bigr)\urcorner$.
Если $t'=Exp(\,\cdot,\cdot\,)$, то дополнительно предполагаем
$u\neq t'$. Тогда
\[
  \ulcorner t\delta\urcorner
  =\ulcorner Exp\bigl(\ulcorner t'\delta\urcorner,
        \ulcorner t'\delta\urcorner^{z_1}\!\cdots \ulcorner t_n\delta\urcorner^{z_n}\bigr)\urcorner.
\]

\begin{proof}
Теперь предположим, что $u=Exp(v,M')$ для некоторых нормализованных
$v$ и $M'$.  Заметим, что $v\neq Exp(\,\cdot\,,\cdot\,)$, поскольку $t'$ нормализован. Тогда
\[
\begin{aligned}
\ulcorner Exp\!\bigl(t'\delta,\,
          \ulcorner t_1\delta\urcorner^{z_1}\!\cdots \ulcorner t_n\delta\urcorner^{z_n}\bigr)\urcorner
   &=\ulcorner Exp\!\bigl(\ulcorner v\delta\urcorner,\,
 \ulcorner M\delta\urcorner\!\cdot \ulcorner t_1\delta\urcorner^{z_1}\!\cdots \ulcorner t_n\delta\urcorner^{z_n}\bigr)\urcorner
     &&(\ast)\\
   &=\ulcorner Exp\!\bigl(v\delta,\,
          M\delta\!\cdot (t_1\delta)^{z_1}\!\cdots (t_n\delta)^{z_n}\bigr)\urcorner\\[4pt]
   &=\ulcorner Exp\!\bigl(v,\,
          M\!\cdot t_1^{z_1}\!\cdots t_n^{z_n}\bigr)\delta\urcorner
     &&(\ast\ast)\\
   &=\ulcorner\ulcorner Exp\!\bigl(v,\,
          M\!\cdot t_1^{z_1}\!\cdots t_n^{z_n}\bigr)\urcorner\delta\urcorner
     &&(\ast\ast\ast)\\
   &=\ulcorner t\delta\urcorner.
\end{aligned}
\]
где в~$(\ast)$ используем, что $u\neq t'$, а значит
$\ulcorner t'\delta\urcorner=\ulcorner v^{\delta}\urcorner$ (при этом $\ulcorner M\delta\urcorner=1$) \;или\;
$\ulcorner t'\delta\urcorner= Exp\!\bigl(\ulcorner v\delta\urcorner,\ulcorner M\delta\urcorner\bigr)$.
В~$(\ast\ast)$ задействуется условие $u\neq t$:
если бы $u= Exp\!\bigl(v,\,M\!\cdot t_1^{z_1}\!\cdots t_n^{z_n}\bigr)$,
то, поскольку $u$ нормализован, нормализован и
$ Exp\!\bigl(v,\,M\!\cdot t_1^{z_1}\!\cdots t_n^{z_n}\bigr)$,
а значит
\(
  u=Exp\!\bigl(v,\,M\!\cdot t_1^{z_1}\!\cdots t_n^{z_n}\bigr)=t,
\)
что противоречит $u\neq t$.
В~$(\ast\ast\ast)$ используем, что $v\neq Exp(\,\cdot\,,\cdot\,)$,
$u\neq t$, а также пункт~2 леммы~3.4.

\smallskip
Предположим теперь, что $t'\neq Exp(\,\cdot\,,\cdot\,)$.
Для обеих ситуаций $u\neq t'$ и $u=t'$ рассуждения
аналогичны приведённым выше:
достаточно заменить $v$ на $t'$ и опустить
$\ulcorner M\delta\urcorner$, $M\delta$ и $M$ во всех формулах.
\end{proof}

\textsc{Лемма 3.12.}
Пусть $t\in forge(E)$ и $\gamma\in forge(E)$.
Предположим, что имеется вывод $D_{\gamma}$ из $E$,
последним шагом которого является применение правила из $\mathcal L_{c}$.
Тогда существует вывод $D'$ из $E$ с целью $t$, удовлетворяющий
\(
  L_{d}(\gamma)\notin D'.
\)

\begin{proof}
Сначала введём некоторое обозначение.  
Пусть $D_{1}=E_{1}\to\cdots\to F_{1}$ и
$D_{2}=E_{2}\to\cdots\to F_{2}$ — две
дедукции, причём $E_{2}\subseteq F_{1}$.
Обозначим через $D=D_{1}\!\cdot D_{2}$ конкатенацию шагов $D_{1}$ и $D_{2}$
(в указанном порядке); при этом из $D_{2}$ удаляются те шаги,
которые выводят сообщения, уже содержащиеся в $F_{1}$,
чтобы результат оставался дедукцией.

По определению дедукции правило $L_{d}(\gamma)$ отсутствует в $D_{\gamma}$.
Пусть $D$ — это $D_{\gamma}$ без последнего шага, то есть
$D_{\gamma}$ состоит из $D$, за которым следует некоторое
$L\in\mathcal L_{c}$.
Положим
\(
  D'' \;=\;
  D \;.\;
  Deriv_{t}(E)
  \;=\;
  D\;.\;D'''
\)
где $D'''$ получается из $Deriv_{t}(E)$ удалением избыточных шагов.
Тогда $D''$ — дедукция с целью $t$.
Рассмотрим два случая.

\medskip\noindent
—\;Пусть $L=L_{c}(\gamma)$.
Тогда $L_{d}(\gamma)\notin D''$, поскольку оба непосредственных
подтерма $\gamma$ были выведены уже в $D$.
Следовательно, $D' = D''$ — искомая дедукция.

\smallskip\noindent
—\;Пусть $L=L_{oc}(\gamma)$.
Если $L_{d}(\gamma)\notin D''$, делать больше нечего.
В противном случае обозначим через $F_{1}$ конечное множество сообщений
дедукции $D$.
Из пункта~(2) определения~2.11 следует, что каждый шаг из $D'''$
вида
\(
  F_{1},F_{2},\gamma
    \;\longrightarrow^{\,L_{d}(\gamma)}\;
  F_{1},F_{2},\gamma,\beta
\)
можно заменить дедукцией из $F_{1},F_{2}$ c целью $\beta$,
в которой правило $L_{d}(\gamma)$ не встречается.
После такой замены и последующего устранения избыточных
шагов получаем требуемую дедукцию~$D'$.
\end{proof}

\appendixsection{Расширение нарушителя Долева–Яо
                     возведением в степень Диффи–Хеллмана}

\textsc{Лемма 4.3.}
Пусть \(E\) — конечное множество нормализованных стандартных сообщений,
\(t\) — стандартное сообщение такое, что \(t\) выводится из \(E\) (относительно \(L\)).
Пусть \(D\) — вывод из \(E\) с целью \(t\).
Тогда существует вывод \(D'\) из \(E\) с той же целью \(t\), удовлетворяющий:

\begin{enumerate}
  \item \(D'\) имеет ту же длину, что и \(D\);
  \item для любого DH-правила \(L\in D'\cap L_{o}\) с головным термом \(t'\)
        выполнено: \(t'\in E\) \;или\;  
        существует \(t'\)-правило \(L'\in D'\cap(L_{d}\cup L_{c})\).
        Более того, если \(L\) — \emph{декомпозиционное} DH-правило,
        то \(t'\in E\) \;или\;
        существует \(t'\)-правило \(L'\in D'\cap L_{d}\).
\end{enumerate}

\begin{proof}
Пусть $D$ — вывод из $E$ с целью $t$.  
Построим по нему вывод $D'$ следующим образом.  
Пусть $L\in D\cap L_{o}$ имеет вершину $t'$ и при этом
\(
  t'\in E,
  D'\cap (L_{d}\cup L_{c})\ \text{не содержит $t'$-правил.}
\)
Тогда в $D$ найдётся правило $L'\in D\cup L_{o}(t')$.
Пусть
\(
  L:\;
  t',t_{1},\dots,t_{n}\;\longrightarrow\;t,
  \qquad
  z_{1},\dots,z_{n},
\)
а
\(
  L':\;
  t'',t'_{1},\dots,t'_{n}\;\longrightarrow\;t',
  \qquad
  z'_{1},\dots,z'_{n}.
\)
Очевидно,
\(
  t
  =\ulcorner Exp\!\bigl(
       t'',\,
       t_{1}^{z_{1}}\!\cdots t_{n}^{z_{n}}
       \,t'_{1}{}^{z'_{1}}\!\cdots t'_{n}{}^{z'_{n}}
     \bigr)\urcorner,
\)
и потому $t$ может быть получен правилом
\(
  \widehat{L}:\;
  t'',t_{1},\dots,t_{n},t'_{1},\dots,t'_{n}\;\longrightarrow\;t,
\)
относящимся к множеству $DH$ и имеющим вершину $t''$.
Следовательно, в выводе $D$ правило $L$ можно заменить
на~$\widehat{L}$.  Итеративно выполняя такую замену,
получаем вывод $D'$, удовлетворяющий условию~1,
и при этом для любого $L\in D'\cap L_{o}$ с вершиной $t'$
не существует предшествующего ему в $D'$ правила
$L'\in D'\cap L_{o}(t')$,
что и даёт условие~2.
Заметим, что если $L$ является декомпозиционным
правилом $DH$, то $t'$ имеет форму $Exp(\,\cdot\,,\cdot\,)$
и, следовательно, не может быть порождён ни одним правилом из~$L_{c}$.
\end{proof}

\textsc{Лемма 4.4.}
Пусть
\(D = E_0 \rightarrow_{\,L_1\,} \dots
      E_{n-1} \rightarrow_{\,L_n\,} E_n\)
— вывод с целью \(g\).

\begin{enumerate}
\item
  Предположим, что для каждого шага
  \(E_{j-1} \rightarrow{\,L_j\,} E_j\) вывода \(D\)
  с \(L_j \in \mathcal L_{d}(t)\)
  существует \(t'\in E_{j-1}\) такое, что
  \(t \sqsubseteq t'\) и
  \(t'\in E_0\) \;или\;
  \(\exists\,i<j: L_i\in \mathcal L_{d}(t')\).
  Тогда из \(L\in D\cap \mathcal L_{d}(t)\)
  (для некоторого \(\mathcal L,t\)) следует
  \(t\in \mathcal S(E_0)\).

\item
  Предположим, что для каждого \(i<n\) и \(t\) с
  \(L_i\in \mathcal L_{c}(t)\)
  найдётся \(j>i\) такое, что
  \(L_j\) является \(t'\)-правилом
  и \(t\in \mathcal S\bigl(\{t'\}\cup E_0\bigr)\).
  Тогда из \(L\in D\cap \mathcal L_{c}(t)\)
  (для некоторого \(L,t\)) следует
  \(t\in \mathcal S(E_0,g)\).
\end{enumerate}

При выполнении обеих предпосылок (а) и (б)
вывод \(D\) является корректным 
выводом с целью \(g\).

\begin{proof}
а. Доказывается непосредственной индукцией по
$j\in\{1,\dots,n\}$.
б. Пусть выполнены предположения пункта 2.  
Докажем индукцией по $\,n-i\,$, что для любого
$i\in\{1,\dots,n\}$ из $L_{i}\in\mathcal L_{c}(t)$ следует
$t\in S(E_{0},g)$.
$n-i=0$.  
Тогда $t=g$, а значит $t\in S(E_{0},g)$.
Предположения пункта а дают
$j>i$ такие, что $L_{j}$ — $t'$-правило и
$t\in S(E_{0},t')$.
Если $L_{j}\in\mathcal L_{d}(t')$, то
$t'\in S(E_{0})$ (см. выше).
Если же $L_{j}\in\mathcal L_{c}(t')$, то по
индукционному предположению $t'\in S(E_{0},g)$,
а значит и $t\in S(E_{0},g)$.
Из предположений пунктов а и б немедленно
следует, что $D$ является корректной
дедукцией с целью~$g$.
\end{proof}

\textsc{Лемма 4.6.}
Пусть $z_{1},\dots ,z_{n}\in\mathbb Z\setminus\{0\}$,
а $s,s_{1},\dots ,s_{n}$ — нормализованные стандартные термины,
удовлетворяющие условиям
$s_{i}\neq s_{j}$ при $i\neq j$, 
$s_{i}\neq1$ и $s_{i}\neq u$ для всех $i$,
$s\neq u$,
\(
  u=\ulcorner Exp\bigl(s,\,
       s_{1}^{\,z_{1}}\!\cdots s_{n}^{\,z_{n}}\bigr)\urcorner,
  u=Exp(\,\cdot,\cdot\,).
\)
Пусть $\delta$ — замена $[\,u\!\rightarrow\!2\,]$.
Тогда
\(
  u
  =
  \ulcorner Exp\bigl(\ulcorner s\delta\urcorner,\,
        \ulcorner s_{1}\delta\urcorner^{z_{1}}\!\cdots
        \ulcorner s_{n}\delta\urcorner^{z_{n}}\bigr)\urcorner.
\)

\begin{proof}
Сначала предположим, что $s\neq Exp(\,\cdot\,,\cdot\,)$.  
Тогда $u=Exp(s,\,s_{1}^{z_{1}}\!\cdots s_{n}^{z_{n}})$.  
Отсюда следует, что $u\notin S(s,s_{1},\dots,s_{n})$,  
а значит $s=s^{\delta}$ и $s_{i}=s_{i}^{\delta}$ для всех $i$.  
Следовательно,
\(
  u \;=\; \ulcorner Exp\!\bigl(\ulcorner s\delta\urcorner,
                      \ulcorner s_{1}\delta\urcorner^{z_{1}}\!\cdots \ulcorner s_{n}\delta\urcorner^{z_{n}}\bigr).
\)

Теперь предположим, что $s=Exp(v,M)$.  
Так как $s$ нормализован, имеем $v\neq Exp(\,\cdot\,,\cdot\,)$.  
Используя, что $u$ имеет вид $Exp(\,\cdot\,,\cdot\,)$, получаем
\(
  u=Exp\!\bigl(v,\ulcorner\,M\!\cdot s_{1}^{z_{1}}\!\cdots s_{n}^{z_{n}}\urcorner\bigr),
  \ulcorner M\!\cdot s_{1}^{z_{1}}\!\cdots s_{n}^{z_{n}}\urcorner\neq1,
  u\notin S(v).
\)
Положим $E=\mathcal F(M)\cup\{s_{1},\dots,s_{n}\}$.
Тогда найдутся подмножество
$E'=\{\,s'_{1},\dots,s'_{n}\}\subseteq E$
и целые ненулевые $z'_{1},\dots,z'_{n}$ такие, что
\(
  u=Exp\!\bigl(v,\,s'_{1}{}^{z'_{1}}\!\cdots s'_{n}{}^{z'_{n}}\bigr),
  u\notin S\bigl(v,E'\bigr).
\)

\medskip\noindent
\textit{Утверждение.}\;
Выполняется равенство
\[
  M^{\delta}\!\cdot s_{1}^{\delta\,z_{1}}\!\cdots s_{n}^{\delta\,z_{n}}
  \;=\;
  s'_{1}{}^{\,z'_{1}}\!\cdots s'_{n}{}^{\,z'_{n}}.
\]

\medskip\noindent
\textit{Доказательство утверждения.}
Пусть 
\(
  M = s_{n+1}^{\,z_{n+1}}\!\cdots s_{n''}^{\,z_{n''}}
\).
Определим множества индексов
\[
  C_{i} \;=\; \bigl\{\,j\in\{1,\dots,n''\}\mid s_{j}=s'_{i}\bigr\}
  \quad (i=1,\dots,n).
\]
Тогда
\(
  z'_{i} \;=\; \sum_{j\in C_{i}} z_{j},
  \qquad
  C \;=\; \bigcup_{i=1}^{n'} C_{i}.
\)

\(
  s_{1}^{z_{1}}\!\cdots s_{n}^{z_{n}}\,
  M
  \;=\;
  \ulcorner\prod_{i=1}^{n'}\prod_{j\in C_{i}} s_{j}^{z_{j}}\urcorner
  \;=\;
  \ulcorner\prod_{i=1}^{n'} {s'_{i}}^{\,z'_{i}}\urcorner,
  \ulcorner\prod_{j\notin C} s_{j}^{z_{j}}\urcorner = 1.
\)
Так как все $s_{j}$ нормализованы, получаем
\(
  \ulcorner\displaystyle \prod_{j\notin C} s_{j}^{\delta\,z_{j}}\urcorner
  =\prod_{j\notin C} \ulcorner s_{j}^{\delta z_{j}\urcorner}
  =1,
\)
а потому, учитывая ${s'}_{i}^{\delta}={s'}_{i}$, равенство утверждения
доказано:
\[
  M^{\delta}\,
  s_{1}^{\delta z_{1}}\!\cdots s_{n}^{\delta z_{n}}
  \;=\;
  {s'_{1}}^{\,z'_{1}}\!\cdots{s'_{n}}^{\,z'_{n}} .
\]

Используя теперь $s\delta=Exp(v\delta,M\delta)$ и факты
$u\neq M$, $u\notin S(v)$, $u\neq s$, получаем
\[
  \ulcorner Exp\!\bigl(
      \ulcorner s\delta\urcorner,
      \ulcorner s_{1}\delta\urcorner^{z_{1}}\!\cdots \ulcorner s_{n}\delta\urcorner^{z_{n}}
  \bigr)\urcorner
  =\ulcorner Exp\!\bigl(
      \ulcorner v\delta\urcorner,
      \ulcorner M\delta\urcorner\,
      \ulcorner s_{1}\delta\urcorner^{z_{1}}\!\cdots \ulcorner s_{n}\delta\urcorner^{z_{n}}
  \bigr)
  =u.\qedhere
\]
\end{proof}

\appendixsection{Правила DH допускают атаки с полиномиально ограниченными показателями произведений}

\textsc{Лемма 5.11.}
Для любого открытого сообщения или продукта $t$,
такого что $\beta(t)=\ulcorner\beta(t)\urcorner$, выполнено $t^{\beta}=t$.

\begin{proof}
Доказательство проводится по структурной индукции по~$t$.
Если $t$ является атомом, парой либо шифрованием,
утверждение очевидно.

Пусть $t=t_{1}^{e_{1}}\!\cdots t_{n}^{e_{n}}$.
Из свойств отображения оценки имеем:
$\beta(t_{i})=\ulcorner\beta(t_{i})\urcorner\neq1$ для всех $i$,
$\beta(t_{i})\neq\beta(t_{j})$ при $i\neq j$,
и $\beta(e_{i})\neq0$ для всех $i$.
Отсюда непосредственно следует, что $t^{\beta}=t$.

Наконец, пусть $t= Exp(u,M)$.  
Сначала заметим, что $\ulcorner\beta(u)\urcorner\neq Exp(u',M')$
для любых нормализованных $u'$ и $M'$.
Действительно, иначе
\(
  \ulcorner \beta(t)\urcorner= Exp\!\bigl(\beta(u),\beta(M)\bigr)
          = Exp(u',M'')=\beta(t),
\)
и, следовательно, $\beta(u)=u'$.
Тогда
\(
   Exp(u',M')=\ulcorner\beta(u)\urcorner=\ulcorner u'\urcorner = u',
\)
что противоречит нормализованности $u'$.

Кроме того, $\ulcorner\beta(M)\urcorner\neq1$, поскольку иначе
\(
   Exp\!\bigl(\beta(u),\beta(M)\bigr)
     =\ulcorner\beta(t)\urcorner=\ulcorner\beta(u)\urcorner,
\)
а мы уже знаем, что $\ulcorner\beta(u)\urcorner\neq Exp(\,\cdot\,,\cdot\,)$.
Тем самым
\(
   Exp\!\bigl(\beta(u),\beta(M)\bigr)=\ulcorner\beta(t)\urcorner
                                     = Exp\!\bigl(\ulcorner\beta(u)\urcorner,\ulcorner\beta(M)\urcorner\bigr),
\)
и значит $\beta(u)=\ulcorner\beta(u)\urcorner$ и $\beta(M)=\ulcorner\beta(M)\urcorner$.
По индукционному предположению отсюда следует
$u^{\beta}=u$ и $M^{\beta}=M$.
По определению $t^{\beta}$ получаем
\(
  t^{\beta}= Exp(u^{\beta},M^{\beta})
          = Exp(u,M)=t.
\)
\end{proof}

\textsc{Лемма 5.13.}
Пусть $E$ — конечное множество открытых сообщений или продуктов,
такое что $S_{\text{ext}}(E)=E$, а $t$ — максимальный
(относительно упорядочения «строгий подтермин»)
элемент множества $E$.
Тогда
\[
  \bigl\lvert\,
        \bigcup_{s\in E} S_{\text{ext}}\!\bigl(s^{\beta}\bigr)
        \Bigr{\rvert}_{\text{exp}}
  \;\le\;
  \bigl\lvert\,
        \bigcup_{s\in E\setminus\{t\}}
            S_{\text{ext}}\!\bigl(s^{\beta}\bigr)
            \Bigr{\rvert}_{\text{exp}}
  +\lVert t\rVert_{\text{ext}}^{\,2}.
\]

Чтобы доказать эту лемму, нам сначала нужно ограничить размер $\lvert\cdots\rvert_{exp}$
произведения. Это про следующее утверждение:

\textsc{Утверждение 1}. Для любого открытого сообщения или продукта $t$, такого что
$t^{\beta}= Exp(u,M)$, выполняется
\[
   \lvert M\rvert_{\exp}
   \;\le\;
   \lvert t\rvert\;\cdot\;\lVert t\rVert_{\exp}
   \;\le\;
   \lVert t\rVert_{\text{ext}}^{\,2}.
\]

\begin{proof}
Докажем индукцией по структуре $t$, что
$\lvert M\rvert_{\exp}\le\lvert t\rvert\cdot\lVert t\rVert_{\exp}$.
Так как $\lvert t\rvert\le\lVert t\rVert_{\text{ext}}$ и
$\lVert t\rVert_{\exp}\le\lVert t\rVert_{\text{ext}}$,
утверждение леммы будет следовать сразу.

\smallskip
Сперва положим, что $t= Exp(u',M')$ и $\beta(u')\neq Exp(\,\cdot\,,\cdot\,)$.
Тогда
\(
   \lvert M\rvert_{\exp}
   \;\le\;
   \lvert M'\rvert_{\exp}
   \;\le\;
   \lVert t\rVert_{\exp}.
\)

\smallskip
Сейчас положим $t= Exp(u',M')$ и $\beta(u')= Exp(\,\cdot\,,\cdot\,)$.
Пусть $u'^{\beta}= Exp(u'',M'')$.
Тогда $M=(M''\!\cdot M')^{\beta}$ и, по индукционному предположению,
\(
   \lvert M\rvert_{\exp}
      \;\le\;
      \lvert M''\rvert_{\exp}+\lvert M'\rvert_{\exp}
      \;\le\;
      \lvert u'\rvert\,\lVert u'\rVert_{\exp}
        +\lVert t\rVert_{\exp}
      \;\le\;
      \lvert t\rvert\,\lVert t\rVert_{\exp},
\)
поскольку $\lVert u'\rVert_{\exp}\le\lVert t\rVert_{\exp}$
и $\lvert u'\rvert<\lvert t\rvert$.

\smallskip
В остальных случаях, если $t^{\beta}= Exp(u,M)$ получается для другого вида $t$,
то существует подтерм $v\sqsubset t$ такой, что $v^{\beta}=t^{\beta}$.
Неравенство тогда следует из индукционного предположения для~$v$.
\end{proof}

Теперь можно приступить к доказательству Леммы 5.13:

\begin{proof}
Доказательство проводим, как и прежде, структурной индукцией по~$t$.
Пусть $E=S_{\text{ext}}(t)$ и обозначим
\(
  \mathcal S(E)=\bigl\lvert\,
        \bigcup_{s\in E} S_{\text{ext}}\!\bigl(s^{\beta}\bigr)
      \bigr\rvert_{\exp}.
\)

Если \(t=\langle t_{1},t_{2}\rangle\), то \(t^{\beta}=t_{1}^{\beta},t_{2}^{\beta}\) и 
\(
  S_{\text{ext}}(t^{\beta})
  \subseteq
  \{t^{\beta}\}\cup
  S_{\text{ext}}(t_{1}^{\beta})\cup
  S_{\text{ext}}(t_{2}^{\beta}).
\)
Поскольку \(S_{\text{ext}}(E)=E\), имеем \(t_{1},t_{2}\in E\setminus\{t\}\).
Так как \(|t^{\beta}|_{\exp}=0\),
\(
  \Bigl\lvert\!\!\bigcup_{s\in E}
     S_{\text{ext}}(s^{\beta})\Bigr\rvert_{\exp}
  =
  \Bigl\lvert\!\!\bigcup_{s\in E\setminus\{t\}}
     S_{\text{ext}}(s^{\beta})\Bigr\rvert_{\exp}.
\)
Для случая шифрования рассуждение аналогично.

Если \(t=t_{1}^{e_{1}}\!\cdots t_{n}^{e_{n}}\),
то
\(
  S_{\text{ext}}(t^{\beta})
  \subseteq
  \{t^{\beta}\}\cup\bigcup_{i} S_{\text{ext}}(t_{i}^{\beta}),
\)
причём \(t_{1},\dots,t_{n}\subseteq E\setminus\{t\}\).
Следовательно,
\(
  S_{\text{ext}}(t^{\beta})
  \subseteq
  \Bigl(\bigcup_{s\in E\setminus\{t\}}
          S_{\text{ext}}(s^{\beta})\Bigr)
  \cup\{t^{\beta}\}.
\)
Так как \(|t^{\beta}|_{\exp}\le |t|_{\exp}\le\|t\|_{\text{ext}}^{2}\),
утверждение верно.

Пусть теперь \(t=Exp(u,M)\) и
\(
  t^{\beta}=u^{\beta}\quad\text{или}\quad
  t^{\beta}=u''\quad\text{или}\quad
  t^{\beta}=Exp(u^{\beta},M^{\beta})
\)
(см. лемму 5.10).
Тогда
\(
  S_{\text{ext}}(t^{\beta})
  \subseteq
  \bigl\{t^{\beta}\bigr\}
  \cup S_{\text{ext}}(u^{\beta})
  \cup S_{\text{ext}}(M^{\beta}),
\)
и опять
\(
  \Bigl\lvert\!\!\bigcup_{s\in E}
     S_{\text{ext}}(s^{\beta})\Bigr\rvert_{\exp}
  =
  \Bigl\lvert\!\!\bigcup_{s\in E\setminus\{t\}}
     S_{\text{ext}}(s^{\beta})\Bigr\rvert_{\exp},
  \qquad |t^{\beta}|_{\exp}=0.
\)

Наконец, пусть
\(
  t= Exp(u,M),\quad
  \beta(u)= Exp(u',M'),\quad
  u^{\beta}= Exp(u'',M'').
\)
Тогда
\(
  t^{\beta}= Exp\!\bigl(u'',(M''\!\cdot M)^{\beta}\bigr),
\)
и
\[
\begin{aligned}
  S_{\text{ext}}(t^{\beta})
  &\subseteq
    \{t^{\beta}\}
    \cup S_{\text{ext}}(u^{\beta})
    \cup S_{\text{ext}}\!\bigl((M''\!\cdot M)^{\beta}\bigr) \\[2pt]
  &\subseteq
    \{t^{\beta}\}
    \cup\{(M''\!\cdot M)^{\beta}\}
    \cup\bigcup_{s\in E\setminus\{t\}} S_{\text{ext}}(s^{\beta}),
\end{aligned}
\]
где, по лемме 5.11,
\(S_{\text{ext}}(M''^{\beta})=S_{\text{ext}}(M'')\subseteq S_{\text{ext}}(u^{\beta})\).
Поскольку \(|t^{\beta}|_{\exp}=0\) и
\(|(M''\!\cdot M)^{\beta}|_{\exp}\le\|t\|_{\text{ext}}^{2}\),
получаем требуемую оценку.
\end{proof}

\textsc{Лемма 5.17.}
Пусть 
\(M = t_{1}^{e_{1}}\!\cdots t_{n}^{e_{n}}\)
и
\(M' = t_{1}'^{\,e_{1}'}\!\cdots t_{n'}'^{\,e_{n'}'}\) —
два открытых продукта такие, что 
\(\beta(t_{i}) = \ulcorner\beta(t_{i}')\urcorner\)
для всех соответствующих множителей.
Пусть для каждого \(i\) задан $\beta$-кортеж 
\((t_{i}^{\beta},\mathcal E_{i})\) терма \(t_{i}\).
Тогда пара
\(
\Bigl((M'\!\cdot M)^{\beta},\;
      \mathcal E_{(M'\!\cdot M)}^{\prime\,\beta}
      \,\cup\!
      \bigcup_{i}\mathcal E_{i}\Bigr)
\)
является $\beta$-кортежем для продукта \(M'\!\cdot M\).

\begin{proof}
Пусть $t=M'\cdot M$. Очевидно, 
\(
  \beta\models\mathcal E'^{\,\beta}_{t}\;\cup\!\bigcup_i\mathcal E_i,
  \quad
  \beta(t^\beta)=\ulcorner\beta(t)\urcorner.
\)
Пусть $\beta'\models\mathcal E'^{\,\beta}_{t}\;\cup\!\bigcup_i\mathcal E_i$.  
Нужно показать, что $\ulcorner\beta'(t^\beta)\urcorner=\ulcorner\beta'(t)\urcorner$.

\medskip
\noindent\textit{Утверждение I.}
Для любых $i,j$ справедливо
\(
  \ulcorner\beta'(t_i)\urcorner=\ulcorner\beta'(t_i^\beta)\urcorner,
  \ulcorner\beta'(t'_j)\urcorner=\ulcorner\beta'(t'_j{}^\beta)\urcorner.
\)

\emph{Доказательство утверждения 1.}
Из $\beta'\models\mathcal E_i$ сразу следует
$\ulcorner\beta'(t_i)\urcorner=\ulcorner\beta'(t_i^\beta)\urcorner$.  
По лемме 5.11 $t'_j{}^\beta=t'_j$, следовательно
$\ulcorner\beta'(t'_j)\urcorner=\ulcorner\beta'(t'_j{}^\beta)\urcorner$.

Пусть классы $C_1,\dots,C_\ell$ заданы так же, как в доказательстве леммы 5.10.

\medskip
\noindent\textit{Утверждение II.}
Для любого $k$ и любых $s,s'\in C_k$ выполнено
\(
  \ulcorner\beta'(s)\urcorner=\ulcorner\beta'(s')\urcorner.
\)

\emph{Доказательство утверждения 2.}
Во-первых, если $s=t'_i$ и $s'=t'_j$, то по лемме 5.11
$s^\beta=s$, $s'^\beta=s'$, и из определения
$\mathcal E'^{\,\beta}_{t}$ следует $\beta'(s)=\beta'(s')$.  
Пусть $s=t_i$, $s'=t'_j$. Тогда $s'^\beta=s'$, и по определению
$\mathcal E'^{\,\beta}_{t}$ имеем
$\ulcorner\beta'(s')\urcorner=\ulcorner\beta'(s^\beta)\urcorner$. А из $\beta'\models\mathcal E_i$ следует
$\ulcorner\beta'(s)\urcorner=\ulcorner\beta'(s^\beta)\urcorner$, откуда $\ulcorner\beta'(s')\urcorner=\ulcorner\beta'(s)\urcorner$.  
Аналогично рассматривается случай $s=t_i$, $s'=t_j$.

\medskip
По определению $\mathcal E'^{\,\beta}_{t}$ для каждого $j\in J$ имеет место
$\beta'(e_{C_j})=0$.  
Используя Утверждения I и II, легко заключить, что
\(\ulcorner\beta'(t^\beta)\urcorner=\ulcorner\beta'(t)\urcorner\).
\end{proof}

\textsc{Лемма 5.20.}
Пусть $t_{1},\dots,t_{n}$ — открытые сообщения,
$s$ — нормализованное сообщение,
$\beta$ — отображение оценки,
а $L\in\mathcal L$ — правило нарушителя такое, что
$\beta(t_{i})=\ulcorner\beta(t_{i})\urcorner$ для всех~$i$
и
$\ulcorner\beta(t_{1})\urcorner,\dots,
 \ulcorner\beta(t_{n})\urcorner \;\rightarrow\; s\in L$.
Тогда существует открытое сообщение $t$,
система линейных уравнений~$\mathcal E$
и расширение $\beta$ отображения $\beta$ такие, что
\begin{enumerate}\itemsep0pt
  \item[\textup{(1)}] $\beta(t)=s$;
  \item[\textup{(2)}] $\beta \models \mathcal E$;
  \item[\textup{(3)}] $\ulcorner\beta'(t_{1})\urcorner,\dots,
        \ulcorner\beta'(t_{n})\urcorner
        \;\rightarrow\;
        \ulcorner\beta'(t)\urcorner$
        для всякого $\beta'\models\mathcal E$;
  \item[\textup{(4)}] $\displaystyle
        \max\bigl\{|e|\;\bigm|\;e\in \mathcal L_{\text{exp}}(t)\bigr\}
        \;\le\;
        \max\bigl\{|e|\;\bigm|\;e\in \mathcal L_{\text{exp}}(t_{1},\dots,t_{n})\bigr\}
        + n$;
  \item[\textup{(5)}] размер $\mathcal E$ полиномиально ограничен
        величиной $\lVert t_{1},\dots,t_{n}\rVert_{\text{ext}}$.
\end{enumerate}
Чтобы доказать лемму, сначала докажем следующее утверждение:

\textit{Утвердждение.} Пусть $t_{0},\dots,t_{n}$ — открытые сообщения и $\beta$ — отображение оценки, такое что 
\(
  \beta(t_{i})=\ulcorner\beta(t_{i})\urcorner
  \quad\text{для всех }i.
\)
Пусть $z_{1},\dots,z_{n}$ — целочисленные переменные, не встречающиеся в термах $t_{i}$.  
Обозначим 
\(
  M \;=\; t_{1}^{z_{1}}\cdots t_{n}^{z_{n}},
  \qquad
  t \;=\; Exp\bigl(t_{0},M\bigr)^{\beta}.
\)
Тогда каждый коэффициент в линейных уравнениях, появляющихся в $t$, не превосходит
\(
  \max\!\bigl\{|e|\;\bigm|\;e\in \mathcal L_{\exp}(t_{0},t_{1},\dots,t_{n})\bigr\}\;+\;n.
\)

\begin{proof}
По лемме 5.11 все $t_{i}^{\beta}=t_{i}$.  
Рассмотрим два случая.

\medskip\noindent
Пусть 
\(\beta(t_{0})=\ulcorner\beta(t_{0})\urcorner\neq Exp(\,\cdot\,,\cdot\,).\)  
Тогда $t= Exp(t_{0},M)^{\beta}$ не вносит новых показателей, и утверждение очевидно.

Пусть 
\(\beta(t_{0})= \ulcorner\beta{t_0} = Exp(u',M')\), 
т.\,е. $t_{0}= Exp(u'',M'')$ для некоторых $u'',M''$ с 
\(\beta(u'')=u'\) и \(\beta(M'')=M'\).  
Если $t=u''$, то новые показатели отсутствуют и предельная оценка выполняется тривиально.  
Иначе
\(
  t
  =  Exp\!\bigl(u'',\,(M''\cdot M)^{\beta}\bigr),
  \quad
  (M''\cdot M)^{\beta}\neq1.
\)
Пусть 
\(\displaystyle M''=t''_{1}{}^{e_{1}^{''}}\cdots t''_{n}{}^{e_{n}^{''}}\).  
По условию $\beta(t''_{i})=\ulcorner\beta(t''_{i})\urcorner$ при всех $i$,
и все $\beta(t''_{i})$ попарно различны.  
Разобьём индексы на классы эквивалентности $C_{1},\dots,C_{\ell}$, как
в доказательстве леммы 5.10. Тогда
\(
 t = \prod_{j\notin J\cup\{1\}}
     \bigl(s_{C_{j}}^{\beta}\bigr)^{e_{C_{j}}},
\)
где в каждом классе $C_{j}$ есть ровно один представитель вида $t''_{i}$ или $t_{i}$.  
Поэтому для каждого $j$ 
\(\lvert e_{C_{j}}\rvert\)
не превосходит
\(\max\{\lvert e\rvert\mid e\in L_{\exp}(t_{0},t_{1},\dots,t_{n})\}+n\).  
Остальные подтермы $t$ лежат внутри некоторых $t_{i}$, и для них оценка
задана индукционным предположением.  
Это завершает доказательство утверждения.
\end{proof}

Теперь докажем Лемму 5.20:

\begin{proof}
Рассмотрим случаи в зависимости от правила нарушителя \(L\).

\medskip
Пусть \(L = L_{p1}\).  
Тогда \(n=1\), 
\(\ulcorner\beta(t_{1})\urcorner=\langle a,b\rangle\), \(s=a\), 
\(t_{1}=\langle a',b'\rangle,\)
где \(a,b\) — нормализованные сообщения, 
\(a',b'\) — открытые сообщения, 
\(\beta(a')=a\), \(\beta(b')=b\).  
Положим
\(
  t=a',\quad
  \mathcal E=\varnothing,
\)
расширения \(\beta\) не требуются.  
Ясно, что условия (1)–(5) выполнены.  
Аналогичные рассуждения работают для \(L_{p2}, L_{ad}\) и \(L_{c}\).

\medskip
Теперь положим, что\(L=L_{sd}\).  
Тогда \(n=2\),
\(\ulcorner\beta(t_{1})\urcorner=\{a\}_{s}^{b}\), \(\ulcorner\beta(t_{2})\urcorner=b\) для нормализованных
\(a,b\).  Значит,
\(
  t_{1}=\{a'\}_{s}^{b},
  t_{2}=b'',
\)
где \(a',b',b''\) — открытые сообщения с
\(\beta(a')=a\), \(\beta(b')=\beta(b'')=b\).  
Положим
\(
  t=a',
  \mathcal E=\mathcal E^\beta_{b,b},
\)
расширения \(\beta\) не вносятся.  
По лемме 5.5 проверяется, что условия (1)–(5) выполняются.

\medskip
Наконец, положим, что \(L=L_{o}\).  
Тогда
\(
  s = \ulcorner Exp\!\bigl(\beta(t_{1}),\,
         \beta(t_{2})^{a_{2}}\cdots\beta(t_{n})^{a_{n}}\bigr)\urcorner,
  a_{i}\in\mathbb Z.
\)
Определим новые переменные \(z_{2},\dots,z_{n}\) и терм
\(
  t \;=\;
   Exp\!\bigl(t_{1},\,t_{2}^{z_{2}}\cdots t_{n}^{z_{n}}\bigr)^{\beta},
  \mathcal E = \mathcal E_{t}^{\beta},
\)
расширив \(\beta\) так, что \(\beta(z_{i})=a_{i}\) для всех \(i\).  
По лемме 5.10 имеем
\(
  s
    = \ulcorner\beta\bigl( Exp(t_{1},t_{2}^{z_{2}}\cdots t_{n}^{z_{n}})\bigr)\urcorner
    = \beta(t),
  \beta\models\mathcal E,
\)
а для любого \(\beta'\models\mathcal E\)
\(
  \ulcorner\beta'(t)\urcorner
    = \ulcorner\beta'\Bigl( Exp(t_{1},t_{2}^{z_{2}}\cdots t_{n}^{z_{n}})\Bigr)\urcorner
    = \ulcorner Exp\!\bigl(\beta'(t_{1}),\,\beta'(t_{2})^{\beta(z_{2})}\cdots\beta'(t_{n})^{\beta(z_{n})}\bigr)\urcorner.
\)
Утверждение о полиномиальном ограничении показателей даёт условие (4),
а из леммы 5.18 вытекает условие (5).
\end{proof}
