\section{Расширение модели нарушителя Долева–Яо за счёт экспоненцирования Диффи–Хеллмана}

Мы расширяем нарушителя Долева–Яо, определённого правилами
\(L_c \cup L_d\) (см.~п.~2.2), добавляя набор правил \(L_\sigma\) —
\emph{DH-правила}, позволяющих нарушителю выполнять
экспоненцирование Диффи–Хеллмана.
Расширенный нарушитель будем называть \emph{DH-нарушителем}.
Наша цель — показать, что для DH-нарушителя задача вывода решается за детерминированное полиномиальное
время, а задача небезопасности решается за недетерминированное полиномиальное время.
Для этого проверим, что выполняются предпосылки
Теоремы 3.2 и Теоремы 3.1.
Напомним, Теорема 3.2 требует:
\begin{enumerate}
  \item \(L_\sigma\) является множеством оракульных правил и
  \item \textsc{OracleRule} решается за полиномиальное время;
\end{enumerate}
дополнительно Теорема 3.1 требует:
\begin{enumerate}
  \item \(L_\sigma\) допускает полиномиальные «произведение-показатель»-атаки.
\end{enumerate}

В разделах 4.1 и 4.2 мы докажем пункт а),
а в разделе 4.3 — пункт б).
Используя Теорему 3.2, придём к выводу,
что для DH-нарушителя задача вывода
решается за полиномиальное время
(см.~Следствие 4.9).

В разделе 5 мы докажем пункт iii);
совместно с Теоремой 3.1 это даст,
что для DH-нарушителя задача небезопасности
решается за недетерминированное полиномиальное время
(см.~Теорему 5.23).

\medskip
Сначала зададим DH-правила \(L_\sigma\).

\begin{definition}[4.1]
Положим \(L_\sigma = L_{\sigma_c}\cup L_{\sigma_d}\), где
\(L_\sigma\) состоит из всех правил вида
\[
  t,\,t_1,\dots,t_n
  \;\longrightarrow\;
  \ulcorner Exp\!\bigl(t,t_1^{\,z_1}\!\cdots t_n^{\,z_n}\bigr)\urcorner
  \;=:\;u,
\]
где \(n\ge1\), \(z_i\in\mathbb Z\setminus\{0\}\),
\(1\le i\le n\), а \(t,t_1,\dots,t_n\) — нормализованные
стандартные сообщения.  
Если \(u\) имеет вид \(Exp(\cdot,\cdot)\),
то правило входит в
\(L_{\sigma_c}(u)\) (множество \emph{композиционных} DH-правил);
иначе — в \(L_{oc}(u)\) (множество
\emph{декомпозиционных} DH-правил).
Нарушителя, использующего \(L_0\) как оракульные правила,
мы называем \emph{DH-нарушителем}.  
Термин \(t\) в правиле называют \emph{головой} правила, а
\(z_1,\dots,z_n\) — \emph{показателями произведения}.
Будем считать без ограничения общности, что
голова декомпозиционного DH-правила имеет форму \(Exp(\cdot,\cdot)\)
(иначе \(t=u\));
также полагаем \(t_i\neq t_j\) при \(i\neq j\) и \(z_i\neq0\) для всех \(i\).
\end{definition}

\medskip
Так как сообщения в правой части DH-правила нормализованы,
легко получить следующее.

\textsc{Лемма 4.2.}
Правила из \(L_{oc}\) являются композиционными,
а правила из \(L_{od}\) — декомпозиционными правилами угадывания.

\subsection{Диффи–Хеллман правила допускают корректные выводы}

Покажем, что \(L_o\) допускает корректные выводы (лемма 4.5).  
Иными словами, \(L_o\) удовлетворяет первому из требований к «oracle-rules»
(см. определение 2.11).  
Доказательство опирается на две вспомогательные леммы, приведённые ниже.

Первая лемма позволяет ограничиться такими выводами, в которых
правила из \(L_o\) применяются только к тем сообщениям,  
которые были либо созданы правилами Долева–Яо\,(DH),  
либо присутствовали в исходном множестве сообщений.

\textsc{Лемма 4.3.}
Пусть \(E\) — конечное множество нормализованных стандартных сообщений,
\(t\) — стандартное сообщение такое, что \(t\) выводится из \(E\) (относительно \(L\)).
Пусть \(D\) — вывод из \(E\) с целью \(t\).
Тогда существует вывод \(D'\) из \(E\) с той же целью \(t\), удовлетворяющий:

\begin{enumerate}
  \item \(D'\) имеет ту же длину, что и \(D\);
  \item для любого DH-правила \(L\in D'\cap L_{o}\) с головным термом \(t'\)
        выполнено: \(t'\in E\) \;или\;  
        существует \(t'\)-правило \(L'\in D'\cap(L_{d}\cup L_{c})\).
        Более того, если \(L\) — \emph{декомпозиционное} DH-правило,
        то \(t'\in E\) \;или\;
        существует \(t'\)-правило \(L'\in D'\cap L_{d}\).
\end{enumerate}

\begin{proof} См. Приложение Б.\end{proof}

Следующая лемма даёт критерий, позволяющий проверить,
корректен ли вывод.

\textsc{Лемма 4.4.}
Пусть
\(D = E_0 \rightarrow_{\,L_1\,} \dots
      E_{n-1} \rightarrow_{\,L_n\,} E_n\)
— вывод с целью \(g\).

\begin{enumerate}
\item
  Предположим, что для каждого шага
  \(E_{j-1} \rightarrow{\,L_j\,} E_j\) вывода \(D\)
  с \(L_j \in \mathcal L_{d}(t)\)
  существует \(t'\in E_{j-1}\) такое, что
  \(t \sqsubseteq t'\) и
  \(t'\in E_0\) \;или\;
  \(\exists\,i<j: L_i\in \mathcal L_{d}(t')\).
  Тогда из \(L\in D\cap \mathcal L_{d}(t)\)
  (для некоторого \(\mathcal L,t\)) следует
  \(t\in \mathcal S(E_0)\).

\item
  Предположим, что для каждого \(i<n\) и \(t\) с
  \(L_i\in \mathcal L_{c}(t)\)
  найдётся \(j>i\) такое, что
  \(L_j\) является \(t'\)-правилом
  и \(t\in \mathcal S\bigl(\{t'\}\cup E_0\bigr)\).
  Тогда из \(L\in D\cap \mathcal L_{c}(t)\)
  (для некоторого \(L,t\)) следует
  \(t\in \mathcal S(E_0,g)\).
\end{enumerate}

При выполнении обеих предпосылок (а) и (б)
вывод \(D\) является корректным 
выводом с целью \(g\).

\begin{proof} См.\ Приложение Б.\end{proof}

Теперь мы можем показать, что правила \(L_{o}\) допускают
корректные выводы.

\textsc{Лемма 4.5.}
Пусть \(E\) — конечное нормализованное множество стандартных сообщений,
а \(g\) — нормализованное стандартное сообщение.
Если \(g\in forge(E)\),
то существует корректный вывод из \(E\) с целью \(g\).

\begin{proof}
Положим \(E_{0}=E\) и
\(
  D\;=\;
  E_{0}\rightarrow_{\,L_{1}\,}
  \dots
  \rightarrow_{\,L_{n}\,}E_{n}
\)
— вывод цели \(g\) минимальной длины.
Будем считать, что \(D\) удовлетворяет свойствам,
сформулированным в лемме 4.3, пункт 2.

Покажем, что \(D\) удовлетворяет предпосылкам леммы 4.4, пунктов 1 и 2.

(1)\quad
Пусть \(L_{j}\in L_{d}(s)\cap \mathcal L_{d}(t)\); тогда \(t\in S(s)\).
Для всех \(i<j\) имеем \(L_{i}\notin L_{oc}(s)\),
поскольку правила из \(L_{oc}\) не создают стандартных терминов,
и \(L_{i}\notin L_{oc}(s)\) по определению вывода  
(иначе \(t\) находился бы в левой части правила \(L_{i}\)).
Следовательно, либо \(s\in E_{0}\),
либо существует \(i<j\) такое, что \(L_{i}\in \mathcal L_{d}(s)\).
Если \(L_{j}\in L_{od}(t)\) и \(t'\) — головной терм правила \(L_{j}\),
то, по определению декомпозиционных DH-правил,
легко видеть, что \(t\in \mathcal S(t')\).
По лемме 4.3, 2 отсюда следует, что
\(t'\in E_{0}\) \;или\;
существует \(t'\)-правило \(L'\in D\cap L_{d}(t')\).
Следовательно, по лемме 4.4, 1 имеем:
если \(L\in D\cap \mathcal L_{d}(t)\) для некоторого \(L\) и \(t\),
то \(t\in \mathcal S(E_{0})\).

(2)\quad
Пусть \(L_i\in \mathcal L_{c}(t)\) и \(i<n\).
По минимальности вывода \(D\) найдётся \(j>i\) такое,
что \(t\) входит в левую часть правила \(L_j\).
Если \(L_j\in L_{d}\), то, как и в пункте 1,
получаем \(t\in \mathcal S(E_0)\).
Если \(L_j\in L_{c}(t')\), то \(t\in \mathcal S(t')\).
Пусть теперь \(L_j\in L_{o}(t')\).
Сначала предположим, что \(t\) является головным термом \(L_j\).
По лемме 4.3, 2 существует \(t\)-правило
\(L'\in D\cap(L_{d}\cup L_{c})\).
Так как \(L_i\in L_{c}(t)\) и, благодаря минимальности \(D\),
термин \(t\) может порождаться ровно одним правилом,
имеем \(L_i=L'\in L_{c}\); следовательно \(t\neq Exp(\,\cdot,\cdot\,)\).
Из определения DH-правил тогда следует \(t\in \mathcal S(t')\).
Пусть теперь \(t\) \emph{не} является головным термом \(L_j\).
Если \(t\notin \mathcal S(t')\), то существует термин \(t''\) —
головной терм \(L_j\) — такой, что \(t\in \mathcal S(t'')\),
и \(t''\) имеет вид \(Exp (\,\cdot,\cdot\,)\)
(иначе \(t\) не мог бы исчезнуть из \(t'\)).
По лемме 4.3, 2 либо \(t''\in E_0\),
либо существует \(t''\)-правило \(L'\in D\cap(L_{d}\cup L_{c})\).
Так как \(t''\) имеет вид \(Exp(\,\cdot,\cdot\,)\),
получаем \(L'\in D\cap L_{d}\).
Из пункта 1 тогда следует \(t''\in \mathcal S(E_0)\),
а значит \(t\in \mathcal S(E_0)\).
\end{proof}

\subsection{Правила Диффи–Хеллмана являются правилами оракула}

Теперь мы докажем оставшиеся свойства, необходимые для oracle rules, и тем самым покажем, что $L_{o}$ действительно образует множество oracle rules (см.~Предложение 4.7).  
Сначала нам понадобится лемма, аналогичная Лемме 3.6.

\textsc{Лемма 4.6.}
Пусть $z_{1},\dots ,z_{n}\in\mathbb Z\setminus\{0\}$,
а $s,s_{1},\dots ,s_{n}$ — нормализованные стандартные термины,
удовлетворяющие условиям
$s_{i}\neq s_{j}$ при $i\neq j$, 
$s_{i}\neq1$ и $s_{i}\neq u$ для всех $i$,
$s\neq u$,
\(
  u=\ulcorner Exp\bigl(s,\,
       s_{1}^{\,z_{1}}\!\cdots s_{n}^{\,z_{n}}\bigr)\urcorner,
  u=Exp(\,\cdot,\cdot\,).
\)
Пусть $\delta$ — замена $[\,u\!\rightarrow\!2\,]$.
Тогда
\(
  u
  =
  \ulcorner Exp\bigl(\ulcorner s\delta\urcorner,\,
        \ulcorner s_{1}\delta\urcorner^{z_{1}}\!\cdots
        \ulcorner s_{n}\delta\urcorner^{z_{n}}\bigr)\urcorner.
\)

\begin{proof} См.\ Приложение Б.\end{proof}

\medskip
Теперь мы готовы сформулировать и доказать следующую теорему.

\textsc{Предложение 4.7.}
Множество правил $L_{o}$ является множеством \emph{правил оракула}.

\begin{proof}
Проверим по очереди условия 1, 2 и 3 определения 2.11.

(1) Непосредственное следствие леммы 4.5.

(2) Утверждение вытекает из того, что ни один термин,
созданный правилом из $L_{oc}$, не может быть
декомпозирован с помощью правила из $L_{d}$.

(3) Пусть $u$ — нормализованное стандартное сообщение,
$F$ — множество стандартных сообщений, причём $1\in F$,
и $t$ — стандартное сообщение такое, что
\(
  F\!\cup\!\{u\}\;\rightarrow_{\,\mathcal L_{c}(u)\,}\;F,
  F\;\rightarrow_{\,L_{o}(t)\,}\;F,t.
\)
Положим $\delta:=[\,u\!\leftarrow\!1\,]$.
Если $u=t$, то $t\delta=1\in forge(F\delta)$,
и требуемое выполнено.
Предположим $u\neq t$.
Из $F\rightarrow_{\,L_{o}(t)\,}F$ следует, что существуют
$t',t_{1},\dots ,t_{n}\in F$ и
$z_{1},\dots ,z_{n}\in\mathbb Z\setminus\{0\}$,
такие что $t_{i}\neq t_{j}$ при $i\neq j$ и
\(
  t=\ulcorner Exp\bigl(t',\,t_{1}^{\,z_{1}}\!\cdots t_{n}^{\,z_{n}}\bigr)\urcorner.
\)
Если $t'\neq Exp(\,\cdot,\cdot\,)$
или $u\neq t'$, то по лемме 3.6
\[
  \ulcorner t\delta \urcorner
  =
  \ulcorner Exp\bigl(\ulcorner t'\delta\urcorner,\,
        \ulcorner t_{1}^{\delta}\urcorner^{z_{1}}\!\cdots
        \ulcorner t_{n}^{\delta}\urcorner^{z_{n}}\bigr)\urcorner.
\]

Таким образом, $t^{\delta}\in forge(F^{\delta})$.
Предположим теперь, что $u=t'=Exp(v,M)$.
Тогда
\begin{align*}
\ulcorner t^{\delta} \urcorner
  &= \ulcorner\ulcorner Exp\!\bigl(v,M^{z_{1}}_{\,1}\cdots t_{n}^{z_{n}}\bigr)\urcorner\delta\urcorner\\
  &= \ulcorner Exp\!\bigl(v,M^{\delta}\,t_{1}^{\delta z_{1}}\cdots t_{n}^{\delta z_{n}}\bigr)\urcorner \tag{*}\\
  &= \ulcorner Exp\!\bigl(v\delta,M^{\delta}(t_{1}\delta)^{z_{1}}\cdots(t_{n}\delta)^{z_{n}}\bigr)\urcorner \tag{**}\\
  &= \ulcorner Exp\!\bigl(v,M\,\ulcorner t_{1}\delta\urcorner^{z_{1}}\cdots\ulcorner t_{n}\delta\urcorner^{z_{n}}\bigr)\urcorner \tag{***}\\
  &= \ulcorner Exp\!\bigl(u,\ulcorner t_{1}\delta\urcorner^{z_{1}}\cdots\ulcorner t_{n}\delta\urcorner^{z_{n}}\bigr)\urcorner.
\end{align*}

В переходе $(*)$ применяем лемму 3.4(2), используя, что $v\ne u$ и $u\ne t$.
В $(**)$ снова используем условие $u\ne t$,
а $(***)$ получаем, поскольку $u\notin S(v,M)$.
Чтобы показать, что $\ulcorner t\delta \urcorner \in forge(\ulcorner F\delta\urcorner)$,
достаточно убедиться, что $u\in forge(\ulcorner F\delta\urcorner)$.
Из $F\setminus\{u\}\rightarrow_{\,\mathcal L_{c}(u)\,}F$
и $u=Exp(\cdot,\cdot)$
получаем $F\!\setminus\!u\rightarrow_{\,L_{oc}(u)\,}F$.
Следовательно, существуют нормализованные термины
$s,s_{1},\dots,s_{n}\in F\setminus\{u\}$
и целые $z'_{1},\dots,z'_{n}\in\mathbb Z\setminus\{0\}$,
такие что $s$ и $s_{i}$ удовлетворяют условиям леммы 4.6
и
\(
  u=\ulcorner Exp\!\bigl(s,s_{1}^{\,z_{1}}\!\cdots s_{n}^{\,z_{n}}\bigr)\urcorner .
\)
Тогда по лемме 4.6
\(
  u=\ulcorner Exp\!\bigl(\ulcorner s\delta \urcorner,
        \ulcorner s_{1}^{\delta}\urcorner^{z_{1}}\!\cdots\ulcorner s_{n}^{\delta}\urcorner^{z_{n}}\bigr)\urcorner,
\)
и, следовательно,
$u\in forge(F\delta)$.
\end{proof}

\subsection{Принятие решения для правил DH}

Следующее предложение показывает, что за полиномиальное время
можно решить, выводится ли заданное сообщение из конечного множества
сообщений при \emph{одном} применении правила оракула.

\textsc{Предложение 4.8.}
Для нарушителя DH задача \textsc{OracleRule}
разрешима в детерминированное полиномиальное время.

\begin{proof}
Нужно построить детерминированный алгоритм полиномиального времени,
который по данным $E$ и $t$ решает, существуют ли
$t',t_{1},\dots,t_{n}\in E$ и $z_{1},\dots,z_{n}\in\mathbb Z$
такие, что
\(
  t=\ulcorner Exp\!\bigl(t',
     t_{1}^{z_{1}}\!\cdots t_{n}^{z_{n}}\bigr)\urcorner .
\)
Нетрудно убедиться, что
\(
  E \rightarrow_{\,L_{o}\,} E,t
\)
точно тогда, когда выполняется одно из условий:

\begin{enumerate}
\item
  $t\neq Exp(\,\cdot,\cdot\,)$ и
  \begin{enumerate}
    \item $t\in E$, \;или
    \item существует $M$ такое, что
          $Exp(t,M)\in E$
          и $\mathcal F(M)\subseteq E$;
  \end{enumerate}

\item
  $t=Exp(v,M)$ и
  \begin{enumerate}
    \item $v\in E$ и $\mathcal F(M)\subseteq E$, \;или
    \item существует $M'$ такое, что
          $Exp(v,M')\in E$
          и
          \(
            E':=\bigl\{\,t'\,\bigl|\,
              \text{мультипликативные показатели $t'$ в $M$ и $M'$\\
              различаются}
            \bigr\}\subseteq E.
          \)
  \end{enumerate}
\end{enumerate}

Исходя из этой характеристики, легко вывести
полиномиальный алгоритм для решения задачи
$E \rightarrow_{\,L_{o}\,} E,t$.
\end{proof}

\medskip
Немедленным следствием предложения 4.8,
а также предложения 4.7 и теоремы 3.2 является:

\textsc{Следствие 4.9.}
Для нарушителя DH задача \textsc{Derive}
решается в детерминированное полиномиальное время.
