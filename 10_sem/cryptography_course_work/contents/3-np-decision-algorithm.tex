\section{NP-алоритм принятия решения}

Теперь сформулируем одну из основных теорем этой работы, утверждающую, что задача \textsc{INSECURE} разрешима в классе \textsf{NP} для любого множества оракульных правил, удовлетворяющего двум условиям. Данная общая теорема будет применена в разделе 4 и разделе 7, чтобы показать, что \textsc{INSECURE} остаётся в \textsf{NP} в присутствии злоумышленника, способного выполнять экспоненциацию Диффи–Хеллмана и использовать коммутативное шифрование с открытым ключом, соответственно.

В теореме предъявляются два требования к множеству оракульных правил. Первое: проблема проверки применимости оракульного правила должна решаться эффективно. Второе: множество правил должно допускать полиномиальные атаки на показатели произведений.

\medskip
\textsc{Определение 3.1.} Задача \emph{oracle rule problem} задаётся как
\[
\mathsf{OracleRule}
\;=\;
\{(E,m)\mid E \;\to_{L_o}\; E,\,m\},
\]
где \(E\) — конечное нормализованное множество стандартных сообщений, а \(m\) — стандартное сообщение; оба подаются в виде DAG.

Говорят, что множество оракульных правил \(\mathcal{L}_o\) \emph{допускает полиномиальные атаки на показатели произведений}, если для любого протокола \(P\) и любой минимальной атаки \((\pi,\sigma)\) на этот протокол существует подстановка \(\sigma'\) такая, что $\sigma'\approx\sigma$
(напомним, это означает совпадение \(\sigma'\) и \(\sigma\) с точностью до показателей произведений), пара \((\pi,\ulcorner\sigma'\urcorner)\) является атакой на \(P\), и  $\|\sigma'\|_{\mathrm{exp}}$
полиномиально ограничена по \(\|P\|\). Заметим, что по лемме 2.4 из этого также следует полиномиальная ограниченность \(\|\ulcorner\sigma'\urcorner\|\) по \(\|P\|\).

\medskip
\textsc{Теорема 3.1.} Пусть \(L_o\) — множество оракульных правил. Если
\begin{itemize}
  \item $\mathsf{OracleRule}\in\mathsf{P}\textsf{TIME}$ и
  \item $L_o\text{ допускает полиномиальные атаки на показатели произведений},$
\end{itemize}
то задача \textsc{INSECURE} лежит в классе \(\mathsf{NP}\).

Теорема доказывается следующим образом. Сначала показывается, что недетерминированный алгоритм из рисунка 1 является недетерминированным полиномиальным алгоритмом и корректно решает задачу \textsc{INSECURE}. Полином \(p\), ограничивающий размер показателей произведений, существует в силу того, что \(L_o\) допускает полиномиальные атаки на показатели произведений. Структура алгоритма такова: на шагах 1 и 2 недетерминированно выбирается «маленькая» атака \((\pi,\sigma)\) на протокол \(P\), а на шагах 3 и 4 проверяется, действительно ли \((\pi,\sigma)\) является атакой на \(P\).

Очевидно, что алгоритм из рисунка 1 корректен. Основная же сложность состоит в том, чтобы доказать полиномиальность его работы. Это делается с помощью результатов из разделов 3.1, 3.2 и 3.3. В дальнейшем мы сначала изложим эти результаты, а затем на их основе докажем полноту алгоритма и оценим его временную сложность.

В разделе 3.1 (см. теорему 3.2) показывается, что следующая задача, далее называемая \emph{задачей вывода (derivation problem)}, решается за полиномиальное время от \(\lVert E,t\rVert\), при условии, что \(\mathsf{OracleRule}\) разрешима в детерминированном полиномиальном времени:
\[
\mathsf{Derive} \;:=\; \bigl\{(E,t)\mid t\in\mathrm{forge}(E)\bigr\},
\]
где \(E\) — конечное множество стандартных сообщений, а \(t\) — стандартное сообщение, заданные в виде DAG.

В разделе 3.2 (см. утверждение 3.14) доказывается, что подстановки минимальных атак строятся из подтермов термов, встречающихся в описании протокола.

На основе предложения 3.14 в разделе 3.3 оценивается число подтермов в подстановках минимальных атак и выводится неравенство \(\lvert\sigma\rvert \le \lvert P\rvert\) (следствие 3.16).

Учитывая эти результаты, можем записать доказательство теоремы 3.1.

\medskip
\noindent\textbf{Доказательство теоремы 3.1.}  
Покажем, что алгоритм из рисунка 1 работает в недетерминированное полиномиальное время и является корректным и полным. Корректность уже была отмечена.

Для доказательства полноты нужно убедиться, что если существует атака \((\pi,\sigma)\) на протокол \(P\), то найдётся атака с подстановкой \(\sigma'\), размер которой ограничен так, как это требуется в шаге 2 алгоритма из рисунка 1. Это немедленно следует из следствия 3.16 и предположения, что \(L_o\) допускает полиномиальные атаки на показатели произведений.

Остаётся показать, что наш алгоритм выполняется в недетерминированное полиномиальное время по величине \(\lVert P\rVert\). Для шагов 1 и 2 это очевидно. Чтобы убедиться в полиномиальности шагов 3 и 4, воспользуемся теоремой 3.2. Пусть $E \;=\;\{\ulcorner S_j\sigma\urcorner \mid j<i\}\;\cup\;\{\ulcorner S_0\urcorner\}$
для некоторого \(i\in\{1,\dots,k\}\), а \(t\) обозначает либо \(\ulcorner S_0\urcorner\), либо \(\mathit{secret}\). По теореме 3.2 проверка $t \;\in\;\mathit{forge}(E)$
может быть выполнена за детерминированное полиномиальное время от \(\lVert E,t\rVert\). По следствию 3.16 имеем
$|E,t| \;\le\;|P|$. Поскольку \(\|\sigma\|_{\mathrm{exp}}\) полиномиально ограничено по \(\lVert P\rVert\), то то же верно и для
$\Bigl\|\{\ulcorner S_j\sigma\urcorner\mid j<i\}\cup\{\ulcorner S_0\urcorner\}\cup\{R_i\sigma\}\Bigr\|_{\mathrm{exp}}$.
По лемме 2.4 из этого следует, что \(\lVert E,t\rVert_{\mathrm{exp}}\) полиномиально ограничено по \(\lVert P\rVert\). Следовательно, шаги 3 и 4 могут быть выполнены за детерминированное полиномиальное время по \(\lVert P\rVert\).  
\(\Box\)

\subsection{Решение задачи вывода}

Покажем, что задачу вывода (derivation problem) можно решить за полиномиальное время, при условии, что задача \\textsf{OracleRule} разрешима в детерминированном полиномиальном времени.

\medskip
\noindent\textbf{Теорема 3.2.} Задача
\[
\mathsf{Derive} \;:=\; \{(E,t)\mid t\in\mathrm{forge}(E)\}
\]
разрешима в классе \textsf{P}\textsf{TIME}, при условии, что \(\mathsf{OracleRule}\in\textsf{P}\textsf{TIME}\).

\begin{proof}
Пусть $d_{t}(E)=\{\,t'\in S(E,t)\mid E\;\to_{L_o}\;E,t'\}$
— множество сообщений $t'$, выводимых из $E$ за один шаг. Поскольку число таких $t'$ линейно по $\|E,t\|$, а проверка шага 
$E\;\to_{L_o}\;E,t'$
выполняется за детерминированное полиномиальное время от $\|E,t\|$, нетрудно убедиться, что $d_{t}(E)$ вычисляется за полиномиальное время от $\|E,t\|$. 
Теперь предположим, что $t\in\mathrm{forge}(E)$. По определению 2.11 существует корректная деривация
$D:\quad E \;\to_{L_{1}}\;E, t_{1} \;\to\;\dots\; \to_{L_{r}}\;E, t_1, \dots, t_r$,
где $t_{r}=t$. В частности, $t_{i}\in S(E,t)$ при $i=1,\dots,r$, и все $t_{i}$ попарно различны, поэтому $r\le|E,t|$. Введём рекурсию
$d_{t}^{0}(E):=E, \qquad d_{t}^{i+1}(E):=d_{t}\bigl(d_{t}^{i}(E)\bigr)$.
Очевидно, что 
$t\in d_{t}^{r}(E) \quad\Longleftrightarrow\quad t\in\mathrm{forge}(E)$.
Поскольку каждый шаг вычисления $d_{t}^{i}(E)$ занимает полиномиальное время от $\|E,t\|$, вытекает, что проверка $t\in\mathrm{forge}(E)$ также выполняется за детерминированное полиномиальное время.  
\end{proof}

\subsection{Характеризация подтермов минимальных атак}

Далее мы покажем, что подстановки минимальных атак могут быть сконструированы путём «связывания» подтермов, которые изначально встречаются в спецификации задачи (см. Утверждение 3.14). Для доказательства этого утверждения сначала установим некоторые свойства замен (Леммы 3.4–3.6), затем свойства подстановок в минимальных атаках (Леммы 3.7–3.9) и, наконец, свойства дериваций (Леммы 3.10–3.13).

В дальнейшем мы всегда предполагаем, что \(L_{o}\) — множество оракульных правил. Если \(t\in\mathrm{forge}(E)\), то обозначим через \(D_{t}(E)\) некоторую корректную деривацию из \(E\) с целью \(t\) (выбранную произвольно). Такая деривация всегда существует в силу определения оракульных правил.

Пусть $P=\bigl(\{\,R_i\Rightarrow S_i\mid i\in\mathcal{I}\},\;\prec_{\mathcal{I}},\;S_0\bigr)$ — протокол, а \((\pi,\sigma)\) — атака на \(P\). Пусть \(k=|\pi|\), а \(R_i\) и \(S_i\) определены согласно Определению 2.8. Напомним, что
$S(P)=S_0\;\cup\;\bigcup_{i\in\mathcal{I}}(R_i\cup S_i), \qquad \mathcal{V}(P)=\mathcal{V}\bigl(S(P)\bigr)$.
Также нам понадобится следующее ключевое понятие.

\textsc{Определение 3.3.} Пусть \(t\) и \(t'\) — два терма, а \(\theta\) — наземная подстановка. Тогда говорят, что \(t\) является \(\theta\)\nobreakdash‑соответствием (или \(\theta\)\nobreakdash‑match) терма \(t'\), что обозначается
\[
t \;\sqsubseteq_{\theta}\; t',
\]
если \(t\) и \(t'\) — стандартные термы и
\[
\ulcorner t\,\theta \urcorner\;=\; t'.
\]

\subsubsection*{3.2.1 Свойства замен}

Следующие леммы устанавливают дистрибутивные свойства функции нормализации, подстановок, оператора экспоненциации и операций замены.

\textsc{Лемма 3.4.} Пусть \(u\) — нормализованный терм, а \(M, M'\) — два произведения, такие что все подтермы из \(\mathcal{F}(M)\cup \mathcal{F}(M')\) нормализованы. Пусть \(s\) — стандартный нормализованный терм, а \(\delta=[s\leftarrow 1]\) — замена. Тогда выполняются равенства:

(1)\quad 
\(\ulcorner (M\cdot M') \delta \urcorner
   =\ulcorner\ulcorner M\cdot M'\urcorner\delta\urcorner\),
в частности 
\(\ulcorner M\delta\urcorner=\ulcorner\ulcorner M\urcorner{\delta}\urcorner\).

(2)\quad 
\(\ulcorner Exp(u,M)\delta\urcorner
   =\ulcorner\ulcorner Exp(u,M)\urcorner\delta\urcorner\),
если \(s\neq\ulcorner Exp(u,M)\urcorner\),
и, в случае если \(s=Exp(\cdot,\cdot)\), также \(s\neq u\).

\begin{proof}
См.~Приложение~A\qedhere
\end{proof}

\medskip
Отметим, что пункт 2 в предыдущей лемме не выполняется без ограничений на $s$.
Следующий пример демонстрирует проблему при
$s=\ulcorner Exp(u,M)\urcorner$:
предположим, что $s=Exp(a,b)$, $u=a$, а
$M=b\!\cdot\!c\!\cdot\!c^{-1}$.
Тогда
$s=\ulcorner Exp(u,M)\delta\urcorner
\neq
\ulcorner\ulcorner Exp(u,M)\urcorner\delta\urcorner=1$.  
Следующий пример иллюстрирует, почему необходимо требование $s\neq u$:
положим $s=u=Exp(a,b)$ и $M=c$.
Тогда
$1=\ulcorner Exp(u,M)\delta\urcorner
\neq
\ulcorner\ulcorner Exp(u,M')\urcorner\delta\urcorner
=Exp(a,b\!\cdot\!c)$.

\textsc{Лемма 3.5.} 
Пусть $\sigma$ — нормализованная замкнутая подстановка,
$E$ — множество нормализованных терминов,
$s$ — нормализованный стандартный неатомарный термин,
а $\delta$ — замена $[s \leftarrow 1]$.
Обозначим $\sigma'=\ulcorner\sigma\delta\urcorner$.
Если не существует стандартного подтермина $t\in E$ такого, что  
$t\sqsubseteq_{\sigma}s$, то $\ulcorner E\sigma'\urcorner = \ulcorner\ulcorner E\sigma\urcorner\delta\urcorner$.

\begin{proof}
  См.\ Приложение~A.
\end{proof}

\medskip

\textsc{Лемма 3.6.} 
Пусть $t',\,t_1,\dots,t_n,\,t,\,u$ — нормализованные стандартные термины,
$z_1,\dots,z_n\in\mathbb Z$, а $\delta$ — замена $[u\leftarrow1]$ такая, что
$u\neq t$ и $t=\ulcorner Exp\bigl(t',\,t_1^{\,z_1}\!\cdots t_n^{\,z_n}\bigr)\urcorner$.
Если $t'=Exp(\,\cdot,\cdot\,)$, то дополнительно предполагаем
$u\neq t'$. Тогда
\[
  \ulcorner t\delta\urcorner
  =\ulcorner Exp\bigl(\ulcorner t'\delta\urcorner,
        \ulcorner t'\delta\urcorner^{z_1}\!\cdots \ulcorner t_n\delta\urcorner^{z_n}\bigr)\urcorner.
\]

\begin{proof}
  См.\ Приложение А.
\end{proof}

\subsubsection*{3.2.2 \,Свойства подстановок минимальных атак}
Следующая лемма позволяет доказать то, что мы далее будем
называть \emph{свойством единственного сопоставления}.

\textsc{Лемма 3.7.}
Пусть $s$ — стандартный термин, $t$ — нормализованный термин,
а $\sigma$ — нормализованная подстановка такая, что
$s\in S(\ulcorner t\sigma\urcorner)$ и
$s\notin S(x\sigma)$ для каждого $x\in\mathcal V(t)$.
Тогда существует стандартный подтермин $t'$ термина $t$,
удовлетворяющий $t'\sqsubseteq_{\sigma}s$.

\textit{Доказательство.}
По лемме 2.4 имеем
\[
  S(\ulcorner t\sigma\urcorner)\;\subseteq\;
  \ulcorner S(t\sigma)\urcorner
  \;\subseteq\;
  \ulcorner S(t)\sigma\;\cup\;\mathcal V(t)\sigma\urcorner.
\]
Так как $\sigma$ находится в нормальной форме, это означает
\[
  S(\ulcorner t\sigma\urcorner)\;\subseteq\;
  \ulcorner S(t)\sigma\urcorner\;\cup\;\mathcal V(t)\sigma.
\]
По предположению $s\in \mathcal S(\ulcorner t\sigma\urcorner)$ и
$s\notin \mathcal S(\mathcal V(t)\sigma)$.
Следовательно, $s\in\ulcorner \mathcal S(t)\sigma\urcorner$,
то есть существует $t'\in \mathcal S(t)$ такое, что
$\ulcorner t'\sigma\urcorner = s$.
\hfill$\square$

Теперь мы докажем \emph{свойство единственного сопоставления}:
если злоумышленник посылает сообщения $m_1,\dots,m_i$,
то существует не более одного способа сопоставить эти сообщения
с $R_1,\dots,R_i$ при каждом $R_j,\;j\in\{1,\dots,i\}$,
определённых выше.

\textsc{Лемма 3.8.} \textit{Пусть дана произвольная последовательность
нормализованных сообщений $m_1,\dots,m_i$.
Существует не более одной нормализованной подстановки $\sigma$ такой, что
\(
  \ulcorner R_j\sigma\urcorner = m_j
\)
для всех $j\in\{1,\dots,i\}$.}

\textit{Доказательство.}
Предположим противное и выберем минимальный
$i\in\{1,\dots,n\}$, для которого существуют две разные
нормализованные подстановки $\sigma$ и $\sigma'$, удовлетворяющие
\(
  \ulcorner R_j\sigma\urcorner=\ulcorner R_j\sigma'\urcorner
\)
для всех $j\in\{1,\dots,i\}$.
Из минимальности $i$ следует, что $\sigma$ и $\sigma'$
совпадают на
\(V_{i-1}=\mathcal V(R_1,\dots,R_{i-1})\)
и различаются на некоторой переменной из
\(\mathcal V(R_i)\setminus V_{i-1}\).
Обозначим через $\sigma_0$ подстановку, равную
$\sigma$ на $V_{i-1}$ и тождественную
на \(\mathcal V(R_i)\setminus V_{i-1}\);
положим \(r=\ulcorner R_i\sigma_0\urcorner\).

По Лемме 2.4, 5 и поскольку
\(R_i\sigma=(R_i\sigma_0)\sigma\) и
\(R_i\sigma'=(R_i\sigma_0)\sigma'\),
получаем
\(
  \ulcorner R_i\sigma\urcorner
  =\ulcorner r\sigma\urcorner
  =\ulcorner r\sigma'\urcorner.
\)
По предположению найдётся переменная
\(x\in\mathcal V(r)\) такая, что \(x\sigma\ne x\sigma'\).
Выберем \(t_x\in \mathcal S(r)\) минимальный
(относительно отношения «подтермин»),
для которого \(x\in\mathcal V(t_x)\) и
\(
  \ulcorner t_x\sigma\urcorner
  =\ulcorner t_x\sigma'\urcorner.
\)
Так как $r$ является возможным кандидатом,
такой термин $t_x$ корректно определён.

Очевидно, $t_x$ не может быть ни переменной, ни константой.
На самом деле легко видеть, что
\(t_x\) обязан иметь вид
\(
  Exp(t_1,t_2^{z_2}\!\cdots t_k^{z_k}).
\)
По Определению 2.6, 3 и так как \(x\in\mathcal V(t_x)\),
для всех \(j\in\{1,\dots,k\}\setminus\{j_0\}\)
термины \(t_j\) являются замкнутыми.
Пусть \(j_0\) — индекс незамкнутого подтермина, т.\,е.\ \(x\in\mathcal V\!\bigl(t_{j_0}\bigr)\).
По минимальности \(t_x\) имеем \(t_{j_0}\sigma\neq t_{j_0}\sigma'\).
Учитывая, что \(t_x\sigma=t_x\sigma'\), рассмотрим возможные случаи:

— $j_0 = 1$: Сначала предположим, что
\(
  \ulcorner t_{j_0}\sigma \urcorner = Exp(b,M)
\)
и
\(
  \ulcorner t_{j_0}\sigma' \urcorner = \operatorname{Exp}(b',M').
\)
Так как \(\ulcorner t_x\sigma \urcorner = \ulcorner t_x\sigma' \urcorner\), имеем
\(b=b'\) и
\(
  \ulcorner M\!\cdot\!\prod_{i=2}^{k} t_i^{z_i} \urcorner
  =
  \ulcorner M'\!\cdot\!\prod_{i=2}^{k} t_i^{z_i} \urcorner.
\)
Отсюда \(M=M'\) и, следовательно,
\(\ulcorner t_{j_0}\sigma \urcorner = \ulcorner t_{j_0}\sigma'\urcorner\),
что противоречит предположению. Если оба термина \(\ulcorner t_{j_0}\sigma \urcorner\) и \(\ulcorner t_{j_0}\sigma' \urcorner\)
не имеют вида \(\operatorname{Exp}(\,\cdot,\cdot\,)\),
то из равенства \(\ulcorner t_x\sigma \urcorner = \ulcorner t_x\sigma' \urcorner\)
также немедленно следует
\(\ulcorner t_{j_0}\sigma \urcorner = \ulcorner t_{j_0}\sigma' \urcorner \).
Если же
\(\ulcorner t_{j_0}\sigma \urcorner = \operatorname{Exp}(b,M)\),
а \(\ulcorner t_{j_0}\sigma' \urcorner\) не имеет вида
\(\operatorname{Exp}(\,\cdot,\cdot\,)\),
то равенство \(\ulcorner t_x\sigma \urcorner = \ulcorner t_x\sigma' \urcorner \)
влечёт
\(
  \ulcorner M\!\cdot\!\prod_{i=2}^{k} t_i^{z_i} \urcorner
  =
  \ulcorner \prod_{i=2}^{k} t_i^{z_i}\urcorner,
\)
а значит \(M=1\), что противоречит тому, что
\(t_{j_0}\sigma\) нормализован.
Следовательно, случай \(j_0 = 1\) невозможен.

— $j_0>1$:  
Сначала предположим, что \(t_1 = Exp(b,M)\).
Из равенства \(\ulcorner t_x\sigma\urcorner = \ulcorner t_x\sigma'\urcorner\) получаем
\[
  \ulcorner M \!\cdot\!
  \ulcorner t_{j_0}^{z_{j_0}}\sigma \urcorner\!
  \cdot
  \prod_{j\in\{2,\dots,k\}\setminus\{j_0\}}
     \bigl(t_j\bigr)^{z_j}\urcorner
  \;=\;
  \ulcorner M\;\cdot\;
  \ulcorner t_{j_0}^{\,z_{j_0}}\sigma'\urcorner\;
  \cdot\!
  \prod_{j\in\{2,\dots,k\}\setminus\{j_0\}} t_j^{\,z_j}\urcorner
\]
Следует
\(
  \bigl(t_{j_0}\sigma\bigr)^{z_{j_0}}
  =
  \bigl(t_{j_0}\sigma'\bigr)^{z_{j_0}},
\)
что противоречит выбору \(t_{j_0}\).
Случай, когда \(t_x\sigma\) не имеет вида
\(\operatorname{Exp}(\,\cdot,\cdot\,)\),
разбирается ещё проще. \qed

Используя предыдущую лемму, покажем, что любой подтермин подстановки,
соответствующей минимальной атаке,
либо принадлежит спецификации протокола,
либо является подтермином нормализованного сообщения,
передаваемого в ходе атаки.

\textsc{Лемма 3.9.}
Пусть \((\pi,\sigma)\) — атака на \(P\) с нормализованной подстановкой
\(\sigma\),
\(x\in\mathcal V(R_i)\) для некоторого
\(i\in\{1,\dots,k\}\),
а \(s\in \mathcal S\!\bigl(\sigma(x)\bigr)\) — стандартный термин.
Тогда существует \(j\le i\) такое, что
\(
  s\in \mathcal S\!\bigl(\ulcorner R_j\sigma\urcorner\bigr)
\)
или существует \(t\in S(P)\) со свойством \(t\sqsubseteq_{\sigma}s\).

\textit{Доказательство.}
Предположим, что не существует \(t\in S(P)\) такого, что
\(t\sqsubseteq_{\sigma}s\),
и при этом
\(s\notin S\!\bigl(\ulcorner R_j\sigma\urcorner\bigr)\)
для всех \(j\le i\).
Отсюда \(s\) не является атомом.
Положим
\(V_j=\mathcal V(R_j)\cup\mathcal V(S_j)\)
и выберем минимальный \(j\), для которого
\(s\in S\!\bigl(V_j\sigma\bigr)\).
По Определению 2.6, 2 имеем
\(s\in S\!\bigl(\mathcal V(R_j)\sigma\bigr)\).
Обозначим
\(
  \sigma'=\ulcorner \sigma[s\!\leftarrow\!1]\urcorner.
\)

По минимальности \(j\) для всех \(l<j\) выполняется
\(R_l\sigma'=R_l\sigma\),
а значит
\(
  \ulcorner R_l\sigma'\urcorner
  =
  \ulcorner R_l\sigma\urcorner.
\)
Поскольку
\(s\notin S\!\bigl(\ulcorner R_j\sigma\urcorner\bigr)\),
получаем
\(
  \ulcorner R_j\sigma'\urcorner[s\!\leftarrow\!1]
  =
  \ulcorner R_j\sigma\urcorner.
\)
По Лемме 3.5 отсюда следует
\(
  \ulcorner R_j\sigma\urcorner
  =
  \ulcorner R_j\sigma'\urcorner.
\)
Заметим, что \(\sigma\) и \(\sigma'\) различаются
хотя бы по одной переменной из
\(\mathcal V(R_1,\dots,R_j)\),
что противоречит Лемме 3.8.\hfill$\square$

\subsubsection*{3.2.3 \,Свойства выводов}
Ниже приводятся несколько полезных свойств выводов, 
позволяющих без труда заменять термины внутри вывода.
Начнём с простого наблюдения, непосредственно вытекающего
из определения правил декомпозиции и композиции.

\textsc{Лемма 3.10.}
Пусть $E$ — нормализованное конечное множество сообщений,
$t$ — сообщение,
а $L$ — $t$-правило, для которого выполнен один шаг вывода
\(
  E \rightarrow_{L} E,\,t,
\)
и при этом $\mathcal S(E,t)\ne \mathcal S(E)$.
Тогда
\(
  \mathcal S(E,t)=\mathcal S(E)\cup\{t\},
\qquad
  \text{и }L\text{ является правилом композиции.}
\)

Следующая лемма утверждает, что если
$t'$ — подтермин термина $t$,
термин $t$ выводится из множества $E$,
но при этом $t'$ не является подтермином ни одного элемента $E$,
то $t'$ также выводится из $E$, причём
последним шагом такого вывода служит правило композиции.

\textsc{Лемма 3.11.}
Предположим, что
$t'\in \mathcal S(t)\setminus \mathcal S(E)$
и
$t\in forge(E)$.
Тогда
\(
  t'\in forge(E),
\)
и существует вывод из $E$ с целью $t'$,
завершающийся правилом композиции.

\begin{proof}
Пусть
\(D=E_0 \rightarrow_{\,L_1\,} E_1 \;\cdots\; \rightarrow_{\,L_n\,} E_n\)
— вывод термина \(t\), причём \(E_0 = E\).
Так как \(t'\) является подтермином элементов \(E_n\),
существует минимальный \(i>0\) такой, что
\(t' \in \mathcal S(E_i)\).
По минимальности \(i\) имеем
\(t'\in \mathcal S(E_i)\setminus \mathcal S(E_{i-1})\).
Применяя Лемму 3.10, заключаем, что
\(D=E_0 \rightarrow_{\,L_1\,} E_1 \;\cdots\; \rightarrow_{\,L_i\,} E_i\)
— это вывод с целью \(t'\).
\end{proof}

\noindent лемма будет использована в доказательстве Леммы 3.13.
Она позволяет строить специальные выводы, в которых заданный термин
никогда не декомпозируется. Это окажется критически важным,
когда потребуется заменить составные термины атомами в ряде выводов.

\textsc{Лемма 3.12.}
Пусть $t\in forge(E)$ и $\gamma\in forge(E)$.
Предположим, что имеется вывод $D_{\gamma}$ из $E$,
последним шагом которого является применение правила из $\mathcal L_{c}$.
Тогда существует вывод $D'$ из $E$ с целью $t$, удовлетворяющий
\(
  L_{d}(\gamma)\notin D'.
\)

\begin{proof} См.\ Приложение А.\end{proof}

\medskip

Теперь мы можем сформулировать лемму, позволяющую
заменять определённые подтермины, возникающие в подстановке атаки,
на более короткие.
Из предположений текущей леммы следует, что
$s$ выводим из $E$ так, что последним правилом является правило композиции.
Это даёт возможность заменить $s$ на более короткий термин,
поскольку при выводе $t$ декомпозиция $s$ уже не потребуется.

\textsc{Лемма 3.13.}
Пусть $E$ и $F$ — два множества нормализованных сообщений, причём
$1\in E\cup F$.
Пусть
$t\in forge(E,F)$ и
$s\in forge(E)$ — неатомарный термин, причём $s\notin \mathcal S(E)$.
Обозначим через $\delta$ подстановку $[\,s\!\leftarrow\!1\,]$.
Тогда
\(
  \ulcorner t\delta\urcorner\;\in\;
   forge\!\bigl(\,\ulcorner E\delta,\,F\delta\urcorner\bigr).
\)

\begin{proof}
По лемме 3.11 существует корректный вывод
$D_s$ из $E$ с целью $s$, в котором последний шаг — правило композиции.
По лемме 3.12 имеется вывод
$D_t$ из $E,F$ с целью $t$, такой, что $L_d(s)\notin D_t$.
Запишем
\(
  D_t:\quad
  E,F \rightarrow_{L_1} E,F,t_1
       \rightarrow_{L_2} E,F,t_2
       \;\cdots\;
       \rightarrow_{L_n} E,F,t_1,\dots,t_n .
\)

Покажем по индукции по $i\;(0\le i\le n)$, что
\[
  \ulcorner t_i\delta\urcorner\;\in\;
   forge\!\bigl(\,\ulcorner E\delta,\,F\delta\urcorner\bigr).
\]

\noindent\emph{База.}
Если $t_0\in E\cup F$, то
\(
  \ulcorner t_0\delta\urcorner\in
  \ulcorner E\delta\cup F\delta\urcorner
  \subseteq
  forge\!\bigl(\,\ulcorner E\delta,\,F\delta\urcorner\bigr).
\)

\emph{Индукционный шаг.}
Пусть утверждение доказано для всех $j<i$; рассмотрим правило $L_i$.

\begin{itemize}
  \item[--$L_i=L_c(\langle a,b\rangle)$:]
        либо $t_i=s$ и тогда
        $t_i\delta=1\in forge(E\delta,F\delta)$,
        либо
        $t_i\delta=\langle a\delta,b\delta\rangle$.
        По предположению индукции
        $\{a\delta,b\delta\}\subseteq
          forge\!\bigl(\,\ulcorner E\delta,\,F\delta\urcorner\bigr)$,
        откуда
        $t_i\delta\in
         forge\!\bigl(\,\ulcorner E\delta,\,F\delta\urcorner\bigr)$.
        Аналогично разбираются случаи $\{a\}_b^{\,s}$ и $\{a\}_K^{\,p}$.

  \item[--$L_i=L_{p1}(\langle t_i,a\rangle)$:]
        здесь $s\neq\langle t_i,a\rangle$,
        так как $L_i\notin L_d(s)$.
        Получаем
        $\langle t_i,a\rangle\delta=\langle t_i\delta,a\delta\rangle$.
        По индукционной гипотезе
        $\langle t_i,a\rangle\delta\in
         forge(E\delta,F\delta)$,
        следовательно
        $t_i\delta\in forge(E\delta,F\delta)$.
        Точно так же обрабатываются правила
        $L_{p2},L_{sd}$ и $L_{ad}$.

  \item[--$L_i\in L_o$:]
        воспользуемся Определением 2.11, 3.
        Пусть $E'$ — множество сообщений, полученное в $D_s$
        непосредственно перед применением последнего шага.
        Тогда
        $E'\setminus\{s\}\rightarrow_{\mathcal L_c(s)}E',s$,
        а также
        \[
          E,F,t_1,\dots,t_{i-1}\rightarrow_{L_i}
          E,F,t_1,\dots,t_i .
        \]
        В частности,
        $E',E,F,t_1,\dots,t_{i-1}\rightarrow_{L_i}E',E,F,t_1,\dots,t_{i-1},t_{i}$,
        откуда по Определению 2.11, 3
        \[
          \ulcorner t_i\delta\urcorner\in
          forge\!\bigl(
            \ulcorner E'\delta,\,
            E\delta,\,F\delta,\,
            t_1\delta,\dots,t_{i-1}\delta\urcorner
          \bigr).
        \]
        Заметим, что $E'\delta=E'$, $E\delta=E$,
        а все сообщения из $E'$ выводимы из $E$.
        Индукционное предположение даёт
        $\ulcorner t_1\delta \urcorner,\dots,\ulcorner t_{i-1}\delta \urcorner\in
         forge\!\bigl(\,\ulcorner E\delta,\,F\delta\urcorner\bigr)$,
        поэтому
        \(
          forge\!\bigl(
            \ulcorner E'\delta,\,
            E\delta,\,F\delta,\,
            t_1\delta,\dots,t_{i-1}\delta\urcorner
          \bigr)
          \subseteq
          forge\!\bigl(\,\ulcorner E\delta,\,F\delta\urcorner\bigr),
        \)
        и поэтому
        \(
          t_i\delta\in
           forge\!\bigl(\,\ulcorner E\delta,\,F\delta\urcorner\bigr).
        \)
\end{itemize}

Для $i=n$ получаем
\(
  t\delta\in forge(E\delta,F\delta),
\)
что и требовалось.
\end{proof}

\subsubsection*{3.2.4 \,Свойства минимальных атак}

Теперь мы готовы перейти к доказательству Предложения 3.14, которое утверждает, что
подстановки минимальных атак всегда можно построить,
связывая подтермины, изначально присутствующие в протоколе~$P$.
Это ключевой шаг, позволяющий ограничить число подтерминов
в минимальных атаках (см.\ Теорему 3.15).

\textsc{Предложение 3.14.}
Пусть $(\pi,\sigma)$ — минимальная атака на протокол $P$.
Тогда для любого
\(s\in S(\mathcal V\sigma)\)
существует
\(t\in \mathcal S(P)\) такое, что \(t\sqsubseteq_{\sigma}s\).

\begin{proof}
Положим $(\pi,\sigma)$ и $s$ как выше.
Предположим обратное\;(*)\,: \emph{для каждого $t$ из условия
$t\sqsubseteq_{\sigma}s$ следует $t\notin \mathcal S(P)$}.
Доведём это допущение до противоречия.

Так как $\mathcal A\subseteq \mathcal S(P)$, получаем $s\notin \mathcal A$.
По лемме 3.9 и (*)
существует $j$ такое, что
\(s\in \mathcal S\!\bigl(\ulcorner R_j\sigma\urcorner\bigr)\).
Выберем минимальный среди возможных $j$ и
рассмотрим $N=j$.
Если
\(s\in \mathcal S\!\bigl(\ulcorner S_i\sigma\urcorner\bigr)\)
для некоторого $i$, то (*)
и лемма 3.7 дают нам переменную
\(x\in\mathcal V(S_i)\) с
\(s\in S(x\sigma)\).
По определению 2.6, (2)
существуют $R_{i'},i'\le i$ такие, что
\(x\in\mathcal V(R_{k})\).
Следовательно, леммы 3.9 и (*) вновь
дают индекс $j\le i$ с
\(s\in S\!\bigl(\ulcorner R_j\sigma\urcorner\bigr)\).
Заметим, что
\(s\notin \mathcal S(S_0)\), иначе
\(s\in \mathcal S(P)\), и (*)
бы не выполнялось.
Минимальность $N$ гарантирует $i\ge N$.
Положим
\(E_j=\ulcorner S_0\sigma,\dots,S_{j-1}\sigma \urcorner\).
Итак, $s$ — неатомарный термин,
не являющийся подтермином $E_N$,
но являющийся подтермином
\(\ulcorner R_N\sigma\urcorner\);
по лемме 3.11 тогда
\(s\in forge(E_N)\).

Обозначим $\delta=[\,s\!\leftarrow\!1\,]$.
Поскольку $(\pi,\sigma)$ — атака,
для всех $1\le j\le k+1$
(где $R_{k+1}=\textit{secret}$) выполняется
\[
  \ulcorner R_j\sigma\urcorner \;\in\;
   forge(E_j).
\]

Рассмотрим два случая.

\medskip
\noindent— \textbf{$j<N$.}
Минимальность $N$ означает, что $s$ не является подтермином
ни \(\ulcorner R_j\sigma\urcorner\), ни $E_j$.
Тогда из
\(\ulcorner R_j\sigma\urcorner\in forge(E_j)\)
получаем
\(
  \ulcorner\ulcorner R_j\sigma\urcorner\delta\urcorner
  \in
  forge\!\bigl(\ulcorner E_j\delta\urcorner\bigr)
  =forge(E_j\delta).
\)

\medskip
\noindent— \textbf{$j\ge N$.}
Полагая
\(t=\ulcorner R_j\sigma\urcorner,\;E=E_N,\;F=E_j\),
применяем лемму 3.13 и
получаем
\(
  \ulcorner\ulcorner R_j\sigma\urcorner\delta\urcorner
  \in
   forge\!\bigl(\ulcorner E_j\delta\urcorner\bigr).
\)

\noindent
Таким образом,
\(\ulcorner\ulcorner R_j\sigma\urcorner\delta\urcorner\in
  forge(\ulcorner E_j\delta\urcorner)\)
в обоих случаях.
Теперь из (*) и леммы 3.5
для всех $j$ выводим
\[
  \ulcorner R_j\sigma\urcorner
  \in
   forge
  \bigl(
    \ulcorner S_0\sigma',\dots,S_{j-1}\sigma'\urcorner
  \bigr),
  \qquad
  \text{где }\sigma'=\ulcorner\sigma\delta\urcorner.
\]
Следовательно, $(\pi,\sigma')$ также является атакой.
(Условия применения леммы 3.5 выполнены.)
Но $\sigma'$ получена из $\sigma$ заменой $s$ на термин $1$,
то есть
\(|\sigma'|<|\sigma|\),
что противоречит минимальности атаки $(\pi,\sigma)$.
\end{proof}

\subsection{Ограничение числа подтерминов минимальных атак}

Опираясь на Предложение 3.14, получаем:

\textsc{Теорема 3.15.}
Для каждой минимальной атаки $(\pi,\sigma)$ на протокол $P$ выполнено
\[
  \mathcal S\!\bigl(\,\ulcorner \mathcal S(P)\sigma\urcorner\bigr)\;=\;
  \ulcorner \mathcal S(P)\sigma\urcorner.
\]

\textit{Доказательство.}
Включение
\(
  \ulcorner \mathcal S(P)\sigma\urcorner
  \subseteq
  S\!\bigl(\,\ulcorner \mathcal S(P)\sigma\urcorner\bigr)
\)
очевидно.
Обратное включение является прямым следствием
Предложения 3.14, из которого получаем
\[
  \mathcal S(\mathcal V\sigma)
  \subseteq
  \ulcorner \mathcal S(P)\sigma\urcorner.
\]
По лемме 2.4 имеем
\(
  S\!\bigl(\,\ulcorner \mathcal S(P)\sigma\urcorner\bigr)
    \subseteq
  \ulcorner \mathcal S\bigl(\mathcal S(P)\sigma\bigr)\urcorner,
  \qquad
  \mathcal S\bigl(\mathcal S(P)\sigma\bigr)
    \subseteq
  \mathcal S(P)\sigma\;\cup\; \mathcal S(\mathcal V\sigma).
\)
Следовательно,
\(
  \mathcal S\!\bigl(\,\ulcorner \mathcal S(P)\sigma\urcorner\bigr)
    \subseteq
  \ulcorner \mathcal S(P)\sigma\urcorner
  \;\cup\;
  \mathcal S(\mathcal V\sigma),
\)
и, учитывая предыдущее включение,
получаем
\(
  S\!\bigl(\,\ulcorner S(P)\sigma\urcorner\bigr)
    \subseteq
  \ulcorner S(P)\sigma\urcorner.
\)
Тем самым равенство доказано.\hfill$\square$

Из Теоремы 3.15 немедленно следует:

\textsc{Следствие 3.16.}
Для каждой минимальной атаки $(\pi,\sigma)$ на протокол $P$
и любого множества $E\subseteq \mathcal S(P)$ выполняется
\(
  \bigl|\,\ulcorner E\sigma\urcorner\bigr|\;\le\;|P|.
\)
В частности,
\(
  |\,\mathcal V(P)\sigma|\;\le\;|P|.
\)

\begin{proof}
Во-первых, мощность множества
\(\mathcal S\!\bigl(\,\ulcorner \mathcal S(P)\sigma\urcorner\bigr)\)
не превышает $|P|$.
Во-вторых,
\(
  \ulcorner E\sigma\urcorner
  \subseteq
  \mathcal S\!\bigl(\,\ulcorner \mathcal S(P)\sigma\urcorner\bigr),
\)
откуда сразу следует требуемое
\(
  \bigl|\,\ulcorner E\sigma\urcorner\bigr|\le|P|.
\)
Подставляя $E=\mathcal V(P)$, получаем
\(
  |\,\mathcal V(P)\sigma|\le|P|.
\)
Заметим, что $\sigma$ нормализована, поэтому
\(
  \sigma(x)=\ulcorner\sigma(x)\urcorner
\)
для любой переменной $x$.
\end{proof}
