\section{Проектирование собираемых метрик}

Так как TCF состоит из технологических ролей (tcf-worker и tcf-coordinator) и репликаторов данных
(Gateway и Destination), которыми
являются отдельные независимые сервисы, сбор метрик нужно интегрировать в каждый из этих компонентов.
Использование этих метрик позволяет администраторам своевременно выявлять проблемы, анализировать
производительность и обеспечивать стабильную работу кластеров и межкластерных репликаторов данных. Таким
образом все собираемые метрики можно разделить на несколько категорий:

\begin{itemize}
  \item метрики, собираемые с инстансов Tarantool;
  \item метрики, собираемые с сервисов репликаторов.
\end{itemize}

Метрики предоставляются в формате Prometheus. Используются следующие типы метрик Prometheus:

\begin{itemize}
  \item counter -- монотонно возрастающий счетчик;
  \item gauge -- числовое значение, которое может как возрастать, так и убывать.
\end{itemize}

Так как TCF -- продукт про репликацию, проектирование метрик сделаны с упором на эти параметры.
В качестве определяющей метрики для ее отслеживания была выбрана метрика vclock-signature (сигнатура
vclock). Сигнатура -- это сумма всех элементов вклока, исключая нулевую компоненту. Vclock расшифровывается
как vector clock (векторные часы). Они состоят из ассоциативного массива, где ключом является
идентификатор сервера репликасета, а значением последний LSN (log sequence number), когда сервер был
мастером. LSN -- число (счетчик), инкрементирующийся каждый раз, когда в журнале WAL (write ahead log)
добавляется новая подтвержденная запись.

Таким образом, если на каком-то из компонентов значения сигнатур не будут совпадать, администратор сможет
найти место, где происходит отставание и своевременно обнаружить проблему.

Ниже указаны название метрик, относящиеся к каждому компоненту TCF, а также их назначение. 

\subsection{Метрики технологических ролей}

Метрики позволяют отслеживать статус кластера, количество переданных и прочитанных данных, а также
диагностировать ошибки при обмене данными между кластерами (в скобках будет указано название метрики в
формате Prometheus \cite{prometheus}):

\begin{itemize}
  \item Состояние кластера (\textit{tcf\_is\_active}) -- активность текущего кластера.
  Тип в prometheus: \textit{gauge}. Принимает значения 1 (активный кластер) и 0 (пассивный кластер).
  \item tcf\_dst\_vclock\_signature -- сигнатура vclock, применённая на соседнем кластере.
  Тип: gauge. Отображает состояние репликации на стороне кластера-получателя (Destination). Может
  использоваться для сравнения с \textit{tcf\_src\_vclock\_signature}. Отгруппирована по uuid репликасета.
  \item tcf\_source\_vclock\_signature -- последняя записанная
  vclock-сигнатура на исходном кластере. Тип: gauge. Как говорилось ранее, разница между значениями vclock
  на кластерах может сигнализировать о задержках в репликации. Отгруппирована по uuid репликасета.
\end{itemize}

\subsection{Метрики Gateway}

Для того, чтобы включить экспорт метрик с сервиса по HTTP, нужно в поле \textit{gateway.metrics\_enabled}
файла конфигурации выставить значение \textit{true}. Были выделены следующие метрики:

\begin{enumerate}[label=\arabic*.]
  \item \textit{tcf\_gateway\_sent\_total} -- суммарное количество записей, отправленных на компонент
  Destination. Тип: counter. Отгруппированы по названию спейса (таблицы) и uuid репликасета.
  \item \textit{tcf\_gateway\_sent\_errors\_total} -- суммарное количество ошибок, возникших при отправке
  данных на компонент Destination. Тип: counter. Отгруппированы по названию спейса (таблицы) и uuid репликасета.
  \item \textit{tcf\_gateway\_read\_total} -- суммарное количество записей, прочитанных с исходного
  кластера. Тип: counter. Отгруппированы по названию спейса (таблицы) и uuid репликасета.
  \item \textit{tcf\_gateway\_read\_errors\_total} -- суммарное количество ошибок, возникших при чтении
  данных с исходного кластера. Тип: counter. Отгруппированы по названию спейса (таблицы) и uuid репликасета.
  \item \textit{tcf\_gateway\_limbo\_vclock\_signature} -- сигнатура vclock из limbo. Тип: gauge.
  Показывает vclock signature, полученную из лимбо (структура данных, обрабатывающая транзакционный
  поток репликации), отгруппированная по набору реплик и Gateway, через который идет нагрузка.
  Отгруппированы по uuid репликасета.
  \item \textit{tcf\_gateway\_sent\_vclock\_signature} -- сигнатура vclock, отправленная из Gateway в
  Destination. Тип: gauge. Фиксирует сигнатуру vclock, отправленную из Gateway в Destination.
  Используется для контроля синхронизации между кластерами. Группируется по наборам реплик.
  Отгруппированы по uuid репликасета.
  \item \textit{tcf\_gateway\_http\_responses\_total} -- показывает количество HTTP-ответов от Gateway с
  конкретным методом, путем и статусом. Отгруппированы по типу запроса, пути запроса, статус-кода ответа.
\end{enumerate}

\subsection{Метрики Destination}

Для того, чтобы включить экспорт метрик с сервиса по HTTP, нужно в поле \textit{destination.metrics\_enabled}
файла конфигурации выставить значение \textit{true}. Аналогичные метрики с минимальными отличиями созданы в Destination:

\begin{enumerate}[label=\arabic*.]
  \item \textit{tcf\_destination\_recv\_total} -- общее количество событий, полученных от компонента Gateway. Тип: counter. Отгруппированы по uuid репликасета и имени спейса (таблицы).
  \item \textit{tcf\_destination\_recv\_errors\_total} -- общее количество ошибок при получении данных. Тип: counter. Отгруппированы по типу ошибки.
  \item \textit{tcf\_destination\_push\_total} -- cуммарное количество событий, отправленных в
  Destination. Тип: counter.
  \item \textit{tcf\_destination\_push\_errors\_total} -- количество ошибок, возникших в Destination
  при попытке отправить данные на целевой кластер. Тип: counter.
  \item \textit{tcf\_destination\_recv\_vclock\_signature} -- сигнатура vclock, полученная от
  компонента Gateway. Тип: gauge. Содержит сумму всех ненулевых значений вектора vclock, полученного
  на Destination от Gateway. Используется для оценки состояния репликации на стороне Destination.
  \item \textit{tcf\_destination\_sent\_vclock\_signature} -- сигнатура vclock, отправленная из
  Destination. Тип: gauge. Фиксирует сумму ненулевых значений vclock, отправленных Destination в его
  кластер. Позволяет отслеживать текущее состояние репликации.
  \item \textit{tcf\_destination\_http\_responses\_total} -- показывает количество HTTP-ответов от
  Destination с конкретным методом, путем и статусом.
\end{enumerate}

\subsection{Системные метрики}

Также были введены системные метрики для репликаторов для мониторинга системных параметров, таких как
количество потоков, отданных системой приложению, объем потребляемой памяти, динамика работы сборщика
мусора и многое другое. Все описанные метрики были интегрированы в код программы и покрыты тестами. Для
того, чтобы экспортировать метрики из приложения, нужно обратиться по HTTP порту по адресу
\textit{http://<server\_addr:server\_port>/metrics}.
