\introduction

Современные распределённые системы предъявляют всё более высокие требования 
к доступности данных, устойчивости к отказам и способности непрерывно 
функционировать при различных видах сбоев. По мере увеличения масштабов 
информационных систем и расширения их географического охвата становится 
необходимым использовать решения, обеспечивающие согласованность 
и доступность данных не только внутри одного кластера, но и между 
удалёнными дата-центрами. Такие механизмы особенно важны для критически 
важных сервисов, где недоступность данных в одном регионе не должна приводить 
к остановке работы всей системы.

Настоящая работа направлена на разработку и исследование 
геораспределённого механизма межкластерной репликации данных, а также 
построение поверх него отказоустойчивой архитектуры с автоматическим 
переключением ролей кластеров. В отличие от типичных решений, 
ограниченных внутрикластерной репликацией, здесь рассматривается 
полноценное взаимодействие независимых кластеров, каждый из которых 
может функционировать автономно, но при этом обязан обеспечивать 
согласованное состояние данных при работе в составе распределённой системы.

Работа разделена на два функциональных модуля:

\begin{itemize}
    \item Разработка механизма асинхронной репликации 
    данных между кластерами. На этом этапе необходимо спроектировать 
    и реализовать протокол обмена данными, обеспечивающий передачу снимков 
    состояния и последовательности событий изменений (логов), а также 
    корректное восстановление данных на стороне принимающего кластера. 
    Особое внимание уделяется формату данных, стратегиям передачи, 
    гарантии идемпотентного применения, а также взаимодействию компонентов 
    при первичном и последующих запускax репликации.

    \item Реализация механизма отказоустойчивости и 
    автоматического переключения между кластерами. Данный модуль 
    дополняет разработанный ранее протокол репликации средствами 
    мониторинга состояния кластеров, выбора текущего «активного» кластера 
    и безопасного переключения при возникновении отказов. В рамках этой 
    части требуется реализовать алгоритмы определения лидирующего кластера, 
    процессы перехода роли, а также обеспечить непротиворечивое продолжение 
    репликации после переключения.
\end{itemize}
