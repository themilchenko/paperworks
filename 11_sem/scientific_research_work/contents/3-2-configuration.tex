\subsection{Конфигурация отказоустойчивости}

Для управления поведением механизма отказоустойчивости в разработанной системе
используется декларативная конфигурация, задаваемая в YAML-файле при запуске
кластера. Такой подход позволяет гибко настраивать политику переключения ролей
кластеров без изменения программного кода и адаптировать систему под различные
эксплуатационные требования.

Конфигурация механизма отказоустойчивости логически выделена в отдельный раздел
\texttt{failover} и применяется симметрично к каждому из Raft-кластеров. Пример
фрагмента конфигурации приведён ниже:

\begin{lstlisting}[frame=rlbt,caption={Блок конфигурации отказоустойчивости системы}]
failover:
  enabled: true
  coordinator: etcd
  etcd_endpoints: ["127.0.0.1:2379"]
  lease_ttl_seconds: 10
  renew_interval_seconds: 3
  min_healthy_nodes: 3
\end{lstlisting}

Назначение параметров конфигурации следующее:

\begin{itemize}
  \item \texttt{enabled} — включает или отключает механизм отказоустойчивости для
  данного кластера. При отключённом значении кластер функционирует в автономном
  режиме без взаимодействия с внешним координатором, что может использоваться
  для локального тестирования и отладки.

  \item \texttt{coordinator} — задаёт тип внешнего координатора состояния,
  используемого для предотвращения сценариев \emph{split-brain}. В рамках данной
  работы используется значение \texttt{etcd}, отражающее применение
  распределённого ключ-значение хранилища с поддержкой аренд и атомарных операций.

  \item \texttt{etcd\_endpoints} — список сетевых адресов узлов координатора
  состояния. Использование нескольких адресов повышает отказоустойчивость
  взаимодействия с координатором и снижает вероятность ложных переключений,
  вызванных временной недоступностью отдельного узла.

  \item \texttt{lease\_ttl\_seconds} — время жизни аренды \texttt{active\_lock} в
  координаторе. Если активный кластер не продлевает аренду в течение данного
  интервала, ключ автоматически удаляется, что инициирует процедуру повышения
  роли пассивного кластера.

  \item \texttt{renew\_interval\_seconds} — периодичность продления аренды активным
  кластером. Значение параметра выбирается таким образом, чтобы в пределах одного
  TTL выполнялось несколько попыток продления, что повышает устойчивость системы
  к кратковременным задержкам и перегрузкам.

  \item \texttt{min\_healthy\_nodes} — минимально допустимое количество «здоровых»
  узлов в активном кластере. Под здоровыми узлами понимаются узлы, поддерживающие
  связность с Raft-лидером. При снижении фактического числа доступных узлов ниже
  заданного порога активный кластер считается деградировавшим и добровольно
  прекращает продление аренды \texttt{active\_lock}, что приводит к автоматическому
  освобождению права на запись.
\end{itemize}
