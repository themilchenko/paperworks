\newglossaryentry{id21}{
  name={Координатор состояния},
  description={компонент распределённой системы, обеспечивающий согласованное хранение и обновление глобального состояния кластеров}
}

\newglossaryentry{id23}{
  name={Атомарная операция},
  description={операция, которая либо выполняется полностью, либо не выполняется вовсе, не оставляя промежуточных состояний}
}

\newglossaryentry{id24}{
  name={TTL},
  description={время жизни объекта в системе, по истечении которого он автоматически удаляется}
}

\newglossaryentry{id25}{
  name={Lease renewal},
  description={процедура продления аренды ресурса для подтверждения актуальности владельца}
}

\newglossaryentry{id28}{
  name={Bootstrap},
  description={процедура первичного запуска нового кластера с созданием начального состояния хранилища}
}

\newglossaryentry{id34}{
  name={Глобальная блокировка},
  description={механизм исключительного владения активной ролью, предотвращающий одновременную активность нескольких кластеров}
}

\newglossaryentry{id35}{
  name={Здоровый узел},
  description={узел кластера, доступный по сети и корректно участвующий в протоколе консенсуса}
}

\newglossaryentry{id36}{
  name={Политика отказоустойчивости},
  description={набор правил, определяющих условия переключения ролей между кластерами}
}

\newglossaryentry{id37}{
  name={Деградация кластера},
  description={состояние, при котором кластер формально работоспособен, но не удовлетворяет заданным требованиям доступности}
}
