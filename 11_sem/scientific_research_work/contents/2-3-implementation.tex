\section{Демонстрация механизма асинхронной репликации между кластерами}

% Тут должны быть следующие сценарии:
%
% 1. Поднять два кластера через clusterctl start. Показать и объяснить
% структуру файлов для каждого кластера. 2. Продемострировать основные события
% в логах: калстер приянял заданный в конфиге статус, репликационный поток
% начался. В случае сервера он начал слушать grpc порт, а в случае клиента по
% round-robin искать мастера и подписываться на его изменения в вале. 3.
% Демонстрация взаимодействия с активным кластером. Минимальный круд на нем,
% проверка, что данные реплицируются внутри кластера. В случае пассивного
% показать, что данные с активного пришли (можно вставить лог сообщения), а
% запись на активный функционально невозможна. 4. Показать, что в случае отказа
% мастера на активном кластере и маастера на пассивном, система продолжает
% работать.
\subsection{Описание конфигурации активного и пассивного кластеров}

Для демонстрации механизма межкластерной асинхронной репликации были развёрнуты
два независимых кластера, каждый из которых содержит по пять узлов. Оба
кластера имеют одинаковую внутреннюю топологию и различаются только ролью
(\textit{active} / \textit{passive}) и адресным пространством.

Внутреннее устройство конфигурационных файлов полностью соответствует формату,
описанному ранее, однако в рамках демонстрации их параметры имеют отдельное
назначение.

На листинге~\ref{listing:active} предствалена конфигурация активного кластера. Отличающими опциями являются поле $cluster\_status$, которое задает стартовое состояние. Не смотря на то, что кластер активный, у него есть список $follow\_list$. Это сделано с расчетом на будущее проектирование механизма отказоустойчивости: если текущий кластер поменяет статус на противоположный, то он знал, к каким адресам ему подключаться для репликации.

\listing[
    caption={Конфигурация активного кластера},
    label=listing:active
]{assets/active\_cluster.yml}

Аналогично, на листинге~\ref{listing:passive} представлена конфигурация пассивного кластера. Отличия, как говорилось ранее, в указании адресов для прослушивания, а также в имене и его изначальном статусе.

\listing[
    caption={Конфигурация пассивного кластера},
    label=listing:passive
]{assets/passive\_cluster.yml}

\subsection{Старт кластеров}

На листинге~\ref{lst:start} показан процесс запуска двух кластеров с
использованием утилиты clusterctl. Каждый кластер поднимается на основе
собственного конфигурационного файла, после чего все узлы инициализируют
локальные каталоги, Raft-компоненты и сетевые интерфейсы. Вывод демонстрирует
корректный старт всех узлов и формирование рабочей топологии.

\listing[
    caption={Старт кластеров с помощью утилиты clusterctl},
    label=lst:start
]{assets/start\_clusters.out}

На листинге~\ref{lst:struct} представлена структура файлов работы кластеров.

\listing[
    caption={Структура системной директории кластеров},
    label=lst:struct
]{assets/files\_structure.out}

Для каждого кластера формируется отдельная директория внутри var/, содержащая подпапки для узлов. Каждый узел имеет:

\begin{itemize}
  \item журнал WAL;
  \item каталог снапшотов;
  \item PID-файл;
  \item лог работы.
\end{itemize}

Структура подтверждает корректную инициализацию подсистем WAL и snapshot для всех узлов обоих кластеров.

На листинге~\ref{lst:status} представлен вывод команды $clusterctl status$ для активного и пассивного
кластеров. Листинг показывает лидера каждого кластера и список последовательных
узлов (followers), что подтверждает стабильную работу алгоритма Raft и
корректное формирование кворума в обоих кластерах.

\listing[
    caption={Статус активного и пассивного кластеров соответственно},
    label=lst:status
]{assets/status\_clusters.out}

На листинге~\ref{lst:fill} демонстрирует практическую работу репликации. Сначала в активный
кластер вставляются десять пар ключ–значение с помощью HTTP-запросов. Затем
выполняются запросы /keys как к активному, так и к пассивному кластеру. Оба
ответа идентичны, что подтверждает успешный механизм межкластерной репликации:
все изменения, произведённые в активном кластере, были корректно доставлены и
применены в пассивном.

\listing[
    caption={Заполнение активного кластера и демонстрация репликации на пассивный кластер},
    label=lst:fill
]{assets/fill\_db.out}

Листинг~\ref{lst:replication} содержит фрагмент внутреннего журнала пассивного лидера. В логе
фиксируются: получение снапшота (если это первый запуск), подключение к
gRPC-потоку, последовательное получение WAL-записей, применение каждого события
к локальному хранилищу, обновление счётчика $last\_applied$.

\listing[
    caption={Лог репликации мастера пассивного кластера},
    label=lst:replication
]{assets/replication.log}

Этот лог подтверждает, что механизм репликации работает в потоковом режиме и не теряет события даже при многократных обновлениях.
