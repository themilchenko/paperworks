\abstract

Ключевые слова: алгоритм Raft, распределённые системы, key-value хранилище,
отказоустойчивость, репликация, геораспределенные системы, межкластерная
репликация, XDRC.

Данная работа посвящена разработке геораспределённого механизма межкластерной
репликации данных и обеспечению отказоустойчивости на уровне взаимодействия
отдельных инсталляций распределённого хранилища. Исследование проводится в два
этапа, каждый из которых соответствует отдельному модулю.

Первый модуль направлен на разработку механизма асинхронной репликации данных
между изолированными кластерами (ЦОДами). Основная цель данного этапа — обеспечить
передачу данных из активного кластера в пассивный без требования строгой
согласованности в реальном времени.

Второй модуль посвящён разработке механизма отказоустойчивости и
автоматического переключения. Основная задача этого этапа заключается в том,
чтобы пассивный кластер мог корректно и надёжно перейти в активное состояние
при недоступности основного, а также обеспечить гарантированное достижение
консистентности перед переключением. На данном этапе рассматриваются вопросы
обнаружения отказов, координации переключения ролей, а также безопасного
завершения процесса репликации.

Итоговая система должна обеспечивать полноценную геораспределённую репликацию и
автоматическое восстановление работоспособности при сбоях.
