\conclusion

В рамках данной научно-исследовательской работы была спроектирована и реализована
геораспределённая система межкластерной репликации данных (xdrc), ориентированная
на использование Raft-кластеров и обеспечивающая отказоустойчивость на уровне
кластеров целиком. В основе системы лежит модель active/passive, при которой
в каждый момент времени операции записи допускаются ровно в одном кластере,
а второй кластер получает данные посредством асинхронной межкластерной репликации.

Ключевым результатом работы является разработка механизма автоматического
переключения ролей кластеров без ручного вмешательства администратора и без
возникновения сценариев \emph{split-brain}. Для этого был введён внешний
координатор состояния, реализованный на базе распределённого ключ-значение
хранилища etcd. Использование примитивов аренды (lease), атомарных транзакций
и дополнительного механизма версионирования состояния (epoch) позволило
гарантировать, что право на выполнение операций записи в системе всегда
принадлежит не более чем одному кластеру.

В ходе работы была реализована и экспериментально подтверждена корректность
работы механизма отказоустойчивости в двух принципиально различных сценариях:
при полном отказе активного кластера и при частичной деградации, выражающейся
в снижении числа доступных узлов ниже заданного порога. Особое внимание было
уделено второму сценарию, в котором активный кластер остаётся формально
работоспособным, однако добровольно отказывается от роли активного на основании
локально наблюдаемой деградации. Такой подход снижает вероятность некорректной
работы в условиях потери кворума и повышает предсказуемость поведения системы
при отказах.

Реализация межкластерной репликации основана на передаче снапшотов состояния и
потока WAL-записей по gRPC и обеспечивает согласованность данных между кластерами
с учётом асинхронной модели доставки. Проведённые эксперименты показали, что
после переключения ролей система сохраняет целостность данных, корректно
обрабатывает возврат ранее отказавшего кластера и не допускает возобновления
приёма записей кластером, утратившим актуальную эпоху.

Следует отметить ряд ограничений предложенного решения. Во-первых, система
обеспечивает отказоустойчивость на уровне кластеров, но не реализует
прозрачную балансировку клиентских запросов внутри кластера и между кластерами.
Во-вторых, межкластерная репликация носит асинхронный характер, что допускает
ограниченное отставание пассивного кластера и потенциальную потерю части
последних записей при аварийном переключении. Эти ограничения осознанны и
соответствуют выбранной архитектурной модели.

В качестве направлений дальнейшего развития системы можно выделить расширение
политик принятия решений при промоуте кластеров (учёт задержек репликации,
метрик нагрузки и сетевой доступности), интеграцию механизмов автоматического
перенаправления клиентских запросов, а также формализацию модели отказов
и анализ её свойств с использованием методов теории распределённых систем.
