\section{Проблемы миграции к постквантовой криптографии}

Необходимость перехода к постквантовой криптографии обсуждается очень активно,
однако вопрос перехода от классических криптографических систем к постквантовым
до сих пор остаётся недостаточно изученным. Поскольку переход к алгоритмам PQC
неизбежен, становится особенно важным исследовать способы его реализации.
Однако из-за масштабности сети Интернет и цифровизации практически всех сфер
такой переход окажется значительно сложнее, чем кажется. Ниже приведены
ключевые сложности, связанные с переходом к PQC.

\begin{enumerate}
  \item Непредсказуемые сроки развития квантовых вычислений:
  существует риск ускоренного появления квантовых компьютеров, темпы разработки
  которых могут превысить ранее ожидаемые.
  \item Сложные критерии перехода к
  PQC: переход от нынешних алгоритмов с открытым ключом к постквантовым
  решениям не является простым обновлением программного обеспечения. Этот факт
  подчёркивается и в документах NIST~\cite{NIST2017SubmissionReqs}.
  \item Атаки
  типа «записать сейчас — взломать позже»: перехваченные сегодня зашифрованные
  данные могут быть расшифрованы квантовыми компьютерами спустя годы, если к
  тому моменту информация сохранит свою ценность.
  \item Выбор релевантных
  стандартов NIST: при подготовке к этому переходу необходимо учитывать последствия
  внедрения новых стандартов PQC ещё на стадии стандартизации.
\end{enumerate}

Существует ещё много работы, необходимой для лучшего понимания вопросов
перехода и решения задач интеграции, безопасности, производительности и других
аспектов, чтобы гарантировать безопасное внедрение квантовых решений в
экосистему глобальной индустрии. При рассмотрении исследовательских проблем
следует выделить две пересекающиеся области:

\begin{enumerate}
  \item \textit{Исследования по переходу на постквантовую криптографию}: Это
исследования того, как кандидатные алгоритмы могут быть использованы в
определённых условиях, а также исследования безопасного перехода для конкретной
области криптографического применения.
  \item \textit{На пути к науке криптографической гибкости}: Организации используют
криптографическую гибкость как подход к шифрованию данных, позволяющий быстро
реагировать на криптографические атаки. Цель криптоагильности — переходить на
новые криптографические стандарты без необходимости в крупных изменениях
инфраструктуры. Гибкость в криптографическом домене проявляется в обобщённом
подходе к переходу на PQC различными способами.
\end{enumerate}

Существует значительное сходство между широким пространством перехода к PQC и
областью криптографической гибкости. Эти сходства демонстрируют весь спектр
трудностей, которые необходимо решить, чтобы обеспечить переход
криптографических систем на постквантовые алгоритмы. В то же время имеются
области, где эти две дисциплины не совпадают или вовсе не пересекаются.

\subsection{Предмиграционные вызовы PQC}

Переход от современных алгоритмов открытого ключа к постквантовой криптографии
не является простой задачей, подобной обновлению программного обеспечения до
новой версии, что также было признано NIST \cite{NIST2017SubmissionReqs}. PQC
затрагивает размеры ключей, шифртестов и подписей, а также требования к
взаимодействию и вычислениям. В итоге NIST заявил, что будет выбран только один
алгоритм в качестве основной замены, поскольку остальные алгоритмы имеют
различные компромиссы по размерам ключей и вычислительным требованиям. В целом
NIST считает безопасным предоставление нескольких вариантов в рамках новых
стандартов PQC.

Ниже представлены несколько областей, которые могут получить выгоду от
дополнительных исследований:

1. \textit{Соображения производительности.}  Алгоритмы PQC требуют больше
вычислительных ресурсов, памяти, хранилища и пропускной способности, поэтому
оценка производительности в различных сценариях внедрения является критически
важной.

2. \textit{Соображения безопасности.}  Переход на новые алгоритмы создаст
множество новых проблем безопасности. Поскольку PQC изучена хуже, чем
существующие системы RSA и ECC, возникают вопросы о размерах ключей, времени вычислений
и других параметрах. Ещё одна важная область — криптоанализ PQC-алгоритмов.

3. \textit{Соображения реализации.}  Реализация PQC-алгоритмов окажется
сложнее, чем кажется. Сложность математических конструкций переносится на
архитектуру конкретных платформ и устройств. Существуют несколько методов
внедрения новых алгоритмов:

1) Один из наиболее исследованных методов — гибридный подход, при котором
используются два криптографических алгоритма: один из текущих стандартов и один
из кандидатов PQC. Это позволяет защититься от атак типа «записать сейчас —
расшифровать позже», сохраняя при этом устойчивость к известным атакам в ранних
этапах перехода. Преимущество метода — возможность сохранять сертификаты
соответствия во время перехода. Недостаток — значительное увеличение требований
к вычислениям, памяти и коммуникациям.

2) Другой подход — использование механизма согласования наборов шифров (cipher
suite negotiation), применяемого, например, в TLS \cite{RFC8446TLS13}. Во время
протокола рукопожатия стороны обмениваются списками поддерживаемых наборов шифров и
выбирают наиболее надёжный, который поддерживают обе стороны. Можно
согласовывать версии наборов шифров, размеры ключей и параметры
\cite{RFC7696Agility}. В эту структуру можно добавить новые алгоритмы PQC и удалить
устаревшие.

3) \textit{Формальное моделирование.}  Формальное моделирование перехода в
криптографии — крайне востребованная область исследований. Независимо от
выбранной стратегии перехода, безопасность системы остаётся проблемой.
Формальные методы позволяют оценивать безопасность фундаментальным образом.
Такое моделирование также необходимо для оценки безопасности внедрения
механизмов перехода в широко используемые криптографические протоколы.
Например, протокол TLS может быть декомпозирован на компоненты, устойчивые к
квантовым атакам, для исследовательских целей.

4) \textit{Автоматизированные инструменты.}  Из-за масштабов криптографической
инфраструктуры переход невозможен без автоматизированных средств. Поэтому
исследования и разработки в этой области также крайне необходимы.

\subsection{Расширение области криптографической гибкости}

Криптографическая гибкость является устоявшейся концепцией в исследовательском
сообществе, однако масштабность миграции к постквантовой криптографии делает
необходимым расширение этой области исследований. Следует разработать науку
криптографической гибкости, которая охватывала бы более широкий спектр целей,
вычислительных доменов, форм гибкости и временных масштабов. Для обеспечения
криптографической гибкости необходимо разрабатывать и исследовать
соответствующие архитектуры и фреймворки в широком диапазоне вычислительных
контекстов, а также формировать согласованные интерфейсы, удовлетворяющие
потребностям различных категорий пользователей.

Существуют и другие важные вызовы, такие как внедрение гибкости после
определения корректной области внутри криптографического решения. Система
криптографической гибкости должна уметь предсказывать области, в которых
потребуется изменение криптографических примитивов, а также обеспечивать
механизмы для своевременного и безопасного переключения между ними.
