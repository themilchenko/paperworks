\section{Заключение}

Квантовая криптография является зарождающейся и стремительно
развивающейся областью. Многие компании по всему миру вкладывают
ресурсы в увеличение знаний и развитие практик, связанных с
постквантовой безопасностью. В связи с разнообразием интересов и
исследований по всему миру возникла необходимость систематизировать
текущее внимание к квантовой безопасности и представить обзор
современных достижений в этой области.

Симметричные алгоритмы являются устойчивыми как к классическим, так и к
квантовым атакам (AES-256 используется для характеристики наивысшего
уровня безопасности всех новых алгоритмов), однако их трудно
реализовывать в квантовых схемах, особенно учитывая, что существующие
квантовые машины способны обрабатывать лишь очень малые размеры
сообщений (порядка 20 бит). Дальнейшие достижения в области квантовых
технологий могут расширить эти возможности, что позволит эффективнее
реализовывать симметричные криптосистемы, такие как AES.

Для симметричных криптосистем квантовые атаки требуют квантового
оракула. Пока симметричная криптография не реализована с использованием
квантовых оракулов, она остаётся безопасной против квантовых атак. Вся
современная классическая информация остаётся защищённой. Однако влияние
квантовых вычислений на асимметричные криптосистемы куда серьёзнее. Для
их взлома не требуется квантовая реализация самих алгоритмов —
злоумышленнику достаточно локальных квантовых ресурсов, чтобы
эксплуатировать уязвимости и взламывать шифрование. Это означает, что
все данные, зашифрованные асимметричными алгоритмами, становятся
уязвимыми при появлении достаточно мощных квантовых компьютеров.

На данный момент квантовые алгоритмы уже существуют для всех крупных
асимметричных криптосистем, и вопрос состоит лишь в том, когда они
будут полностью сломаны. Исследователи пытаются либо повысить сложность
задач, на которых основаны современные алгоритмы (RSA, ECC), либо найти
новые задачи, достаточно трудные даже для квантового компьютера. Второй
подход получил наибольшее развитие, и активные исследования ведутся в
областях решётчатых задач, кодов с исправлением ошибок, некоммутативных
криптосистем и криптосистем на основе хеширования. Однако многие
предлагаемые алгоритмы сложно реализовать, и их производительность
требует оптимизации для широкого практического применения.

В 2017 году NIST объявил международный конкурс будущего стандарта для асимметричной криптографии. Было
отмечено, что необходимость определения стандарта становится всё более
острой, поскольку время разработки и внедрения алгоритмов не должно
превышать время, за которое появятся системы, способные взломать
используемые сегодня криптосистемы. Конкурс NIST по постквантовой
криптографии включал три раунда, причём все материалы были публичными.
Это позволило сообществу выявить множество уязвимых алгоритмов, и лишь
15 из первоначальных 69 дошли до финального раунда.

В настоящем обзоре были рассмотрены эти 15 алгоритмов. NIST объявил
приём статей для третьей конференции по стандартизации PQC, которая
состоится в июне 2021 года. На конференции планируется обсуждение
различных аспектов алгоритмов и получение ценной обратной связи,
которая поможет принять решения о стандартизации.
