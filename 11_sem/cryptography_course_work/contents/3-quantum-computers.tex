\section{Квантовые компьютеры}

Квантовая механика известна своей парадоксальностью и тем, что она
противоречит интуитивным представлениям. Это связано с такими явлениями,
как суперпозиция, квантовая запутанность и квантовая неопределённость.
Суперпозиция позволяет частице находиться сразу в нескольких состояниях.
Запутанность описывает корреляции между частицами, которые невозможны
в классическом мире. Принцип квантовой неопределённости утверждает,
что наблюдение одной характеристики частицы приводит к потере информации
о другой~\cite{Giles2019QuantumComputing}.

Базовой единицей информации в классических вычислениях является бит~\cite{Feldman2003QM},
который может принимать два дискретных значения — 0 и 1. В квантовой
информатике элементарной единицей является кубит~\cite{Yanofsky2007}.
Кубит — это нормированный вектор в двумерном комплексном пространстве.
Его состояние записывается в виде
$\rangle\psi\rangle = \alpha|0\rangle + \beta|1\rangle$,
где $|0\rangle$ и $|1\rangle$ — базисные векторы, а $\alpha$ и $\beta$ — комплексные
амплитуды, удовлетворяющие условию $|\alpha|^{2} + |\beta|^{2} = 1$.
Для описания систем из двух и более кубитов используется тензорное произведение.

Набор квантовых вентилей, объединённых в схему, образует квантовый
алгоритм. К распространённым квантовым вентилям относятся управляемый
NOT (CNOT), преобразование Адамара, фазовые и инверсионные вентили.

В квантовых вычислениях измерение преобразует кубит в классический бит
(0 или 1). Если состояние кубита равно
$\rangle\psi\rangle = \alpha|0\rangle + \beta|1\rangle$,
то вероятность получения 0 равна $|\alpha|^{2}$, а вероятность получить 1 —
$|\beta|^{2}$. После измерения суперпозиция исчезает, и кубит переходит в
соответствующее базисное состояние.

Теорема о запрете клонирования~\cite{Kuzyk2019} утверждает, что невозможно создать
точную копию произвольного квантового состояния. Ни одна квантовая схема
не может принять на вход кубит и вывести исходный кубит вместе с его точной
копией. 

Квантовая запутанность — это явление, при котором система из двух или
более кубитов описывается только совместным состоянием, а не отдельно для
каждого кубита. Например, состояние
$(|00\rangle + |11\rangle)/\sqrt{2}$
является состоянием Белла, и состояния отдельных кубитов не могут быть
определены независимо. Измерение одного кубита сразу определяет состояние
второго.

Существует множество физических платформ для реализации кубитов, включая
сверхпроводниковые кубиты, фотонные кубиты (основанные на поляризации
света) и другие. Фотонные кубиты удобны для дальних квантовых коммуникаций,
тогда как сверхпроводниковые платформы лучше подходят для реализации
квантовых процессоров благодаря сильным взаимодействиям между кубитами.

\subsection{Распределение квантового ключа (протокол BB84)}

Протокол BB84 представляет собой защищённый метод распределения общего
ключа между двумя участниками. Для его работы требуется два канала:
квантовый канал и классический канал. Классический канал является
двунаправленным, тогда как квантовый может быть односторонним (только
от отправителя к получателю, без обратной передачи).

Бит и базис, используемые для кодирования и отправки кубита, выбираются
случайным образом. В протоколе используются два различных набора
ортогональных базисов~\cite{Nurhadi2018QKD} (например, горизонтальная/вертикальная
поляризация фотонов: $0^{\circ}, 90^{\circ}$, и диагональная поляризация:
$45^{\circ}, -45^{\circ}$). Алиса и Боб заранее договариваются о том, каким
образом кодируются биты 0 и 1 в каждом из двух базисов. Важно отметить,
что измерение состояния фотона в одном базисе уничтожает информацию,
соответствующую другому базису.

Алиса генерирует случайную последовательность битов и кодирует каждый бит
в случайно выбранном базисе. Сформированные кубиты передаются Бобу через
квантовый канал. Боб, не зная выбор Алисы, измеряет каждый полученный кубит
с помощью случайно выбранного базиса (диагонального или вертикального/
горизонтального).

После передачи квантовой последовательности Алиса и Боб переходят
к классическому каналу, по которому они объявляют, какие базисы были
использованы для каждого бита (но не сами значения битов). Все позиции,
где их базисы различаются, отбрасываются. Таким образом формируется
финальный общий ключ, известный только Алисе и Бобу.

Защищённость протокола обеспечивается квантовым принципом невозможности
клонирования: Ева не может создать копию кубитов. Если она попытается
измерять кубиты, выбранный ею неправильный базис приведёт к уничтожению
части информации. Когда Боб верифицирует несколько случайных битов
с Алисой через классический канал, любое вмешательство будет обнаружено.

Если Ева попытается перехватить все кубиты и измерять их в случайных
базисах, она неизбежно выберет неверные базисы для части из них, что
приведёт к потере информации. Из-за этого она не сможет повторно
подготовить правильные кубиты для Боба — атака будет выявлена.

\subsection{Квантовые алгоритмы}

\subsubsection{Алгоритм Шора}

Алгоритм Шора был разработан для решения двух фундаментальных задач,
обеспечивающих стойкость современных криптосистем: задачи разложения
числа на простые множители и задачи вычисления дискретного логарифма.
Для классических вычислений обе задачи имеют решения субэкспоненциальной
сложности. Используя квантовый алгоритм поиска периода, основанный на
квантовом преобразовании Фурье (QFT), Шор предложил вероятностные
алгоритмы, работающие за полиномиальное время. Алгоритм Шора способен
разложить число на множители за полиномиальное время, тогда как
лучшая классическая процедура — General Number Field Sieve — требует
субэкспоненциального времени~\cite{Shor1997}.

Оба алгоритма Шора (для факторизации и для дискретного логарифма)
состоят из двух основных частей. Первая часть — классическая: она
сводит исходную задачу к задаче поиска периода определённой функции.
Вторая часть — квантовая — использует квантовый параллелизм и
квантовое преобразование Фурье для нахождения периода.

Алгоритм Шора является вероятностным: он не всегда находит делитель
числа с первой попытки. Однако при многократном запуске вероятность
успеха стремительно возрастает, и факторизация становится практически
гарантированной.

\subsubsection{Алгоритм Гровера}

Алгоритм Гровера — это квантовый алгоритм поиска, позволяющий
выполнять поиск в неупорядоченной базе данных за $O(\sqrt{n})$~\cite{Grover1996}.
В классической модели вычислений такой поиск требует $O(n)$ операций.
Алгоритм Гровера также является вероятностным: его можно запускать
несколько раз, увеличивая вероятность нахождения нужного элемента.

Работа алгоритма основана на повторении итераций Гровера
(amplitude amplification). Количество итераций можно увеличивать,
чтобы повысить вероятность успешного обнаружения элемента.

Алгоритм Гровера может применяться не только для поиска в базе данных:
он используется для нахождения среднего и медианы выборки, для
вычисления обратных значений функций и других задач. Эти возможности
делают его полезным инструментом для квантового криптоанализа.
В частности, алгоритм Гровера может использоваться для ускоренного
перебора ключей в симметричных криптосистемах, уменьшая эффективную
стойкость алгоритмов, таких как AES.

\subsubsection{Алгоритм Саймона}

Алгоритм Саймона представляет собой квантовую схему, позволяющую
определить, является ли функция взаимно-однозначной (1--1) либо
двухкратно-однозначной (2--1). Свойство такой функции формулируется
следующим образом: если два различных входных значения отображаются в
один и тот же выход, то их XOR равен некоторому постоянному вектору
$b$. Если $b$ состоит из всех нулей, функция является 1--1; в противном
случае она является 2--1.

Классический алгоритм для поиска такого совпадения требует
$O(2^{n-1} + 1)$ вычислений. Квантовый алгоритм Саймона обеспечивает
экспоненциальное ускорение по сравнению с классическим подходом~\cite{Hui2019Simons}.

Алгоритм Саймона стал источником идей для последующего создания
алгоритма Шора. Сам по себе он также имеет криптографические приложения,
при условии, что доступен корректный оракульный (query) интерфейс к
функции.

\subsection{Современные достижения в области квантовых компьютеров}

Теоретические основы квантовых вычислений были заложены Ричардом
Фейнманом в 1982 году. Бернстайн и Вазиряни~\cite{BernsteinVazirani1993}
доказали, что квантовые компьютеры способны нарушать расширенную
гипотезу Чёрча–Тьюринга. Они представили алгоритм рекурсивного
преобразования Фурье, показавший возможности квантовых схем. Позднее,
в 1994 году, Питер Шор~\cite{Shor1997} предложил алгоритмы, способные решать
отдельные вычислительные задачи существенно быстрее классических
подходов. Несмотря на эти результаты, квантовые вычисления долгое время
оставались преимущественно теоретической областью.

Существует два основных типа квантовых компьютеров: универсальные
квантовые компьютеры и квантовые отжигатели (quantum annealers).
Универсальные квантовые компьютеры опираются на кубиты и квантовые
логические вентили, позволяющие выполнять широкий спектр вычислений
через квантовое программирование. Квантовые отжигатели не используют
логические вентили и предназначены главным образом для решения задач
оптимизации.

Компании, такие как D-Wave, разрабатывают квантовые отжигатели.
Эти устройства более устойчивы к шуму и содержат порядка 4000 кубитов.
D-Wave предлагает свои системы коммерчески компаниям, решающим
оптимизационные задачи. Квантовые отжигатели используют свойства
кубитов для поиска состояния с минимальной энергией, которое
соответствует оптимальному решению. В отличие от классических
компьютеров, перебирающих варианты последовательно, квантовые машины
могут исследовать множество комбинаций одновременно. Ряд отраслей
уже применяет системы D-Wave~\cite{DWAVEPublications} для задач планирования и получает
быстрые оптимальные решения. Существенным недостатком квантовых
отжигателей является невозможность выполнять универсальные вычисления:
задачу необходимо предварительно свести к формулировке оптимизации.
В криптографии основное внимание уделяется именно универсальным
квантовым компьютерам.

Исследования и разработка универсальных квантовых компьютеров также
активно развиваются. Закон Мура~\cite{CBInsightsQCReport} постепенно перестаёт
действовать, так как размеры транзисторов стремятся к физическому
минимуму, и квантовые эффекты начинают мешать их корректной работе.
Квантовые вычисления рассматриваются как возможный путь дальнейшего
роста вычислительных мощностей при физических ограничениях
классических технологий.

Количество стартапов, занимающихся квантовыми вычислениями, стремительно
растёт~\cite{CBInsightsQCReport}. Компания PsiQuantum, поддерживаемая Microsoft, объявила
о разработке оптического квантового процессора на один миллион кубитов,
способного произвести революцию в вычислениях. Многие стартапы работают
над созданием квантового аппаратного и программного обеспечения.

Крупнейшие технологические компании также активно инвестируют в
квантовые технологии. Google и IBM соревнуются в достижении квантового
превосходства. Они создают низкошумные квантовые процессоры. В 2019 году
Google представила квантовый чип из 53 кубитов, продемонстрировавший
вычисление, недоступное классическому суперкомпьютеру. IBM также
активно развивает квантовое направление и уже создала несколько машин,
планируя построить систему на 1000 кубитов к 2023 году~\cite{IBMQuantumRoadmap2020}.
Amazon и Microsoft сотрудничают со стартапами для интеграции квантовых
вычислений в AWS и Azure.

Квантовые вычисления приведут к парадигмальному сдвигу в IT-индустрии.
Исследователям предстоит решить проблемы коррекции ошибок и повышения
стабильности квантовых устройств. Квалифицированных специалистов в
области крайне мало, но в ближайшие десятилетия квантовые вычисления
могут радикально преобразить промышленность благодаря новым вычислительным
возможностям. Квантовые компьютеры создаются для решения задач, которые
классически не поддаются решению за полиномиальное время. Первым
практическим применением станет запуск алгоритма Шора, что сделает
асимметричные криптосистемы уязвимыми. Поэтому необходимо разрабатывать
новые криптографические задачи, устойчивые к квантовым атакам. Это
привело к развитию постквантовой криптографии и запуску процесса
стандартизации NIST в этой области.
