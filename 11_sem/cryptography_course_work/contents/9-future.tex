\section{Перспективы и будущие вызовы}

Процесс стандартизации NIST PQC проведёт свою следующую встречу в июне
2021 года. На ней будут представлены новые комментарии по всем
финалистам и альтернативным алгоритмам. Постквантовая криптография —
это новая область исследований, которая в ближайшие годы станет
повсеместной.

Существует шесть основных семейств, на основе которых разрабатываются
криптосистемы. Все они имеют свои преимущества и недостатки с точки
зрения битовой безопасности, размеров ключей, накладных расходов по
памяти и вычислительных требований. Исследователи могут находить новые
математические задачи, трудные для квантовых компьютеров, и создавать
на их основе новые криптосистемы. Также возможно выявление новых задач,
которые требуют меньших накладных расходов и меньших вычислительных
ресурсов.

Исследователи активно работают над предложенными NIST криптосистемами.
В течение долгого времени они проводят криптоанализ алгоритмов, изучают
проблемы реализации, математические атаки, атаки по побочным каналам и
так далее. Это важнейшая область исследований, и криптоанализ
финалистов и альтернатив позволит повысить безопасность алгоритмов и
предотвратить стандартизацию потенциально небезопасных решений.

Существующие алгоритмы также могут быть оптимизированы за счёт новых
аппаратных и программных решений. Это может помочь уменьшить временные
и пространственные накладные расходы алгоритмов. Исследователи могут
разрабатывать методы снижения вычислительных требований, создавая
специализированное оборудование или улучшая программные реализации.
Также возможно уменьшить накладные расходы по памяти.

Перспективным направлением является разработка лёгкой постквантовой
криптографии. Рост числа IoT-устройств делает лёгкие криптографические
алгоритмы особенно актуальными. Такие устройства обмениваются
зашифрованными данными, так как часть информации может быть
конфиденциальной (например, коммерческие тайны). Поэтому требуется
постквантовая лёгкая криптография — алгоритмы, устойчивые к квантовым
атакам и требующие минимальных ресурсов.

Числовые методы и алгоритмические исследования — ещё одно направление,
связанное с постквантовыми сложными задачами. Учёные могут работать над
новыми математическими техниками взлома задач, на которых основана
безопасность постквантовых криптосистем. Значительное количество
исследований проводится по многомерным квадратичным отображениям и
решётчатым задачам. Исследователи предлагают новые методы решения этих
трудных задач.

Квантовый криптоанализ и разработка новых квантовых алгоритмов — ещё
одна область, исследованная пока очень слабо. Шор предложил два
квантовых алгоритма, которые нарушили безопасность современных
асимметричных алгоритмов. Возможно, для постквантовых криптосистем
также будут обнаружены новые квантовые алгоритмы, способные
скомпрометировать их безопасность. Поэтому необходимо проводить
дальнейшие исследования в этом направлении.
