\section{Сравнение производительности постквантовых криптосистем}

Таблица~1 показывает уровни безопасности, обеспечиваемые постквантовыми
криптосистемами, рассмотренными в данном исследовании. Производительность
алгоритма обычно измеряется числом тактов, необходимых для выполнения, а
также объёмом аппаратных ресурсов (логики/площади), требуемых для его
аппаратной реализации. На производительность также влияет длина ключа или
любой другой внутренний параметр, используемый алгоритмом. По этой причине
многие схемы имеют несколько вариантов с различными параметрами, что
существенно влияет на их производительность.

\begin{table}[h]
\centering
\caption{Семейства и уровни безопасности PQC-алгоритмов}
\begin{tabularx}{\textwidth}{|X|X|X|}
\hline
\textbf{Алгоритм} & \textbf{Семейство алгоритмов} & \textbf{Уровень безопасности} \\ \hline
Classic McEliece   & Кодовые               & 5 \cite{Basu2019HardwareEval} \\ \hline
SABER              & Решётки               & 1, 3, 5 \cite{Basu2019HardwareEval} \\ \hline
CRYSTALS-Kyber     & Решётки               & 1, 3, 5 \cite{Basu2019HardwareEval} \\ \hline
NTRU-HRSS          & Решётки               & 1 \cite{Basu2019HardwareEval} \\ \hline
NTRU-HPS           & Решётки               & 1, 3, 5 \cite{Dang2020Benchmarking} \\ \hline
CRYSTALS-Dilithium & Решётки               & 1, 2, 3 \cite{Basu2019HardwareEval} \\ \hline
SIKE               & Изогении              & 1, 2, 3, 5 \cite{Dang2020Benchmarking} \\ \hline
SPHINCS+           & Хеш-подписи           & 1, 3, 5 \cite{Basu2019HardwareEval} \\ \hline
\end{tabularx}
\end{table}

Для примера, kyber512, kyber768 и kyber1024 по сути являются одним и тем же
алгоритмом, но используют разную длину ключа и другие параметры, что влияет на
производительность и уровень безопасности шифра. Чтобы сравнение было
корректным, сравнивать нужно алгоритмы с одинаковым уровнем безопасности.

В анализе, проведённом Канадом Басу и др. [65], авторы установили, что среди
алгоритмов, использующих механизмы инкапсуляции ключей, NTRU-HRSS обладает
самой высокой латентностью (в тактах процессора). Среди алгоритмов электронных
подписей уровня 1 наименьшую латентность показывает CRYSTALS-Dilithium. Для
операций декапсуляции NTRU-HRSS (уровень 3) и Classic McEliece (уровень 5)
имеют наибольшую задержку, в то время как CRYSTALS-Dilithium демонстрирует
наименьшую, что отражено в таблице 2. Эти результаты получены при тестировании
алгоритмов на ПЛИС Virtex-7.

Из этих результатов можно сделать вывод, что при использовании базовых
алгоритмов без модификаций CRYSTALS-Dilithium является хорошим кандидатом для
IoT как для инкапсуляции, так и для декапсуляции. Однако для декапсуляции
практически применимы только алгоритмы уровня 1, поскольку все протестированные
алгоритмы уровня 5 обладают слишком большой задержкой. Поскольку IoT обычно
предполагает аппаратные реализации, тестирование на FPGA является хорошим
ориентиром для оценки производительности алгоритма на устройстве IoT. Техники
такие как loop unrolling и loop pipelining могут дополнительно снизить общую
задержку.

Брайан Хешион и др. [66] использовали расширение eXternal Benchmarking
eXtension (XXBX) для тестирования производительности PQC-алгоритмов на
32-битном ARM-микроконтроллере EK-TM4C123GXL. Алгоритмы оценивались по
потреблению RAM, ROM, скорости (тактовые циклы) и энергии.

По использованию RAM NTRU является самым дорогим алгоритмом, а наименьший объём
оперативной памяти потребляет Kyber (во всех трёх уровнях безопасности). По ROM
NTRU также самый тяжёлый, а следующим идёт Saber, хотя он использует лишь
половину объёма ROM по сравнению с NTRU. По скорости (числу тактов) SIKE
является самым медленным с большим отрывом — одна операция инкапсуляции
занимает семь минут. Kyber, Saber и NTRU имеют сопоставимую производительность,
причём уровни безопасности выше требуют немного больше времени, что показано в
таблице 2. Та же картина наблюдается и в энергопотреблении, поскольку более
длительное выполнение алгоритма означает и более длительное потребление
мощности.
