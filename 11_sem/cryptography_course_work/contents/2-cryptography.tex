\section{Криптография}

Криптография — это наука, изучающая методы сокрытия сообщений с использованием
математических средств. Содержание сообщения скрывается посредством
криптографического алгоритма шифрования, который преобразует исходные данные
таким образом, чтобы сделать их непонятными для постороннего наблюдателя.
Полученные скрытые данные называются шифртестом. Исходное сообщение называют
открытым текстом или plaintext. Шифртест может быть вновь преобразован в
исходное сообщение с помощью алгоритма расшифрования. Конкретная реализация
криптографического метода называется криптосистемой. Криптоанализ — это наука,
изучающая методы анализа и взлома криптосистем, которые считаются защищёнными.

Криптография предоставляет пользователям широкий набор сервисов информационной
безопасности. Во-первых, она обеспечивает конфиденциальность данных: шифрование
делает информацию недоступной для всех, кроме обладателей секретного ключа
(предполагаемых получателей), которые могут восстановить исходный текст.
Во-вторых, она обеспечивает целостность данных: криптографические механизмы
позволяют удостовериться, что информация не была изменена третьей стороной.
В-третьих, криптография предоставляет средства аутентификации — отправитель и
получатель могут подтвердить личности друг друга и убедиться, что не
взаимодействуют со злоумышленником. Кроме того, она позволяет доказать
подлинность личности субъекта. Наконец, криптография обеспечивает невозможность
отказа от авторства (non-repudiation), то есть предоставляет механизм
подтверждения того, что сообщение действительно было отправлено конкретным
участником.

Криптография имеет множество прикладных применений \cite{CryptoEveryday2021}.
Жёсткие диски компьютеров содержат большое количество деловой, военной и личной
конфиденциальной информации, утечка которой может привести к серьёзным
последствиям. Криптографические методы используются на устройствах хранения
данных для обеспечения их защищённости даже при физическом доступе
злоумышленника к носителю.

Широко применяется криптография и в компьютерных сетях. Она служит для
аутентификации, создания цифровых подписей пользователей и отметки времени
важных документов. Цифровые платёжные системы — такие как Google Pay и Paytm —
основываются на криптографических механизмах для защиты банковских данных
пользователей.

Шифрование и расшифрование используются в электронной почте, в системах
мгновенного обмена сообщениями, таких как WhatsApp, а также в социальных сетях
— например, Instagram и Facebook.

\subsection{Симметричная криптография}

Симметричная криптография -- это раздел криптографии, в рамках которого и
шифрование, и расшифрование выполняются с использованием одного и того же
ключа. При условии, что длина ключа достаточно велика, а сам ключ является
случайным, перебор всех возможных ключей выходит за пределы возможностей
современных вычислительных систем.

Однако из-за того, что вся безопасность схемы основана на одном секретном
ключе, критически важным аспектом симметричной криптографии является безопасная
передача ключа от отправителя к получателю.

\subsubsection{Data Encryption Standard (DES)}

Data Encryption Standard (DES) — один из наиболее известных криптографических
алгоритмов, уязвимость которого к полному перебору ключа была доказана ещё в
доквантовую эпоху. Тем не менее увеличение длины ключа позволяет частично
компенсировать эту уязвимость, что привело к появлению модифицированных схем,
таких как Triple-DES (3DES), который может работать с двумя или тремя ключами,
экспоненциально увеличивая количество операций, необходимых для полного
перебора.

DES основан на композиции преобразований Фейстеля, в которой сообщение делится
на две половины, проходящие чередующиеся преобразования в каждом раунде.
\cite{PaarPelzl2010}. Каждый блок сообщения имеет длину 64 бита, при это в
каждом раунде обрабатывается 32 бита. В следующем раунде необработанные биты
меняются местами с обработанными, и процесс повторяется.

Алгоритм DES включает 16 таких раундов, которым предшествует начальная
перестановка (Initial Permutation, IP) и завершает всё конечная перестановка
(Final Permutation, FP) — обратная к IP.

Фейстелева функция DES включает следующие операции:

\begin{enumerate} \item \textbf{Расширение.} Входной \(32\)-битовый
  блок преобразуется в \(48\)-битовый путём дублирования и добавления первых и
  последних бит соседних \(4\)-битовых блоков к текущему блоку.

  \item \textbf{Смешивание с ключом.} Каждый раунд использует
    собственный \(48\)-битовый подключ, порождаемый из исходного ключа с
    помощью алгоритма выработки подключей. Результат операции расширения
    складывается по модулю
    два (XOR) с подключом.

  \item \textbf{Подстановка.} Полученный \(48\)-битовый блок
    разбивается на восемь \(6\)-битовых блоков, каждый из которых проходит
    через нелинейные преобразования в \(S\)-блоках. Таблицы \(S\)-блоков
    определены стандартом NIST. Каждый \(S\)-блок отображает \(6\) бит во входе
    в \(4\) бита на выходе, формируя итоговый \(32\)-битовый блок для
    следующего раунда.

  \item \textbf{Перестановка.} Выходы \(S\)-блоков переставляются
    согласно \(P\)-таблице, что обеспечивает распределение выходов каждого
    \(S\)-блока по четырём различным \(S\)-блокам в следующем раунде.
\end{enumerate}

Алгоритм DES обеспечивает необходимую степень рассеивания и перемешивания
благодаря \(S\)-
и \(P\)-блокам, что делает выходной шифртест псевдослучайным. В таких
условиях единственным практическим способом взлома является полный перебор
ключа.

На сегодняшний день DES считается небезопасным. Алгоритм был впервые взломан в
1997 году в рамках проекта DESCHALL, а рост вычислительных мощностей привёл к
тому, что время перебора стало практически ничтожным. В 2017 году был
продемонстрирован наиболее быстрый взлом DES в модели выбранного открытого
текста с использованием радужных таблиц, позволивший получить ключ менее чем за
\(25\) секунд.

Для увеличения эффективной длины ключа и усложнения перебора была разработана
усиленная схема --- алгоритм 3DES.

\subsubsection{3-DES}

Тройной DES (3DES) был предложен как решение
уязвимости DES, связанной с малой длиной ключа
(56 бит), путём «расширения» эффективного размера
ключа за счёт использования трёх различных независимых
ключей \(K_{1}\), \(K_{2}\), \(K_{3}\), каждый длиной 56 бит.

Шифрование открытого текста выполняется по схеме
«Шифрование–Расшифрование–Шифрование» (EDE).
Сначала открытый текст шифруется DES с ключом
\(K_{1}\). Затем промежуточный результат расшифровывается
DES с ключом \(K_{2}\). Наконец, полученный текст снова
шифруется DES с ключом \(K_{3}\), формируя шифртест
\cite{PaarPelzl2010}.

В зависимости от того, как выбираются ключи, можно
увеличить сложность атаки полного перебора. Если
использовать три независимых значения \(K_{1}, K_{2}, K_{3}\),
то достигается максимальная стойкость 3DES, эквивалентная
эффективной длине ключа 168 бит.

Тем не менее и эта схема оказалась уязвимой из-за
малого размера блока, что увеличивает вероятность
коллизий. Эта слабость была использована в атаке
\textit{Sweet32}~\cite{BhargavanLeurent2016}, в рамках которой
коллизии могут быть найдены примерно за
\(2^{36*2}\) попыток.

\subsubsection{Advanced Encryption Standard (AES)}

AES~\cite{NIST2001AES} является одним из наиболее широко применяемых
симметричных криптографических алгоритмов. Его стойкость основана
на псевдослучайности шифртеста, формируемого в результате
многократного применения стандартизованной сети
подстановки–перестановки (SP-сети). На практике используются версии
AES с длинами ключа \(128\), \(192\) и \(256\) бит, которым
соответствуют \(10\), \(12\) и \(14\) раундов, каждый из которых состоит
из четырёх преобразований. Подключи генерируются с помощью алгоритма
вырботки подключей. Необходимые уровни запутывания и рассеивания достигаются
многократными итерациями подстановок и смешивания.

Каждый раунд AES включает четыре преобразования:

\begin{enumerate}
    \item \textbf{SubByte.}
    Каждый байт состояния заменяется значением из
    заранее определённого нелинейного соответствия.
    Подстановка осуществляется через фиксированный
    \(S\)-блок, содержащий \(256\) элементов.

    \item \textbf{ShiftRows.}
    Строки состояния циклически сдвигаются согласно
    правилам, указанным в стандарте NIST. При
    необходимости допускаются модифицированные
    варианты сдвига.

    \item \textbf{MixColumns.}
    Каждая колонка интерпретируется как полином и
    умножается на кодирующий полином с последующим
    приведением результата по модулю неприводимого
    полинома над полем Галуа \( \mathbb{GF}(2^8) \).

    \item \textbf{AddRoundKey.}
    К массиву состояния применяется операция XOR
    с раундовым подключом, порождённым по расписанию
    ключей.
\end{enumerate}

Все операции выполняются над полем Галуа
\( \mathbb{GF}(2^8) \). Алгоритм расшифрования повторяет те же шаги,
что и шифрование, но в обратном порядке, используя инверсные
операции.

Хотя выявлены уязвимости, связанные с повторяющимися структурами
открытого текста или некорректным выбором ключа, доказано, что эти
атаки не уменьшают сложность до уровня, позволяющего осуществить
полный взлом. С учётом современных вычислительных возможностей
AES считается стойким. На сегодняшний день AES является наиболее
широко применяемым симметричным шифром в проводных и беспроводных
протоколах безопасности.

\subsection{Асимметричная криптография}

Асимметричная криптография — это направление криптографии, в котором ключ
разделён на две части: открытый ключ и закрытый ключ. Открытый ключ
передаётся отправителю, который использует его для шифрования данных,
тогда как закрытый ключ применяется получателем для расшифрования.
Такой подход обеспечивает конфиденциальность и подлинность сообщений.

\subsubsection{RSA}

RSA — это публичная криптосистема, широко используемая для безопасной
передачи данных. Она была предложена Рональдом Райвестом (Rivest), Ади
Шамиром (Shamir) и Леонардом Адлеманом (Adleman) в 1977 году.

Алгоритм основан на задаче факторизации целого числа. Наиболее
эффективный классический алгоритм факторизации — General Number Field
Sieve — имеет субэкспоненциальную сложность. RSA использует этот факт, что
делает алгоритм вычислительно стойким.

Алгоритм опирается на однонаправленные функции: их легко вычислять, но
трудно обращать (например, перемножить два простых числа легко, а вот
разложить произведение на множители — сложно). Теорема о распределении
простых чисел и оценка Чебышёва гарантируют, что среди чисел длиной
200 десятичных цифр приблизительно \(1/1000\) являются простыми.
Поэтому пользователи RSA могут сравнительно просто находить большие
простые числа. Для проверки простоты используется тест Ферма, который
даёт вероятностный результат.

Тест простоты Ферма утверждает, что если число \(n\) простое, то для
случайного \(a\) выполняется равенство \(a^{\,n-1} \equiv 1 \pmod{n}\). При
повторении теста для большого количества различных значений \(a\) и
получении результата \(1\) вероятность того, что \(n\) является простым,
возрастает. Если число составное, тест Ферма проваливается как минимум
для половины значений из \(\mathbb{Z}_n\). Поэтому каждая успешная
проверка уменьшает вероятность ошибки примерно вдвое.

Для организации защищённого обмена данными получатель генерирует пару
ключей. Он выбирает два больших простых числа \(p, q\) и вычисляет
\(n = p \cdot q\). Затем вычисляется \(L = (p-1)(q-1)\). Выбирается число
\(e\), взаимно простое с \(L\). Пара \((e, n)\) публикуется как открытый
ключ. Закрытый ключ определяется как \(d = e^{-1} \pmod{(p-1)(q-1)}\).

Зашифрование выполняется отправителем по формуле \(C = m^{e} \bmod n\),
а расшифрование — получателем по формуле \(m = C^{d} \bmod n\).

Для больших размеров ключей RSA (1024 бита и выше) не существует
эффективных методов решения задачи факторизации. Все практические
атаки на RSA основаны на попытке разложения \(n\) на простые множители
\cite{SANSVPN2020}. Из-за субэкспоненциальной сложности факторизации
взлом RSA при длине ключа 2048 бит считается нереалистичным при
современном уровне вычислительной мощности. По этой причине RSA
широко используется в разных протоколах и системах, требующих
асимметричной криптографии.

\subsubsection{Протокол обмена ключами Диффи~-- Хеллмана}

Протокол обмена ключами Диффи~-- Хеллмана (DHKE) основан на задаче
дискретного логарифма. Эта задача представляет собой одностороннюю
функцию: прямое вычисление легко, а обратное чрезвычайно сложно.
Для дискретного логарифма выбирается большое простое число \(N\).
В группе \(\mathbb{Z}_N\) определяется генератор \(g\), степени которого
пробегают через все элементы группы \cite{PaarPelzl2010DL}.

Возведение \(g\) в степень \(p\) в \(\mathbb{Z}_N\) вычисляется легко, однако
найти \(p\), зная \(N\), \(g\) и значение \(o\), крайне трудно. Математически:
\(o = g^{p} \bmod N\). Зная \(g\), \(p\) и \(N\), легко вычислить \(o\), но зная
лишь \(o\), \(g\) и \(N\), трудно определить \(p = \mathrm{dlog}_g(o)\). Здесь
\(\mathrm{dlog}\) означает дискретный логарифм.

Решение задачи дискретного логарифма также имеет субэкспоненциальную
сложность. Поскольку безопасность DHKE опирается на трудность этой
задачи, взлом данного протокола требует экспоненциальных ресурсов.

Протокол использует два общедоступных параметра: простое число \(p\) и
генератор \(q\) группы \(\mathbb{Z}_p\). Стороны выбирают секретные случайные
значения \(a\) и \(b\) и вычисляют открытые ключи:
\(A = q^{a} \bmod p\) и \(B = q^{b} \bmod p\).

Значения \(A\) и \(B\) публикуются. Каждая сторона затем вычисляет общий
ключ: \(X = B^{a} \bmod p\) и \(X = A^{b} \bmod p\). Оба выражения дают одно и
то же значение \(X\), которое используется далее в качестве секретного ключа в симметричных
криптографических алгоритмах.

Протокол DHKE не обеспечивает аутентификацию. Поэтому возможна атака
«человек посередине» (Man-in-the-middle), при которой злоумышленник устанавливает два
разных общих ключа — с Алисой и с Бобом — выдавая себя за другую
сторону, перехватывая, изменяя и пересылая сообщения.

При неправильном выборе параметров возможна атака с использованием
малых подгрупп. Существуют и другие виды
атак, использующие особенности выбора простых чисел. Тем не менее при
корректной модернизации инфраструктуры протокол Диффи~-- Хеллмана
остаётся столь же стойким, как задача вычисления дискретного
логарифма.

\subsubsection{Криптография на эллиптических кривых}

Криптография на эллиптических кривых (ECC) основана на свойствах
эллиптических кривых. Эллиптическая кривая обладает свойством, что любая
прямая пересекает её не более чем в трёх точках. Это свойство используется
для построения однонаправленной функции. Определяется операция \(A \bullet A\)
\cite{PaarPelzl2010ECC}. Если выполнить её \(n\) раз, получается значение \(Y\).
Вычислить \(Y\) просто, но определить число применений операции крайне
сложно. Если \(A \bullet A \bullet \dots\) (выполнено \(p\) раз) даёт \(A^{p} = O\),
то отсюда следует \(p = \mathrm{dlog}_{A}(O)\), что является сложной задачей
в группе точек эллиптической кривой.

ECC имеет максимальное количество точек, определяемое порядком группы.
Это приводит к эффекту «оборачивания» линий, аналогичному модульной
арифметике. Максимум точек зависит от размера ключа.

ECC получила значительный интерес в последние годы, поскольку требуемые
для таких систем длины ключей значительно меньше RSA. Например, эллиптическая кривая с
ключом \(256\) бит обеспечивает ту же стойкость, что и RSA-ключ \(3072\) бит.
ECC применяется в Bitcoin, Tor, WhatsApp~\cite{Wagner2020ECCIntro}. Её стойкость
основана на трудности решения дискретного логарифма в группе точек
эллиптических кривых. NIST стандартизовал множество кривых, рекомендуемых
для использования в ECC. При атаке только на основе шифртеста единственный
путь взлома ECC — найти алгоритм для нахождения дискретного логарифма.
На данный момент решений с реалистичным временем работы не существует.

ECC уязвима к атакам по сторонним каналам. Эти атаки основаны на
измерении физических параметров реализации: атаке по времени,
анализу потребляемой мощности, дифференциальной атаке по мощности
и анализу отказов. Атакующий может анализировать вариации, амплитуды и
форму пиков напряжения~\cite{ECCSideChannel2004}.

Другим видом атак являются twist-security атаки~\cite{ECCSideChannel2004}.
Они успешны при выполнении ряда условий и могут привести к раскрытию
закрытого ключа. В twist-атаке злоумышленник выбирает специальный
вариант публичного ключа и обменивается им с жертвой для вычисления
общего ключа, а затем использует его для восстановления закрытого ключа.
Тем не менее при корректном выборе кривых и параметров подобных атак
легко избежать. Отмечается, что некоторые стандартные кривые NIST
могут содержать потенциальные бэкдоры.
