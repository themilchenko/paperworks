\abstract

Появление полноценных крупномасштабных квантовых компьютеров
приведёт к серьёзным последствиям для современной криптографии. Большинство
симметричных и асимметричных криптографических алгоритмов уязвимы перед
квантовыми алгоритмами. Алгоритм поиска Гровера обеспечивает квадратичное
ускорение перебора ключа в симметричных схемах, таких как AES и 3DES.
Безопасность асимметричных алгоритмов — таких как RSA, Диффи–Хеллмана
(Diffie–Hellman) и эллиптические криптосистемы (ECC) — опирается на
вычислительную сложность задач факторизации и дискретного логарифмирования.
Лучшие известные классические алгоритмы решают эти задачи за экспоненциальное
время, тогда как алгоритм факторизации Шора способен решать их за
полиномиальное время. Существенные достижения в области квантовых вычислений
сделают все широко используемые сегодня асимметричные криптосистемы уязвимыми.
В данной работе анализируются классические криптосистемы с точки зрения их
стойкости к квантовым атакам, рассматриваются основные семейства постквантовых
криптосистем, обсуждается статус процесса стандартизации постквантовой
криптографии, проводимого NIST, а также предлагаются направления дальнейших
исследований в этой области.

Ключевые слова: постквантовая криптография, квантовые компьютеры, алгоритм Шора,
процесс стандартизации постквантовой криптографии NIST.
