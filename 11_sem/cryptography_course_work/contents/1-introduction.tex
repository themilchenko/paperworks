\section{Введение}

В современном мире огромную роль играют коммуникации. Сегодняшнее общество опирается на
Интернет как на базовый механизм любой дистанционной взаимосвязи. Это порождает
необходимость защищать передаваемые данные и обеспечивать их
конфиденциальность. Криптография — это область, полностью посвящённая защите
информации, где множество исследователей разрабатывают и анализируют алгоритмы,
позволяющие сохранять приватность и целостность данных.

Постепенно формируется новая парадигма вычислений — квантовые компьютеры,
которые способны радикально изменить возможности классических вычислительных
систем, используемых сегодня. Квантовый компьютер сможет выполнять отдельные
задачи, недоступные даже современным высокопроизводительным многопроцессорным
системам. Концепция квантовых вычислений основана на квантовой физике — на
понимании поведения систем на субатомном уровне, заложенном в начале XX века
такими учёными, как Шрёдингер (Schrödinger), Бор (Bohr), Гейзенберг
(Heisenberg), Эйнштейн (Einstein) и др. Позднее, в 1980-е годы, идеи и
математический аппарат квантовой физики были применены для моделирования
вычислительных устройств, способных выполнять определённые задачи значительно
быстрее классических компьютеров.
Квантовые компьютеры используют в качестве входных данных квантовые биты и
формируют результат, опираясь на принципы квантовой механики. Такие компьютеры
способны решать задачи, которые классические вычислительные системы не могут
выполнить за разумное время. Их потенциал распространяется на вычисления,
недостижимые даже для современных суперкомпьютеров. Области применения
чрезвычайно широки: навигационные системы, сейсмология, физические
исследования, фармацевтика и многое другое \cite{Humble2018}. Квантовые компьютеры позволяют
решать сложные научные задачи, которые остаются нерешёнными для самых мощных
суперкомпьютеров. Они станут революционным фактором и для искусственного
интеллекта благодаря огромному приросту вычислительной мощности. Таким образом,
потенциальные области применения квантовых вычислений практически безграничны.

Однако у любой технологии есть две стороны. С одной стороны, квантовые
компьютеры дают надежду быстро решать множество сложных задач из самых разных
областей. С другой — их появление может создать серьёзные угрозы. Один из
наиболее опасных сценариев связан с криптоанализом.

Криптоанализ изучает методы, позволяющие восстановить содержимое зашифрованных
данных без знания секретной информации, используемой отправителем и получателем
для зашифрования и расшифрования. Иначе говоря, криптоанализ — это искусство
взлома криптосистем. К сожалению, многие математические задачи, на которых
основана безопасность современных криптографических алгоритмов, окажутся
решаемыми на квантовых компьютерах. Это резко упростит взлом криптографических
схем, широко применяемых в цифровых коммуникациях.

К широко используемым симметричным алгоритмам относятся Advanced Encryption
Standard (AES) и 3DES (Data Encryption Standard). Их стойкость будет фактически
снижена вдвое благодаря алгоритму поиска Гровера \cite{Grover1996}. Асимметричные алгоритмы —
такие как RSA, схема Диффи–Хеллмана (Diffie–Hellman) и криптография на эллиптических
кривых (ECC) — опираются на вычислительную сложность задач факторизации и
дискретного логарифмирования. Квантовые компьютеры способны решать эти задачи
практически мгновенно, что делает безопасность асимметричных криптосистем
крайне уязвимой.

Постквантовая криптография — это направление, изучающее криптосистемы,
устойчивые как к классическим, так и к квантовым атакам. Эти криптосистемы
строятся на задачах, которые, как считается, неразрешимы ни для классических,
ни для квантовых компьютеров. Основные семейства постквантовых криптосистем
включают: криптографию на решётках, криптографию на изогениях, некоммутативную
криптографию, кодовые криптосистемы, схемы на основе хэширования и многомерную
криптографию \cite{Bernstein2009}.

Необходимость разработки более совершенных криптографических алгоритмов привела
к появлению инициатив NIST по созданию квантово-устойчивых стандартов. В рамках
программы стандартизации были собраны заявки и предложения со всего мира, после
чего началось многоэтапное обсуждение и экспертиза для выбора наиболее надёжных
схем.

В данной работе рассматриваются последствия появления квантовых компьютеров для
существующих криптографических алгоритмов, анализируется их устойчивость перед
квантовым противником. Также изучаются основные направления и наиболее
исследованные постквантовые криптосистемы, потенциально устойчивые даже к
квантовым атакам. Завершает обзор анализ наиболее перспективных кандидатов на
включение в стандарт NIST и обсуждение направлений дальнейших исследований в
области постквантовой криптографии.

В 2018 году Мавроэидис (Mavroeidis) и соавт. \cite{Mavroeidis2018} опубликовали работу,
посвящённую влиянию квантовых вычислений на современную криптографию. В ней
рассматривались четыре из шести основных семейств постквантовой криптографии, а
также подробно анализировались подписи, основанные на хэшировании. Настоящая работа
расширяет этот обзор: в ней представлены математические трудные задачи, лежащие
в основе всех семейств постквантовых схем, кратко описаны ключевые
представители каждого семейства, перечислены финалисты NIST PQC, а также
намечены перспективные направления дальнейших исследований.
