\section{Безопасность классической криптографии}

Появление квантовых компьютеров поставило под сомнение безопасность
классических криптографических алгоритмов. В симметричной криптографии
эффективная стойкость (в битах) уменьшается вдвое при наличии квантового
доступа. Асимметричная криптография с появлением квантовых вычислительных
ресурсов фактически утратит свою безопасность.

\subsection{Квантовая стойкость DES}

Алгоритм DES достигает требуемой степени случайности за счёт множественных
раундов преобразований. Однако сравнительно небольшой размер ключа
(56 бит) делает DES уязвимым для полного перебора при современных
вычислительных возможностях. Алгоритм 3DES увеличивает эффективную
длину ключа до 168 бит, жертвуя при этом скоростью шифрования. Увеличение
ключевого пространства до $2^{168}$ обеспечивает стойкость против
классического перебора.

Тем не менее, как и для AES, алгоритм Гровера может быть использован
для квантового поиска по ключу, что снижает необходимое количество
операций до примерно $2^{84}$, что уже не является достаточным уровнем
безопасности. Кроме того, 3DES подвержен коллизиям, что делает его
уязвимым к алгоритму Саймона. 3DES также чрезвычайно медленный и
ресурсоёмкий, поэтому не рассматривается как вариант для широкого
использования.

Модели упрощённого DES (SDES)~\cite{Almazrooie2016SimplifiedDES} в настоящее время
реализуются в квантовых схемах для анализа структуры раунда Фейстеля.
Реализация S-блоков требует сложных компоновок квантовых вентилей,
и их нельзя упростить, поскольку именно они обеспечивают нелинейность
алгоритма. Сложность реализации 3DES в виде квантовых схем могла бы
предположить некоторый уровень стойкости, однако из-за слабости
базового DES он практически не рассматривается.

В целом симметричные криптографические алгоритмы считаются плохо
приспособленными для реализации на квантовых моделях, но при этом
они существенно более устойчивы к квантовому поиску, чем асимметричные
криптосистемы.

\subsection{Квантовая стойкость AES}

AES зарекомендовал себя как один из наиболее надёжных криптографических
алгоритмов, оставаясь устойчивым к исчерпывающим атакам перебором,
которые доступны современным вычислительным системам.

Однако недавний прогресс в области квантовых вычислений вновь поставил
вопрос о стойкости AES перед лицом квантового перебора ключей. Хорошо
известно, что асимметричные криптосхемы — такие как RSA, ECC и другие —
полностью разрушаются под воздействием квантового исчерпывающего поиска
благодаря огромному параллелизму, обеспечиваемому суперпозицией
кубитов~\cite{Bogomolec2019SymmetricPQ}.

Существуют различные алгоритмы, использующие принцип суперпозиции, такие
как алгоритм Саймона, поиск Гровера и алгоритм Шора.

Алгоритм Гровера уменьшает сложность полного перебора ключей с
$O(N)$ до $O(\sqrt{N/M})$, где значение $M$ может быть сведено к 1 при выборе
функций, имеющих единственное решение, что справедливо для реализации
AES в квантовой схеме.

Ключевым недостатком криптоанализа симметричных систем на основе поиска
Гровера является то, что сама криптосистема должна быть реализована в
виде квантовой схемы. Существуют работы, посвящённые реализации AES
в квантовой форме, однако атака на основе алгоритма Гровера требует
наличия квантового оракула, способного выполнять AES. Таким образом,
применимость атаки ограничена квантовыми моделями вычислений.

Кроме того, как отмечалось выше, квантовые алгоритмы позволяют сократить
число испытаний только до $N^{1/2}$, где $N$ — размер ключевого пространства.
Следовательно, стойкость можно сохранить простым увеличением длины
ключа. Переход от AES-128 к AES-256 делает алгоритм практически
непригодным для взлома методом квантового перебора.

\subsection{Квантовая стойкость RSA}

Квантовый алгоритм факторизации Шора представляет собой квантовую
схему, способную разлагать большие числа на простые множители за
полиномиальное время~\cite{Valenta2017PostQuantumRSA}. Это является огромным
ускорением по сравнению с классическими методами, для которых задача
факторизации требует экспоненциального времени. По мере развития
квантовых вычислений задача разложения на множители перестанет быть
трудной, что сделает взлом RSA тривиальным. Первая практическая оценка ресурсов для факторизации RSA требовала
более миллиарда кубитов. Однако к 2019 году этот показатель был
снижен до примерно 20 миллионов кубитов~\cite{GidneyEkera2019}. В частности,
показано, что 2048-битное число RSA может быть разложено менее чем
за 8 часов, используя около 20 миллионов шумных кубитов. Современные
квантовые компьютеры обладают лишь 50–100 кубитами, но в течение
следующих 25 лет программные и аппаратные улучшения могут сделать
факторизацию крупных чисел практической.

Алгоритм Шора использует квантовое преобразование Фурье (QFT),
чтобы воспользоваться квантовым параллелизмом. Задача разложения числа
на множители сводится к задаче поиска периода. В работе авторов реализовано
множество оптимизаций, уменьшающих требуемое число кубитов. Вариант
алгоритма, предложенный Экерой и Хастадом~\cite{GidneyEkera2019}, сокращает число
необходимых операций умножения. Дополнительно используются оптимизации,
такие как оконная арифметика и «oblivious carry runways».

Основная идея алгоритма факторизации Шора описывается следующим образом~\cite{Blanda2014ShorBlog}.  
Пусть $N = pq$, где $p$ и $q$ — простые числа. Чтобы разложить $N$,
выбирается случайное число $a < N$ такое, что $\gcd(a, N) = 1$. Рассматривается
функция $f(x) = a^{x} \bmod N$, и определяется её период $r$ (на квантовой схеме).
Если период $r$ нечётный, алгоритм повторяется с новым значением $a$.
Если $r$ — чётный, то выполняется $f(x) = f(x + r)$, и можно вычислить
делители $N$ как $\gcd(a^{r/2 \pm 1}, N)$. Нахождение периода осуществляется
с помощью квантового преобразования Фурье.

Первая часть алгоритма заключается в инициализации входного и выходного
регистров: входной регистр -- суперпозиция всех значений от 0 до $N-1$,
выходной -- нулевой. Эти регистры затем запутываются посредством
квантовой схемы, реализующей функцию $f(x) = a^{x} \bmod N$.

После вычисления суперпозиции значений функции $f(x)$ над всеми $x$
к входному регистру применяется квантовое преобразование Фурье.
Затем выполняется измерение входного регистра, которое даёт значение $y$.
Поскольку регистры запутаны, состояние выходного регистра коллапсирует
в значение $f(x_{0})$, соответствующее одному из $x$.

Отношение $Y/N$ приводится к несократимой дроби, и знаменатель этой
дроби рассматривается как возможный кандидат на период. Если он
действительно является периодом функции, задача решена; если нет --
проверяются кратные значения. В случае неудачи регистры повторно
инициализируются, и процедура выполняется заново (результат измерения
случаен и может приводить к отличным значениям $y$).

Алгоритм Шора подрывает фундаментальную идею, лежащую в основе RSA.
Следовательно, в постквантовую эпоху требуется замена RSA, и в настоящее
время NIST проводит поиск и стандартизацию новых криптографических
алгоритмов, устойчивых к квантовым атакам.

\subsection{Квантовая безопасность протокола Диффи--Хеллмана}

Протокол обмена ключами Диффи--Хеллмана основан на задаче дискретного логарифмирования. 
Идею алгоритма Шора можно использовать для получения экспоненциального ускорения при решении этой задачи.

Задача дискретного логарифмирования может быть преобразована в задачу нахождения периода двумерной функции. 
Эта задача решается с помощью квантового преобразования Фурье, которое использует квантовый параллелизм для ускорения вычислений.

Пусть $p$ — большое простое число, а $q$ — порождающий элемент группы $\mathbb{Z}_p$. 
Пусть также $Y = q^{k}$. Чтобы взломать протокол, необходимо определить значение $k$. Определим двумерную функцию: $f(x_1, x_2) = q^{x_1} y^{x_2}$. Период этой функции находится из условия: $f(x_1 + w_1,\, x_2 + w_2) = f(x_1, x_2)$. После нахождения периода можно вычислить: $k = - (w_1 / w_2) \bmod p$ \cite{Smith2019PrePostQDH}.

Этот алгоритм является полиномиальным решением для взлома протокола Диффи--Хеллмана. 
Так как безопасность DH основана на задаче дискретного логарифмирования, а эта задача станет разрешимой с появлением мощных квантовых компьютеров, протокол перестанет быть безопасным.

\subsection{Квантовая безопасность на эллиптических кривых}

Криптография на эллиптических кривых (ECC) основана на задаче дискретного логарифмирования 
в группах точек эллиптических кривых. В своей фундаментальной работе Шор опубликовал два 
квантовых алгоритма: один — для факторизации больших чисел, а другой — для решения задачи 
дискретного логарифмирования в произвольной группе.

Все возможные атаки на ECC сводятся к атаке на задачу дискретного логарифмирования. 
На классических компьютерах известны только экспоненциальные алгоритмы её решения. 
Алгоритм Шора является полиномиальным и использует квантовый параллелизм и квантовое 
преобразование Фурье для нахождения периода функции. Это позволяет эффективно решать 
дискретное логарифмирование и, следовательно, взламывать ECC за полиномиальное время.

Так как ECC требует значительно меньших размеров ключей по сравнению с RSA, взлом 
эллиптических кривых станет возможен раньше. Количество кубитов, необходимых для взлома ECC, 
существенно меньше, чем количество кубитов, требуемых для взлома RSA. Поэтому в условиях 
появления полноценного квантового компьютера ECC станет уязвимой раньше RSA.
