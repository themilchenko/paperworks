\section{Стандартизационный процесс постквантовой криптографии NIST}

Как было показано в предыдущих разделах, современные алгоритмы с открытым 
ключом и схемы цифровой подписи не обеспечивают безопасность при наличии 
достаточно мощного квантового компьютера. Последние достижения в области 
квантовых вычислений привели к созданию 53-кубитного квантового компьютера 
компанией Google. Хотя такая машина ещё совершенно не подходит для взлома 
криптографии с открытым ключом, сам факт её существования изменил вопрос 
с «если» на «когда» такие схемы будут сломаны. Google и IBM ведут гонку 
по созданию устойчивых к шуму квантовых компьютеров с большим числом кубитов.

В результате разрабатываются новые алгоритмы, которые не опираются на 
задачу факторизации или другие задачи, уязвимые к параллельным 
квантовым вычислениям. Большинство таких алгоритмов основано на решётках 
и на кодах с исправлением ошибок. Наиболее перспективными считаются 
алгоритмы, использующие труднорешаемые задачи на решётках, поскольку они 
гарантируют безопасность как в худшем, так и в среднем случае.

Учитывая возрастающую необходимость постквантовой защиты, NIST объявил 
мировой конкурс задач и алгоритмов, которые могут заменить 
существующие схемы шифрования с открытым ключом и схемы цифровой подписи.

На конкурс NIST по постквантовой криптографии было подано 69 алгоритмов. 
Лучшие из них предстоит выбрать к 2023 году. Эти алгоритмы будут 
стандартизованы и будут использоваться в течение многих последующих лет. 
Из 69 алгоритмов 20 представляют собой схемы цифровой подписи и 49 — схемы 
шифрования с открытым ключом. После первого раунда были отобраны 26 алгоритмов.

Алгоритмы первого раунда оценивались по трём основным критериям: 
безопасность, стоимость и производительность, а также характеристики 
алгоритма и его реализации \cite{Chen2016NISTIR8105}.

Безопасность являлась ключевым фактором. Алгоритмы должны были обеспечивать 
семантическую стойкость к адаптивным атакам с выбранным шифртестом. 
NIST также учитывал алгоритмы, обладающие семантической стойкостью 
к атакам с выбранным открытым текстом. Схемы цифровой подписи должны 
были обеспечивать невозможность подделки подписей при адаптивной атаке 
с выбранным сообщением. NIST определил пять категорий безопасности 
и представил перечень дополнительных требований (desiderata), включая устойчивость 
к побочным каналам и атакам многократного ключа.

Стоимость и производительность были вторым по значимости аспектом. 
Стоимость включает вычислительную эффективность (скорость работы алгоритма) 
и требования к памяти (размер кода и объём ОЗУ).

Характеристики алгоритма и реализации включали простоту и элегантность 
конструкции, а также гибкость (работу на разных платформах и возможность 
параллелизации).

После второго раунда NIST отобрал 15 алгоритмов. Эти алгоритмы можно 
классифицировать по типу математически сложной задачи, лежащей в основе 
их безопасности.

\subsection{Кандидаты на основе решёток}
\subsubsection{CRYSTALS-KYBER (финалист)}

CRYSTALS-KYBER является финалистом в процессе стандартизации NIST PQC. 
Он относится к семейству примитивов Cryptographic Suite for Algebraic Lattices (CRYSTAL). 
Та же криптографическая платформа используется для создания алгоритма цифровой подписи DILITHIUM. 
Kyber представляет собой механизм инкапсуляции ключей (KEM), соответствующий IND-CCA2 безопасности. 
Алгоритм основан на сложности задачи \emph{module learning-with-errors} (MLWE) \cite{Alagic2020NISTIR8309}.

KYBER предлагается как алгоритм с открытым ключом, обеспечивающий уровни безопасности, 
сопоставимые со схемами AES. NIST рассматривает AES-128, AES-192 и AES-256 как эталоны безопасности 
для уровней 1, 3 и 5 соответственно. Варианты Kyber (kyber-512, kyber-768, kyber-1024) обеспечивают 
сопоставимые уровни безопасности (примерно в пределах \(2^{30}\) от AES-256). 
Алгоритм считается одним из наиболее конкурентоспособных благодаря своей высокой производительности 
по сравнению с другими предлагаемыми криптосхемами. Повышение уровня безопасности достигается 
изменением порядка блочных матриц, используемых в алгоритме.

Из-за высокой скорости работы решёточных систем и меньших размеров открытых ключей такие системы 
представляют собой привлекательный вариант для внедрения \cite{KyberWebsite}. KYBER демонстрирует это, 
используя ключи от 800 байт до 1{,}5 килобайт, в то время как современным схемам, например RSA, 
требуется ключ размером около 2 килобайт.

Так как Kyber основан на MLWE, необходимо защищаться от многих алгебраических атак, применимых к LWE. 
В ранних версиях возникали вопросы о влиянии сжатия открытого ключа на криптоанализ шифртеста, 
но негативных эффектов обнаружено не было. В дальнейших версиях сжатие было удалено, а параметры 
изменены в сторону симметричных примитивов, основанных на SHAKE256, вместо ранее использовавшегося 
SHA3-256. CRYSTALS-KYBER также разделяет программный каркас с CRYSTALS-DILITHIUM (схемой цифровой подписи), 
что делает эту пару более удобной для совместной реализации.

CRYSTALS-KYBER, наряду с другими постквантовыми алгоритмами, внедрён Cloudflare в библиотеку 
Reusable Cryptographic Library. Amazon добавил гибридный режим KYBER в AWS Key Management Service. 
IBM использует комплект KRYSTAL (KYBER и DILITHIUM) в своём, по их утверждению, первом в мире 
ленточном накопителе, защищённом от квантовых атак.

\subsubsection{SABER (финалист)}

SABER — это ещё один решёточный KEM-алгоритм, 
являющийся финалистом процесса стандартизации NIST PQC. 
Он представлен в трёх вариантах: LightSABER, SABER и FireSABER, 
обеспечивающих уровни безопасности 1, 3 и 5 соответственно, 
согласно требованиям NIST. Уровни безопасности LightSABER, SABER и FireSABER 
соответствуют AES-128, AES-192 и AES-256. 
Алгоритм основан на задаче \emph{Module Learning with Rounding} (MLWR), 
которая отличается от MLWE тем, что ошибки вносятся с помощью округления значений. 
Как и в CRYSTALS-KYBER, переход между версиями осуществляется 
изменением размерностей блочных матриц, используемых в алгоритме. 
В отличие от KYBER, SABER применяет умножение без использования NTT, 
что делает его уникальным среди алгоритмов третьего раунда NIST.

SABER использует обучение с округлением (learning with rounding), 
которое не требует выборки ошибок из распределения. 
Это уменьшает потребность в псевдослучайности и упрощает реализацию. 
Кроме того, безопасность схемы опирается всего на один базовый элемент, 
что позволяет легко адаптировать алгоритм к различным уровнем требований по безопасности.

SABER обеспечивает значения core-SVP, сходные с CRYSTALS-KYBER, 
и характеризуется простыми операциями, что делает его значительно проще в реализации. 
SABER работает в константное время, поэтому атаки по времени выполнения 
не влияют на безопасность схемы. Он рассматривается как потенциально 
очень подходящий для анонимных коммуникаций.

Все целые значения вычисляются по модулю 2, что снижает пропускную способность алгоритма; 
в сочетании с LWR это делает сжатие PKE криптографически обоснованным. 
По итогам третьего раунда стандартизации NIST предлагается улучшить SABER, 
проверив его устойчивость к атакующим побочным каналам и устойчивость к неправильному использованию. 
Это один из наиболее перспективных кандидатов на стандартизацию.

\subsubsection{NTRU (финалист)}

NTRU — это решёточный криптосистемный финалист, 
чья безопасность не выводится из сложности задач 
Ring Learning With Errors (RLWE) или Module Learning With Errors (MLWE). 
Это отличает его от других решёточных криптосистем. 
Вместе с Classic McEliece он является одним из самых старых алгоритмов 
среди всех поданных на конкурс. Благодаря возрасту существует обширная 
исследовательская литература, посвящённая атакам и оценке сложности задач, 
что повышает доверие к алгоритму. Текущая заявка NTRU третьего раунда 
представляет собой объединение заявок второго раунда NTRUEncrypt и 
NTRU HRSS-KEM \cite{Alagic2020NISTIR8309}.

Для NTRU отсутствует формальное сведение «в худшем случае → в среднем». 
В нём используются две модели оценки стоимости вычислений. 
Нелокальная модель использует метрику, аналогичную core-SVP, 
которой пользуются другие решёточные схемы. 
Также существует локальная модель, которая даёт более высокие оценки безопасности. 
Нелокальная модель обеспечивает максимум 4-ю категорию безопасности.

Алгоритм работает быстрее и компактнее, чем широко используемый RSA. 
В схеме присутствуют некоторые избыточности, которые можно убрать 
ценой утраты идеальной корректности. Параметризация NTRU исследуется 
с 1990-х годов, поэтому он является одной из наиболее хорошо 
описанных и проверенных криптосистем третьего раунда. 
Дополнительным преимуществом является отсутствие ограничений 
интеллектуальной собственности: большинство патентов, связанных 
со структурами алгоритма, уже истекли.

\subsubsection{CRYSTALS-DILITHIUM}

CRYSTALS\textminus DILITHIUM — это алгоритм цифровой подписи, 
относящийся к криптографическому набору Cryptographic Suite for Algebraic Lattices \cite{Alagic2020NISTIR8309}. 
Сложность алгоритма опирается на задачу коротких целых решений (SIS) 
и задачу модуля ``обучения с ошибками'' (MLWE). 
Его стойкость к атакам с адаптивным выбором сообщений основана 
на трудности решёточных задач над модульными решётками. 
Дизайн схемы основан на подходе Fiat--Shamir with Aborts, 
однако в ней используется равномерное распределение вместо гауссовского. 
Это делает реализацию значительно проще в вычислительном отношении, 
что даёт преимущество по сравнению с её основным конкурентом — FALCON.

DILITHIUM обладает наименьшей суммарной длиной открытого ключа 
и подписи среди решёточных схем цифровой подписи, использующих 
только равномерное распределение \cite{DilithiumWebsite}. 
У DILITHIUM также самые низкие показатели core-SVP 
среди всех решёточных криптосистем третьего раунда, 
и на данный момент он не достигает уровня безопасности NIST Level 5. 
Тем не менее, DILITHIUM демонстрирует хорошие результаты 
в практических экспериментах и доступен для использования 
с различными наборами параметров.

\subsubsection{FALCON (финалист)}

Fast-Fourier Lattice-based Compact Signature over NTRU — 
это решёточная криптосистема. Её безопасность основана 
на трудности задачи кратчайшего целочисленного решения (Shortest Integer Solution, SIS) 
в решётках NTRU. Конструкция алгоритма опирается на решётки NTRU 
и использует ``ловушечный'' алгоритм выборки Fast Fourier sampling. 
Схема представляет собой комбинацию решёток NTRU, 
быстрой выборки по Фурье и структуры GPV 
(решёточные схемы подписи формата ``хэшируй-и-подписывай'').

Существуют доказательства стойкости FALCON 
как в модели случайного оракула (ROM), так и в квантовой модели случайного оракула (QROM). 
Алгоритм требует сложной реализации, включающей 
большое количество операций с плавающей точкой, 
древовидные структуры данных и выборку из дискретных гауссовских распределений.

FALCON обладает минимальными требованиями к пропускной способности 
среди всех схем цифровой подписи. Он обеспечивает высокую скорость 
создания подписи и проверки подписи, однако этап генерации ключей работает медленнее. 
FALCON можно легко интегрировать в существующие протоколы и приложения 
с хорошими показателями производительности. После второго раунда 
алгоритм получил реализацию с постоянным временем выполнения.

NIST включил FALCON в финальный список кандидатов 
в категории цифровых подписей \cite{Alagic2020NISTIR8309}. 
Планируется стандартизировать либо FALCON, либо DILITHIUM. 
Дополнительный анализ необходим в части ошибок, связанных 
с операциями с плавающей точкой, и уязвимостей к побочным каналам. 
Также требуется более детальное изучение используемого выборщика (sampler). 
Алгоритм генерации ключей FALCON использует менее 30\,KB оперативной памяти \cite{FalconWebsite}.

\subsection{Кандидаты криптогафии на основе корректирующих кодов}

\subsubsection{Classic McEliece}

Classic McEliece — самый старый криптосистемный кандидат,
представленный в третьем раунде стандартизации NIST PQC \cite{Alagic2020NISTIR8309}. 
Он основан на криптосистеме McEliece 1979 года, использующей скрытые коды Гоппы. 
Изначальная схема не разрабатывалась под ограничения массового применения 
и обеспечивала безопасность в модели OW-CPA (one-way chosen-plaintext attack), 
то есть противник не может эффективно восстановить сообщение по известным 
открытому ключу и шифртексту при случайном выборе сообщения \cite{ClassicMcElieceWebsite}. 
Текущее представление модифицирует оригинальную систему, 
обеспечивая более эффективную реализацию и CCA-защищённость, 
что частично ослабляет прежнюю OW-CPA стойкость. 
Современный вариант представляет собой объединение схем NTS-KEM 
и Classic McEliece первого раунда.

Classic McEliece уникален тем, что за десятилетия исследований 
не было достигнуто \emph{никакого} прогресса в атаках на эту криптосистему, 
несмотря на рост вычислительных ресурсов. 
Она использует строгие методы преобразования из PKE в IND-CCA2 KEM, 
что обеспечивает стойкость в ROM, а при корректной параметризации — и в QROM. 
Для защиты от атак на хеш-функции применяются хорошо изученные, 
неструктурированные хеш-функции.

Classic McEliece уязвим к атакам декодирования по множеству позиций (Information Set Decoding), 
что устраняется выбором такого числа ошибок, чтобы в шифртексте не возникало 
неискажённых участков сообщения. 
Устойчивость к атакам по сторонним каналам обеспечивается реализацией 
с постоянным временем выполнения.  
Схема также стойка против CCA-атак благодаря хешированию ошибок, 
вносимых в шифртекст; при условии криптостойкости хеш-функций 
этот метод работает существенно медленнее, чем другие возможные атаки.

Classic McEliece генерирует очень маленькие шифртексты — около 256 байт, 
что делает его удобным для внедрения в сетевые протоколы. 
Однако главным недостатком схемы является огромный размер открытого ключа 
— порядка 1{.}5 мегабайт.

\subsubsection{BIKE (Альтернативный кандидат)}

BIKE — это кодовый алгоритм для механизмов инкапсуляции ключей. 
Название является аббревиатурой от \textit{Bit Flipping Key Encapsulation}. 
Алгоритм основан на квазициркульных кодах с матрицей проверок умеренной плотности 
(Quasi-Cyclic Moderate Density Parity-Check). 
Он берёт идеи из схемы McEliece. 
Перед финальной подачей ко второму раунду разработчики существенно изменили алгоритм: 
уменьшили полосу пропускания и представили новый декодер (Black Gray Flip) \cite{BIKEWebsite}.

Безопасность BIKE опирается на сложную задачу кодовой теории кодирования — задачу различения. 
Алгоритм предоставил параметры только для категорий безопасности 1–4, 
но не для категории 5. 
BIKE базируется на предположениях о сложности задач 
Quasi-Cyclic Syndrome Decoding и QCCF. 
При этих предположениях схема IND-CPA-стойкая. 
Кроме того, её IND-CCA-стойкость доказана при условии корректности декодера. 
Устойчивость к атакам определяется через атаки декодированием по множеству позиций 
и оценки их сложности. 
Тем не менее существуют риски атак по сторонним каналам и вопросы CCA-безопасности — 
в первую очередь из-за недостаточной уверенности в новом декодере.

Поскольку BIKE недавно изменила архитектуру, 
полноценно оценить её безопасность по широкому спектру направлений пока невозможно. 
Однако NIST рассматривает её как перспективного кандидата и включил в список 
альтернативных алгоритмов третьего раунда \cite{Alagic2020NISTIR8309}. 
После устранения замечаний по безопасности и повышению доверия к декодеру 
алгоритм может быть стандартизован. 
BIKE рассматривается как потенциально надёжный резервный вариант 
на случай обнаружения уязвимостей в решениях на решётках.

\subsubsection{HQC (Альтернативный кандидат)}

Hamming Quasi-Cyclic (HQC) — это криптосистема, основанная на теории кодирования 
для шифрования с открытым ключом.  
Она базируется на сложности \textit{decisional quasi-cyclic syndrome decoding with parity} —
задачи декодирования синдрома в квазициркульных кодах.  
HQC является IND-CPA-стойкой, и авторы также заявляют её CCA2-стойкость 
\cite{Alagic2020NISTIR8309}.

Разработчики HQC представили новый декодер, основанный на кодах 
Рида–Маллера и Рида–Соломона. Это позволило существенно уменьшить размер ключей. 
Несмотря на сокращение, публичный ключ и шифртест HQC всё ещё 
в 1.6–2 раза и 4–5 раз больше, чем у BIKE соответственно.  
Пропускная способность HQC хуже, чем у BIKE, однако 
генерация ключей и процедуры декапсуляции работают значительно быстрее.  
Алгоритм использует жёстко заданные таблицы для ускорения арифметики в $GF(2^m)$, 
а также имеет декодирование с постоянным временем выполнения.

HQC раньше был уязвим к атаке по стороннему каналу через выбранный шифртест, 
но эта уязвимость была устранена постоянным временем декодирования.  

Основные недостатки HQC включают ненулевую вероятность ошибки при расшифровании,  
крупные шифртесты по сравнению с BIKE и большие открытые ключи.  
Атаки на метрику Хэмминга изучаются уже более 50 лет, что повышает уверенность 
в криптографической стойкости HQC \cite{HQCWebsite}.  
HQC представляет собой эффективный алгоритм с декодированием в постоянное время.  

HQC был выбран как альтернативный кандидат третьего раунда благодаря 
детально проработанным аспектам безопасности среди кодовых алгоритмов.  
Однако он не был включён в финал из-за сравнительно слабых 
показателей производительности.

\subsection{Кандидаты на основе многомерной криптографии}

\subsubsection{Rainbow (финалист)}

Rainbow — это алгоритм цифровой подписи, основанный на варианте
несбалансированной схемы «масло–уксус» (Unbalanced Oil and Vinegar, UOV).
С момента появления в 2005 году схема практически не изменялась.  
Rainbow характеризуется малыми размерами подписей, а также очень быстрой
процедурой подписания и проверки. Многоуровневая структура UOV делает его
естественно устойчивым к атакам по сторонним каналам и защищённым от
классических атак на UOV, таких как атака Кипниса–Шамира.  
Однако усложнение структуры привело к новым возможностям для эксплуатации
алгоритма. После 2008 года новые атаки отсутствовали, и считалось, что
существующие атаки можно нейтрализовать выбором корректных параметров.

В настоящий момент Rainbow соответствует уровням безопасности NIST 1, 3 и 5
и считается NP-трудным \cite{Alagic2020NISTIR8309}.  
Однако после подачи в раунд 3 были опубликованы две атаки Уарда Бёлленса.
Первая атака снижает стойкость Rainbow-1, Rainbow-3 и Rainbow-5 на 7, 4 и 19 бит
соответственно; она основана на модификации классической атаки
Кипниса–Шамира и также затрагивает схему подписи UOV.  
Вторая атака ещё сильнее — она уменьшает стойкость Rainbow-1, Rainbow-3 и
Rainbow-5 на 20, 40 и 55 бит соответственно.  

В условиях появления этих атак маловероятно, что NIST утвердит текущие
параметры схемы. Разработчикам теперь требуется подобрать новые параметры,
обеспечивающие требуемый уровень безопасности.  
Дополнительным недостатком Rainbow являются очень крупные открытые и закрытые
ключи — их размер может достигать 1.8 МБ.

\subsubsection{GeMSS (альтернативный кандидат)}

GeMSS (A Great Multivariate Short Signature) — это схема цифровой подписи,
основанная на парадигме «большого поля». Схема заявляет стойкость уровня
EUF-CMA благодаря универсальной нефиксифицируемости примитива HFEv
(Hidden Field Equations). GeMSS была вдохновлена алгоритмом QUARTZ и
использует современные результаты в области многомерной криптографии
для повышения эффективности и безопасности.

GeMSS опирается на хорошо изученную математическую задачу. Схема обеспечивает
быстрое вычисление проверки подписи и формирует наиболее короткие подписи
среди всех кандидатов. Однако ей требуются большие открытые ключи и длительное
время генерации подписи. Кроме того, схема сложна для реализации на
малопроизводительных устройствах. Крупные размеры открытых ключей затрудняют
её использование в протоколах TLS и SSH.

Авторы GeMSS продолжают работу над повышением производительности алгоритма
\cite{GeMSSWebsite}. Основным конкурентом является Rainbow, который более эффективен
на низкопроизводительных устройствах, обладает хорошей безопасностью и меньшими
размерами ключей. В случае серьёзных уязвимостей в Rainbow именно GeMSS станет
основным кандидатом на стандартизацию \cite{Alagic2020NISTIR8309}. По этой причине NIST включил
GeMSS в список альтернативных алгоритмов.

\subsection{Кандидаты криптографии на основе изогений}

\subsubsection{SIKE (альтернативный кандидат)}

SIKE является единственным алгоритмом, основанным на эллиптических кривых.
Несмотря на то, что алгоритмы на эллиптических кривых легко взламываются
алгоритмом Шора, SIKE решает эту проблему, используя псевдослучайные
суперсингулярные изогенные кривые, которые обеспечивают взаимно-однозначное
отображение точек, а не умножение точки на скаляр. Благодаря близости SIKE
к ECC и протоколам Диффи–Хеллмана, в случае стандартизации он станет одним
из самых простых для внедрения в существующие системы.

SIKE имеет самый маленький размер ключей среди всех алгоритмов третьего раунда
(около 750 байт), даже для высоких уровней безопасности. Благодаря широким
исследованиям эллиптических кривых выбор параметров и защита от
побочных атак значительно проще.

Основным недостатком SIKE является сравнительно небольшое количество
исследований, посвящённых поиску изогеней для эллиптических кривых,
поэтому уверенность в стойкости ниже, чем у других классов алгоритмов.
Кроме того, SIKE примерно на порядок медленнее большинства остальных
кандидатов.

В настоящее время вокруг SIKE продолжаются дискуссии, поскольку были
обнаружены атаки на упрощённые версии алгоритма, что вызывает вопросы
о его долгосрочной стойкости. Однако разработчики отмечают, что
для текущих параметров, представленных в рамках конкурса,
алгоритм сохраняет конкурентный уровень безопасности \cite{Alagic2020NISTIR8309}.

\subsection{Кандидаты на основе хешированных подписей}

\subsubsection{SPHINCS+ (альтернативный кандидат)}

SPHINCS+ — это схема цифровой подписи, основанная на хеш-функциях.  
Она представляет собой усовершенствованную версию SPHINCS и обеспечивает
защиту от мультицелевых атак \cite{SPHINCSplusWebsite}. Её безопасность полностью
сводится к стойкости используемой хеш-функции.

SPHINCS+ принадлежит к классу хешированных схем подписи, изучавшихся ещё
до появления современных криптосистем вроде RSA. В постквантовом мире
SPHINCS+ считается наименее вероятным кандидатом среди всех схем третьего
раунда, для которых возможно появление криптоаналитической атаки.
Алгоритм имеет ясную структуру и чёткие спецификации.

SPHINCS+ уязвим для атак по отказам и атак по побочным каналам. Разработчики
работают над уменьшением этих уязвимостей и улучшением производительности.

Генерация подписи заметно медленнее, чем у других кандидатов, а сами подписи
значительно больше по размеру. Даже самые короткие подписи SPHINCS+
в четыре раза больше подписей DILITHIUM и требуют примерно в тысячу раз
больше вычислительных ресурсов.

Из-за низкой скорости и большого размера подписи интеграция SPHINCS+ в
TLS потребует серьёзной переработки, поскольку нынешние протоколы
ориентируются на схемы с открытым ключом. NIST рассматривает SPHINCS+ как
резервный вариант на случай провала основных кандидатов третьего раунда
\cite{Alagic2020NISTIR8309}. Также он может быть стандартизован для систем,
где требуется максимально высокий уровень безопасности и допустимо снижение
скорости и увеличение размера подписи.

\subsection{Прочие криптографические кандидаты}

\subsubsection{PICNIC}

PICNIC — это алгоритм цифровой подписи, обеспечивающий стойкость как к
квантовым, так и к классическим атакам. Его безопасность не опирается на
какие-либо числовые или алгебраические трудные задачи. Вместо этого PICNIC
основан на системе доказательств с нулевым разглашением и примитивах
симметричной криптографии, таких как хеш-функции и блочные шифры
\cite{PicnicWebsite}. Текущая реализация PICNIC базирует свою безопасность на
безопасности хеш-функции и блочного шифра LowMC.

PICNIC характеризуется большими размерами подписей и медленными
процедурами формирования и проверки подписи. При этом алгоритм обладает
маленькими публичными ключами. Его реализация уязвима к серьёзным атакам
по побочным каналам. PICNIC имеет модульную архитектуру: все его
строительные блоки могут заменяться. Рассматривались варианты схемы,
основанные на AES, но они требуют значительно большего размера ключей.

NIST считает PICNIC ещё недостаточно зрелым \cite{Alagic2020NISTIR8309}.
Его дизайн активно развивается, и алгоритм остаётся концептуально новым.
Существенным преимуществом является гибкость: компоненты схемы можно
подбирать отдельно. PICNIC демонстрирует высокий потенциал развития как в
части повышения производительности, так и в части укрепления безопасности.
NIST включил его в список альтернативных кандидатов.
